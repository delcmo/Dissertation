%%%%%%%%%%%%%%%%%%%%%%%%%%%%%%%%%%%%%%%%%%%%%%%%%%%
%
%  New template code for TAMU Theses and Dissertations starting Fall 2012.  
%  For more info about this template or the 
%  TAMU LaTeX User's Group, see http://www.howdy.me/.
%
%  Author: Wendy Lynn Turner 
%	 Version 1.0 
%  Last updated 8/5/2012
%
%%%%%%%%%%%%%%%%%%%%%%%%%%%%%%%%%%%%%%%%%%%%%%%%%%%

%%%%%%%%%%%%%%%%%%%%%%%%%%%%%%%%%%%%%%%%%%%%%%%%%%%%%%%%%%%%%%%%%%%%%%
%%                           SECTION I
%%%%%%%%%%%%%%%%%%%%%%%%%%%%%%%%%%%%%%%%%%%%%%%%%%%%%%%%%%%%%%%%%%%%%

\pagestyle{plain} % No headers, just page numbers
\pagenumbering{arabic} % Arabic numerals
\setcounter{page}{1}

\chapter{\uppercase {Discretization method and implementation details of the entropy viscosity method.}}
\section{Implementation of the entropy viscosity method with continuous Galerkin finite element method:}
After describing the theoretical approach that leads to the derivation of the dissipative terms consistent with the entropy minimum principle and the definition of the viscosity coefficient, this section focuses on the implementation of the method: details are given on how to compute the jump, the entropy residual and the dissipative terms, for instance. Most of the explanation given in this section can be applied to any schemes other than the Galerkin continuous finite element method. Special attention is required for the jump since their definition is scheme dependent. \\

The dissipative terms consist of second order spatial derivatives that are reduced to first order derivative by integrating per part after multiplying by the test function and integrating over the computational domain. An integral over the boundary of the computational domain also appears and raises the question of how to treat the dissipative terms on the boundary. It is often suitable to set the viscosity coefficients to zero on the boundary so that the boundary term coming from integrating per part the dissipative terms is also zero. Hyperbolic systems of equations alike the multi-D Euler equations, require particular treatment to compute the flux on the boundaries based on the study of the characteristic and the eigenvalues: the boundary terms of the dissipative terms could interfere with the hyperbolic flux and lead to the wrong value at the boundaries. The dissipative terms require to evaluate at each quadrature points of each cell the first-order spatial derivatives of one or multiple variables. With Galerkin finie element method, this task is easily done by using the derivative of the test functions. \\
Once the dissipative terms are implemented, the viscosity coefficient has to be computed: the viscosity coefficient is evaluated on the fly from the entropy residual and the jumps. The viscosity coefficient requires to compute the first-order viscosity and the high-order viscosity coefficients at the quadrature points.