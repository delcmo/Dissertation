%%%%%%%%%%%%%%%%%%%%%%%%%%%%%%%%%%%%%%%%%%%%%%%%%%%
%
%  New template code for TAMU Theses and Dissertations starting Fall 2012.  
%  For more info about this template or the 
%  TAMU LaTeX User's Group, see http://www.howdy.me/.
%
%  Author: Wendy Lynn Turner 
%	 Version 1.0 
%  Last updated 8/5/2012
%
%%%%%%%%%%%%%%%%%%%%%%%%%%%%%%%%%%%%%%%%%%%%%%%%%%%

%%%%%%%%%%%%%%%%%%%%%%%%%%%%%%%%%%%%%%%%%%%%%%%%%%%%%%%%%%%%%%%%%%%%%%
%%                           SECTION I
%%%%%%%%%%%%%%%%%%%%%%%%%%%%%%%%%%%%%%%%%%%%%%%%%%%%%%%%%%%%%%%%%%%%%

\pagestyle{plain} % No headers, just page numbers
\pagenumbering{arabic} % Arabic numerals
\setcounter{page}{1}
%%%%%%%%%%%%%%%%%%%%%%%%%%%%%%%%%%%%%%%%%%%%%%%%%%%
%%%%%%%%%%%%%%%%%%%%%%%%%%%%%%%%%%%%%%%%%%%%%%%%%%%
\chapter{\uppercase {Discretization method and implementation details of the entropy viscosity method.}}
%%%%%%%%%%%%%%%%%%%%%%%%%%%%%%%%%%%%%%%%%%%%%%%%%%%
%%%%%%%%%%%%%%%%%%%%%%%%%%%%%%%%%%%%%%%%%%%%%%%%%%%
%==============================================================================
\section{Implementation of the entropy viscosity method (EVM) with continuous Galerkin finite element method:}
%==============================================================================
After describing the theoretical approach that leads to the derivation of the dissipative terms consistent with the entropy minimum principle and the definition of the viscosity coefficient, this section focuses on the implementation of the method: details are given on how to implement and compute the jump, the entropy residual and the dissipative terms, for instance. Most of the explanation given in this section can be applied to any schemes other than the Galerkin continuous finite element method. Special attention is required for the jump since their definition is scheme dependent. In this section, to explain the steps involved in the implementation of the EVM, we consider a on-uniform $2$-D mesh family $\Omega$. Each member of this family is called element, $e$, and the set of its faces is denoted by $\delta e = \left\{ \delta e_k \right\}$, where $k$ is the number of faces. To integrate the integral over each element $e$ and the boundaries $\delta e$, a quadrature rule, $Q = \left\{ q \right\}$ is used. For the purpose of this section, the EVM is applied to a hyperbolic system of $n$ equations given in \eqt{eq:system}. 
\begin{equation}\label{eq:system}2
\partial_t U(\vec{r},t) + \div F(U(\vec{r},t)) = \div G(U(\vec{r},t))
\end{equation}
where $U(\vec{r},t)$ and $F(U(\vec{r},t))$ are the vector conservative variables and the hyperbolic flux, respectively. The term $G(U(\vec{r},t))$ denotes the artificial dissipative terms that are assumed consistent with the entropy minimum principle \cite{jlg3}. All of the terms depend on space, $\vec{r}$ and time, $t$. It is also assumed that the eigenvalues, $\lambda = \left\{ \lambda_1, \lambda_2, \dots, \lambda_n \right\}$ are known. Lastly, the entropy residual is assumed to be of the following form:
\begin{equation}\label{eq:ent_residual_ex}
\partial_t S \left( U(\vec{r},t) \right) +  \grad \Phi \left( U(\vec{r},t) \right) \geq 0
\end{equation}
where $S$ is the entropy function and $\Phi$ is the associated entropy flux. \\ 
The first step in the implementation of the EVM is the integration of the dissipative terms over each element of the mesh. The continuous finite element approach consists of multiplying each term by a test function and then, integrating over the volume of each element. Since the dissipative terms are second-order spatial derivatives, an integration per part is performed leading to:
\begin{equation}
\div G(U(\vec{r},t)) \rightarrow \int_e \div G(U(\vec{r},t)) \phi(\vec{r}) = \int_{\delta e} G(U(\vec{r},t)) \phi(\vec{r}) - \int_e G(U(\vec{r},t)) \grad \phi(\vec{r}) \nonumber
\end{equation} 
where $\phi$ is a test function. Then, each integral is evaluating using the quadrature rule at the set of quadrature points $\vec{r}_q$:
\begin{equation}
\int_{\delta e} G(U(\vec{r},t)) \phi(\vec{r}) - \int_e G(U(\vec{r},t)) \grad \phi(\vec{r}) = \sum_{q,\delta e} G(U(\vec{r}_q,t)) \phi(\vec{r}_q) - \sum_{q,e} G(U(\vec{r}_q,t)) \grad \phi(\vec{r}_q) \nonumber
\end{equation}
The dissipative term $G(U(\vec{r},t))$ is function of the viscosity coefficient and the derivative of the conservative variables that need to be evaluated at the quadrature points. Obtaining the derivative values at the quadrature points with a continuous finite element discretization type is straightforward. On the other hand, computing the viscosity coefficient at the same quadrature points require a little bit more of computational work and is explained in the following. Details about the integral over the faces of each element will be given in the next paragraph.\\
The viscosity coefficient is not obtained by solving a PDE, but simply computed on the fly from the conservative variables. The definition of the viscosity coefficient $\mu(\vec{r},t)$ involves two other viscosity coefficients: a first-order viscosity coefficient $\mu_{max}(\vec{r},t)$ that is an upper bound, and a high-order viscosity coefficient often also called entropy viscosity coefficient that is denoted by $\mu_e(\vec{r},t)$. It is assumed that these viscosity coefficients are function of both space and time. The first-order viscosity coefficient is function of the local maximum eigenvalues and the mesh size $d$: $\mu_{max}(\vec{r},t) = \frac{d}{2} \max \lambda_k (\vec{r},t)$. For a shape regular mesh, the mesh size $d$ is expected to be finite. It is assumed that each element has its own finite mesh size denoted by $d_e$. For instance, when using libMesh (REF), a function can be called in order to get the mesh size or diameter of the cell under consideration. Obtaining the local maximum eigenvalue is self explanatory and has to be done for each quadrature point. At this point, in each element, $q$ different values of the first-order viscosity coefficient is available to us that we denote by the set: $\mu_{max}^e = \left\{ \mu_{max}(\vec{r}_q,t) \right\}$. \\
The high-order viscosity coefficient is trickier to compute since it involves the entropy residual and the jumps at the interface between cells.   
%The dissipative terms consist of second order spatial derivatives that are reduced to first order derivative by integrating per part after multiplying by the test function and integrating over the computational domain. An integral over the boundary of the computational domain also appears and raises the question of how to treat the dissipative terms on the boundary. It is often suitable to set the viscosity coefficients to zero on the boundary so that the boundary term coming from integrating per part the dissipative terms is also zero. Hyperbolic systems of equations alike the multi-D Euler equations, require particular treatment to compute the flux on the boundaries based on the study of the characteristic and the eigenvalues: the boundary terms of the dissipative terms could interfere with the hyperbolic flux and lead to the wrong value at the boundaries. The dissipative terms require to evaluate at each quadrature points of each cell the first-order spatial derivatives of one or multiple variables. With Galerkin finite element method, this task is easily done by using the derivative of the test functions. \\
%Once the dissipative terms are implemented, the viscosity coefficient has to be computed: the viscosity coefficient is evaluated on the fly from the entropy residual and the jumps. The viscosity coefficient requires to compute the first-order viscosity and the high-order viscosity coefficients at the quadrature points.