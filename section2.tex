%%%%%%%%%%%%%%%%%%%%%%%%%%%%%%%%%%%%%%%%%%%%%%%%%%%
%
%  New template code for TAMU Theses and Dissertations starting Fall 2012.  
%  For more info about this template or the 
%  TAMU LaTeX User's Group, see http://www.howdy.me/.
%
%  Author: Wendy Lynn Turner 
%	 Version 1.0 
%  Last updated 8/5/2012
%
%%%%%%%%%%%%%%%%%%%%%%%%%%%%%%%%%%%%%%%%%%%%%%%%%%%

%%%%%%%%%%%%%%%%%%%%%%%%%%%%%%%%%%%%%%%%%%%%%%%%%%%%%%%%%%%%%%%%%%%%%%
%%                           SECTION I
%%%%%%%%%%%%%%%%%%%%%%%%%%%%%%%%%%%%%%%%%%%%%%%%%%%%%%%%%%%%%%%%%%%%%
%%%%%%%%%%%%%%%%%%%%%%%%%%%%%%%%%%%%%%%%%%%%%%%%%%%
%%%%%%%%%%%%%%%%%%%%%%%%%%%%%%%%%%%%%%%%%%%%%%%%%%%
\chapter{\uppercase {Discretization method and implementation details of the entropy viscosity method.}}
%%%%%%%%%%%%%%%%%%%%%%%%%%%%%%%%%%%%%%%%%%%%%%%%%%%
%%%%%%%%%%%%%%%%%%%%%%%%%%%%%%%%%%%%%%%%%%%%%%%%%%%
%==============================================================================
\section{Spatial and Temporal Discretization Method:}
%==============================================================================
Because the stabilization of the numerical solution of hyperbolic equation systems is intimately connected with the temporal and
spatial discretization, before discussing the entropy viscosity stabilization method used in RELAP-7, a short summary of the
numerical methods it employs is given.
%
\subsection{Spatial Discretization Algorithm\label{sec:spatial_discretization}}
%
RELAP-7 currently employs the continuous finite element method implemented
via the INL MOOSE framework.  For simplicity of exposition this section focuses on
the weak statement associated to the strong form of the one-dimensional, variable
cross-sectional area form of the Euler equations~\eqref{E-S:18}--\eqref{E-S:20}
summarized in Section~\ref{sec:single_phase_equations}.  Start by writing the
equations in ``vector'' form as
\begin{align}
  \label{eqn:va_system_notation}
  \vec{R}(\vec{U}) \equiv \frac{\partial \vec{U}}{\partial t} + \frac{\partial \vec{F}}{\partial x} - \vec{S} = \vec{0}
\end{align}
where
\begin{align}
  \vec{U} \equiv
  \begin{bmatrix}
    \rho A
    \\
    \rho u A
    \\
    \rho E A
  \end{bmatrix},
  \qquad
  \vec{F} \equiv
  \begin{bmatrix}
    \rho u A
    \\
    \left(\rho u^2 + p\right) A
    \\
    \rho u H A
  \end{bmatrix}
\end{align}
and $\vec{S} = \vec{S}(\vec{U})$ consists of the remaining source
terms.  Note that $\vec{U}$ and $\vec{F}$ are identical to their
meanings in the ``constant-area'' equations, up to multiplication by
the area, $A$.
%
The variational statement proceeds by
dotting~\eqref{eqn:va_system_notation} by an ``admissible'' vector test function
$\vec{W}$ (more details of which will be subsequently given),
integrating over the domain $\Omega$, and applying the divergence
theorem.  Solutions $\vec{U}$ are sought such that
\begin{align}
  \label{eqn:va_system_weak}
  \int_{\Omega} \left(\frac{\partial \vec{U}}{\partial t} \cdot \vec{W} - \vec{F}\cdot \frac{\partial \vec{W}}{\partial x} - \vec{S}\cdot\vec{W} \right) \text{d}\Omega
  + \int_{\Gamma} \left( \vec{F} \cdot \vec{W} \right) \vec{n}_x \;\text{d}\Gamma = 0
\end{align}
holds for \emph{all} admissible $\vec{W}$.
%(RAB--I don't think the following two sentences are valid!!!)
%Note that~\eqref{eqn:va_system_weak} remains in a ``continuous''
%setting, that is, we have not yet introduced a mesh or finite element
%discretization.  Equation~\eqref{eqn:va_system_weak} is merely a
%``weak'' restatement of the ``strong'' equations~\eqref{eqn:va_system_notation}.
%
Note that the test function $\vec{W}$ is not chosen arbitrarily.
In particular, it is required that $\vec{W}$ come from the space of
vector functions
\begin{align}
  \label{eqn:W_set}
  \vec{W} \in \left\{
      \begin{bmatrix}
        w \\ 0 \\ 0
      \end{bmatrix},
      \begin{bmatrix}
        0 \\ w \\ 0
      \end{bmatrix},
      \begin{bmatrix}
        0 \\ 0 \\ w
      \end{bmatrix}
    \right\}
\end{align}
where $w \in \mathcal{W}$ is a scalar test function.  In the present
work, and in general practice, the space $\mathcal{W}$ is taken to be
(a subspace of) the Hilbert space $H^1(\Omega)$. This choice, for
instance, guarantees enough smoothness that~\eqref{eqn:va_system_weak}
makes sense.
%
The approximate problem then proceeds by selecting only test functions
from a finite-dimensional subspace of $\mathcal{W}$, denoted by
$\mathcal{W}^h$, which is spanned by the basis $\{\phi_i\}$,
$i=1,\ldots,N$.  We then seek $\vec{U}^h$ with components in the same
space as $\mathcal{W}^h$, satisfying the boundary conditions, and such
that
\begin{align}
  \label{eqn:va_system_weak_h}
  \int_{\Omega} \left(\frac{\partial \vec{U}^h}{\partial t} \cdot \vec{W}^h
    - \vec{F}^h\cdot \frac{\partial \vec{W}^h}{\partial x} - \vec{S}^h\cdot\vec{W}^h \right) \text{d}\Omega
  + \int_{\Gamma} \left( \vec{F}^h \cdot \vec{W}^h \right) \vec{n}_x \;\text{d}\Gamma = 0
\end{align}
holds for all $\vec{W}^h$ defined analogously to~\eqref{eqn:W_set},
with components in $\mathcal{W}^h$. Note that~\eqref{eqn:va_system_weak_h} 
has been placed in a ``continuous'' setting, that is, a mesh and finite element
discretization has been introduced requiring a continuous solution.  
Equation~\eqref{eqn:va_system_weak_h} remains a ``weak'' restatement of the ``strong'' 
equations~\eqref{eqn:va_system_notation} in the sense that derivatives of the  
solution and its flux need not be continuous. More will be said of this subsequently,
in the upcoming section on stabilization methods.  Written out in component form,
and denoting the components of $\vec{U}^h$ by $U_0^h$, $U_1^h$, and
$U_2^h$, \eqref{eqn:va_system_weak_h} expands to:
\begin{align}
  \label{eqn:1Dmass_weak}
  &
  % F_i^{(U_0)} \equiv
  \int_{\Omega} \left( \frac{\partial U_0^h}{\partial t} \phi_i -
    U_1^h \frac{\partial \phi_i}{\partial x} \right) \text{d}{\Omega}
  + \int_{\Gamma} U_1^h \vec{n}_x \phi_i \; \text{d}{\Gamma} = 0
  \\[12pt]
  % Momentum equation
  \nonumber
  &
  % F_i^{(U_1)} \equiv
  \int_{\Omega} \left[ \left(\frac{\partial U_1^h}{\partial t}
      - U_0^h g_x + \frac{f}{2 d_h}  U_1^h \left|\frac{U_1^h}{U_0^h}\right| - p^h \frac{\partial A}{\partial x} \right) \phi_i
    - \left( \frac{(U_1^h)^2}{U_0^h} + p^hA \right) \frac{\partial\phi_i }{\partial x} \right] \text{d}{\Omega}
  \\
  \label{eqn:1Dmomentum_weak}
  &\qquad + \int_{\Gamma} \left( \frac{(U_1^h)^2}{U_0^h} + p^hA \right) \vec{n}_x \phi_i \; \text{d}{\Gamma} = 0
  \\[12pt]
  % Energy equation
  \nonumber
  &
  % F_i^{(U_2)} \equiv
  \int_{\Omega}
  \left[
  \left(
    \frac{\partial U_2^h}{\partial t} + H_w a_w (T^h-T_w) A - U_1^h g_x
  \right)
  \phi_i
    -  U_1^h H^h \frac{\partial \phi_i}{\partial x}
  \right]
  \text{d}{\Omega}
  \\
  \label{eqn:1Denergy_3eq_weak}
  &\qquad
  +
  \int_{\Gamma} U_1^h H^h \vec{n}_x \phi_i \; \text{d}{\Gamma} = 0
\end{align}
Equations~\eqref{eqn:1Dmass_weak}--\eqref{eqn:1Denergy_3eq_weak} must
hold for $i=1,\ldots,N$.  Note that the approximate pressure, $p^h$,
temperature, $T^h$, and enthalpy, $H^h$ are functions of the conserved
variables $U_0^h, U_1^h, U_2^h$.  As mentioned, a continuous
Galerkin formulation is employed, and therefore the unknowns are expressed in
the same basis used for the test functions, i.e.
\begin{align}
  U_0^h &= \sum_j (U_0)_j \phi_j
  \\
  U_1^h &= \sum_j (U_1)_j \phi_j
  \\
  U_2^h &= \sum_j (U_2)_j \phi_j
\end{align}
The coefficients $(U_0)_j$, $(U_1)_j$, and $(U_2)_j$ vary in time only, and comprise the
solution vector at each iteration.  Note
that~\eqref{eqn:1Dmass_weak}--\eqref{eqn:1Denergy_3eq_weak} are
so-called ``semi-discrete'' equations: they have been discretized in
space, but the temporal derivatives remain in continuous form.  In
Section~\ref{sec:time_integration} the various time discretization methods
employed in RELAP-7, to approximately time-integrate the equations,
are summarized.  Furthermore, it is well-known
that a continuous Galerkin discretization of this set of hyperbolic
equations is equivalent to a central difference method for a certain
choice of integration rule, and therefore will exhibit oscillatory
instabilities unless some artificial diffusion is added to stabilize
the method.  In Section~\ref{sec:entropy_viscosity}, stabilization will be
discussed further, and in particular the new entropy viscosity 
stabilization scheme will be presented in detail.
%
\subsection{Backward Euler\label{sec:backward_euler}}
%
The backward Euler method~\cite{Butcher_2003} is a well-known, first-order, A-stable
implicit time integration method.  Given a generic
semi-discrete equation in a form similar to~\eqref{eqn:1Dmass_weak}--\eqref{eqn:1Denergy_3eq_weak},
\begin{align}
  \int_{\Omega} \left(\frac{\partial u^h}{\partial t} + G(u^h) \right) \phi_i \; \text{d}{\Omega} = 0
\end{align}
the backward Euler method results in the temporal discretization
\begin{align}
  \label{eqn:backward_euler_fully_discrete}
  \int_{\Omega} \left( \frac{u^{n+1} - u^n}{\Delta t} + G(u^{n+1}) \right) \phi_i \; \text{d}{\Omega} = 0
\end{align}
where $\Delta t$ is the timestep, $t^{n+1} = t^n + \Delta t$, and $u^n
\equiv u^h(t^n)$ is a shorthand notation used to refer to the finite
element solution at time level
$n$. Equation~\eqref{eqn:backward_euler_fully_discrete} is a
fully-discrete (possibly nonlinear) equation which must be satisfied
for each $i$.

Note that the backward Euler method, when applied to the linear convection equation
\begin{align}
  \label{eqn:convection}
  \frac{\partial u}{\partial t} + a \frac{\partial u}{\partial x} = 0
\end{align}
yields a leading-order truncation error term of the form
\begin{align}
  \nonumber
  \left. \frac{\partial u}{\partial t} \right|_{t^{n+1}}  &=
  \frac{u^{n+1} - u^n}{\Delta t} + \frac{\Delta t}{2}\left.\frac{\partial^2 u}{\partial t^2}\right|_{t^{n+1}} + \mathcal{O}(\Delta t^2)
  \\
  \label{eqn:truncation}
  &= \frac{u^{n+1} - u^n}{\Delta t} + \frac{a^2 \Delta t}{2}\left.\frac{\partial^2 u}{\partial x^2}\right|_{t^{n+1}} + \mathcal{O}(\Delta t^2)
\end{align}
where~\eqref{eqn:truncation} follows from differentiating the
continuous equation~\eqref{eqn:convection} with respect to time:
\begin{align}
  \label{eqn:second_wave}
  \frac{\partial^2 u}{\partial t^2} = -a \frac{\partial }{\partial t} \left( \frac{\partial u}{\partial x} \right)
  = -a \frac{\partial }{\partial x} \left( \frac{\partial u}{\partial t} \right)
  = -a \frac{\partial }{\partial x} \left( -a \frac{\partial u}{\partial x} \right)
  = a^2 \frac{\partial^2 u}{\partial x^2} \,\,.
\end{align}
Rearranging terms in~\eqref{eqn:truncation} and adding $a\frac{\partial u}{\partial x}$ to both sides allows us to write
\begin{align}
  \label{eqn:truncation_rearranged}
    \frac{u^{n+1} - u^n}{\Delta t}
    + a \frac{\partial u}{\partial x}
    =
    \frac{\partial u}{\partial t}
    + a \frac{\partial u}{\partial x}
    - \frac{a^2 \Delta t}{2} \frac{\partial^2 u}{\partial x^2}
    + \mathcal{O}(\Delta t^2)
\end{align}
where all the continuous derivatives are assumed to be evaluated at
time level $t^{n+1}$.  Thus, the semi-discrete form of the linear convection on
the left-hand side of~\eqref{eqn:truncation_rearranged} is equal to
the continuous parabolic partial differential equation on the
right-hand side, which includes ``artificial'' diffusion or viscosity of
$\mathcal{O}(\frac{a^2 \Delta t}{2})$, to within $\mathcal{O}(\Delta
t^2)$.
% This in turn implies that the solution to the semi-discrete equation
% \begin{align}
%     \frac{u^{n+1} - u^n}{\Delta t}
%     + a \frac{\partial u}{\partial x} = 0
% \end{align}
% coincides with the solution of a ``diffuse'' version of the original
% equation to within $\mathcal{O}(\Delta t^2)$.
For this reason, we often say that the backward Euler time
discretization is inherently stabilizing for the hyperbolic
equation~\eqref{eqn:convection}.  Obviously, the artificial viscosity for the
complete scheme is a composite of the artificial viscosity of both the time
and spatial discretization. 

The backward Euler time integration method may generate excessive artificial
viscosity and should, therefore, only be used for transients with RELAP-7
as an initial scoping calculation, or if only the steady-state solution is of
interest. For accurate transient solutions with RELAP-7, the BDF2 time
integration method, described next, is highly recommended because it is
a second-order (in time) discretization.
%
\subsection{BDF2\label{sec:bdf2}}
%
The backward differentiation formula (BDF) is a family of implicit
methods for numerically integrating ordinary differential equations.
Some notable members of this family include BDF1, which is equivalent
to the backward Euler~\cite{Ascher_1998} method discussed in
Section~\ref{sec:backward_euler}, and BDF2, which is the highest-order
BDF method which is still A-stable.  For fixed step-size $\Delta t$, the BDF2 method 
applied to the ordinary differential equation
\begin{align}
  \frac{\partial u}{\partial t} &= f(t, u)
  \\
  u(t=0) &= u_0
\end{align}
yields the update step:
\begin{align}
\label{eqn:bdf2}
u^{n+1} = \frac{4}{3}u^n - \frac{1}{3}u^{n-1} + \frac{2}{3} \Delta t f(u^{n+1}, t^{n+1})
\end{align}
Dividing through by $\frac{2}{3} \Delta t$, equation~\eqref{eqn:bdf2} can be 
alternatively written as
\begin{align}
\label{eqn:bdf2_scaled}
\frac{\frac{3}{2}u^{n+1} - 2u^n  + \frac{1}{2}u^{n-1}}{\Delta t} = f(u^{n+1}, t^{n+1})
\end{align}
The left-hand side of~\eqref{eqn:bdf2_scaled} can be interpreted as a
backward-difference approximation to the continuous time derivative
$\frac{\partial u}{\partial t}$, and may be employed in a manner analogous
to~\eqref{eqn:backward_euler_fully_discrete} to derive a fully-discrete
system of equations:
\begin{align}
  \label{eqn:bdf2_fully_discrete}
  \int_{\Omega} \left( \frac{\frac{3}{2}u^{n+1} - 2u^n  + \frac{1}{2}u^{n-1}}{\Delta t} + G(u^{n+1}) \right) \phi_i \; \text{d}{\Omega} = 0
\end{align}
based on the semi-discrete equations~\eqref{eqn:1Dmass_weak}--\eqref{eqn:1Denergy_3eq_weak}.
Since BDF2 requires two old timesteps, the method must be
``bootstrapped'' by a lower-order method, such as backward Euler, when
starting.  This means that a much smaller time step size should be used
for start-up, at the beginning of a transient.  The BDF2 method is 
recommended for most transient simulations with RELAP-7.
%
\subsection{Jacobian-Free Newton Krylov Solver\label{sec:jfnk}}
%
The RELAP-7 code solves coupled multi-physics problems using the
Jacobian-Free Newton Krylov (JFNK) approach via the MOOSE
framework. Field equations solved in the current RELAP-7 code include
PDEs to describe one-dimensional fluid flow in pipe systems and heat
conduction in solids, as well as ODEs to describe physics in
zero-dimensional components and the point kinetics equations.

The JFNK method is a fully-coupled, multi-level method for solving large
nonlinear equation systems. In general, it consists of at least two
levels: the outer Newton loop for the nonlinear solve and the inner
Krylov loop for the linear systems of equations associated to Newton
iteration.  The JFNK method has become an increasingly popular option
for solving large nonlinear equation systems arising from
multi-physics problems over the last 20 years, and has branched out
into a number of different disciplines~\cite{Knoll_2004}.

In what follows, a brief description of the JFNK method as it
applies to the RELAP-7 application is given.  The FEM-discretized field
equations are first written as
\begin{equation}
  \label{eqn:F}
  \vec{F} (\vec{u}) = \vec{0}
\end{equation}
where $\vec{F}$ represents the nonlinear equation system and $\vec{u}$
is the solution vector. Newton's method requires an initial guess,
$\vec{u}^0$, to start the iteration process. For the transient
problems of interest here, the solution at a previous time step is
generally used as the initial guess for the method. At the $k^{th}$
iteration, the residual vector is defined as
\begin{align}
  \vec{r}^k \equiv \vec{F}(\vec{u}^k) \,.
\end{align}
Clearly if $\vec{u}^k$ satisfies~\eqref{eqn:F} \emph{exactly}, the
$k^{th}$ residual will be zero.  To update the solution vector, the
following equation is solved for the update vector, $\delta
\vec{u}^{k+1}$:
\begin{equation}
  \label{eqn:newton_correction}
%  \tensor{J}(\vec{u}^k) \delta \vec{u}^{k+1} = - \vec{r}^k
J (\vec{u}^k) \delta \vec{u}^{k+1} = - \vec{r}^k
\end{equation}
where $J(\vec{u}^k)$ is the Jacobian
matrix evaluated at $\vec{u}^k$.  In index notation,
\begin{equation}
  \label{eqn:jacobian_matrix}
  J_{ij} \equiv \frac {\partial F_i} {\partial u_j} \,\,.
\end{equation}
After $\delta \vec{u}^{k+1}$ is obtained, the $(k+1)^{st}$ solution iterate
is computed by
\begin{equation}
  \vec{u}^{k+1} = \vec{u}^{k} + \delta \vec{u}^{k+1} \,\,.
\end{equation}
The Newton iteration is terminated when one of the following conditions is met:
\begin{enumerate}
\item The residual vector norm, $|\vec{r}^k|$, is sufficiently small.
\item The relative residual vector norm $\frac{|\vec{r}^k|}{|\vec{r}^0|}$  is sufficiently small.
\item The step size norm, $|\delta \vec{u}^{k+1}|$ is sufficiently small.
\end{enumerate}

Note that~\eqref{eqn:newton_correction} represents a large
linear system of equations.  In the JFNK method, we need not
explicitly form the matrix $J$: only its action on a vector
(via matrix-vector product) is required.
Effective preconditioning is generally required for Krylov subspace
methods to be efficient, i.e., for the method to converge in a
reasonable number of iterations. A preconditioned version of
equation~\eqref{eqn:newton_correction} can be expressed as (using
right preconditioning as an example),
\begin{equation}
  \label{eqn:newton_correction_preconditioned}
%  \tensor{J}^k \tensor{P}^{-1} \left(\tensor{P} \delta \vec{u}^{k+1} \right)= -\vec{r}^k
  J^k P^{-1} \left(P \delta \vec{u}^{k+1} \right)= -\vec{r}^k
\end{equation}
where $P$ is the preconditioning matrix. In the approach current
used in RELAP-7, an analytical Jacobian matrix is computed according
to~\eqref{eqn:jacobian_matrix}, and passed to the underlying numerical solver
library as the matrix $P$ for preconditioning purposes.
%==============================================================================
\section{Hyperbolic system of equation and boundary conditions:}
\begin{itemize}
\item describe general way for bc: waves, characteristic equations, variables, ...
\item Look at Burger's equations.
\item Do multi-D Euler equations.
\end{itemize}
%==============================================================================
%==============================================================================
\section{Implementation of the entropy viscosity method (EVM) with continuous Galerkin finite element method:}
%==============================================================================
After describing the theoretical approach that leads to the derivation of the dissipative terms consistent with the entropy minimum principle and the definition of the viscosity coefficient, this section focuses on the implementation of the method in a Galerkin continuous finite element scheme. Details are given on how to implement and compute the jump, the entropy residual and the dissipative terms, for instance. Special attention is required for the jump since their definition is scheme dependent. A non-uniform $2$-D mesh family $\Omega$ is considered. Each member of this family is called element, $k$, and the set of its faces is denoted by $\delta k = \left\{ \delta k_j \right\}$, where $j$ is the number of faces. To integrate the integral over each element $e$ and the boundaries $\delta k$, a quadrature rule, $Q = \left\{ q \right\}$ is used.

For academic purpose, the multi-D Burger's equations are considered and recalled here along with the definition of the viscosity coefficients \eqt{eq:system_sct2}: 
\begin{subequations}
\label{eq:system_sct2}
\begin{equation}\label{eq:bg_sct2}
\partial_t u(\vec{r},t) + \div \left[ \vec{n} \frac{u(\vec{r},t)^2}{2} \right] = \div \left( \mu(\vec{r},t) \grad u(\vec{r},t) \right) = \div \vec{g}
\end{equation}
%
\begin{equation}\label{eq:res_sct2}
R_e(\vec{r},t) = \partial_t \left( s(\vec{r},t)\right) +  \grad \left( \vec{n} \Phi(\vec{r},t) \right) \leq 0
\end{equation}
%
\begin{equation}\label{eq:visc_sct2}
\mu(\vec{r},t) = \max \left( \mu_e(\vec{r},t), \mu_{max}(\vec{r},t) \right)
\end{equation}
%
\begin{equation}\label{eq:visc_max_sct2}
\mu_{max}(\vec{r},t) = \frac{h}{2} |u(\vec{r},t)|
\end{equation}
%
\begin{equation}\label{eq:visc_ev_sct2}
\mu_{e}(\vec{r},t) = h^2 \frac{\max\left( R_e(\vec{r},t), J \right)}{|| s - \bar{s} ||_\infty}
\end{equation}
\end{subequations}
where $u(\vec{r},t)$ is a conservative variable that depends on both space and time. The entropy function $s$ and the conservative flux in the entropy residual $R_e$ are defined as $s(\vec{r},t) = \frac{u(\vec{r},t)}{2}$ and $\Phi = \frac{u(\vec{r},t)^3}{3}$, respectively. The Burger's equation is known to admit an unique eigenvalue $\lambda = u(\vec{r},t)$. The vector $\vec{n}$ has the same definition as in \sect{sec:section2}. The jump $J$ is assumed piecewise constant and details regarding its evaluation are given later in this SECTION. The normalization parameter $|| s - \bar{s} ||_\infty$ used in \eqt{eq:visc_ev_sct2} denotes the infinite norm over the entire computational domain of the quantity $s - \bar{s}$ where $\bar{s}$ is the average entropy over the computational domain as well.\\
The first step in the implementation of the EVM is the integration of the dissipative terms over each element of the mesh. The continuous finite element approach consists of multiplying each term by a test function and then, integrating over the computational domain. Since the dissipative terms are second-order spatial derivatives, an integration per part is performed leading to:
%
\begin{equation}\label{eq:diss_term_sct2}
\div \vec{g} \rightarrow \int_\Omega W^h \cdot \div \vec{g}^h\text{d}\Omega  = \int_\Omega \grad W^h \cdot \vec{g}^h\text{d}\Omega - \int_\Gamma \hat{n} \cdot \vec{g}^h W^h \text{d}\Gamma
\end{equation} 
%
In \eqt{eq:diss_term_sct2}, the integral over the domain $\Omega$ is transformed into a sum over the elements and evaluated by using the quadrature rule. The other term, consists of an integral over the boundary of the computational domain $\Gamma$. It requires to compute the dissipative flux $g$, and thus, the viscosity coefficient $\mu$ at the boundary. The evaluation of the integral over $\Gamma$ can be avoided by assuming that the viscosity coefficient is zero at the boundary which yields:
 %
\begin{equation}\label{eq:diss_term2_sct2}
\div \vec{g} \rightarrow \int_\Omega W^h \cdot \div \vec{g}^h\text{d}\Omega  = \int_\Omega \grad W^h \cdot \vec{g}^h\text{d}\Omega
\end{equation} 
%
The dissipative term $\vec{g}^h$ is function of the viscosity coefficient and the derivative of the conservative variable $u$ that need to be evaluated at the quadrature points. Obtaining the derivative values at the quadrature points with a continuous finite element discretization type is straightforward. On the other hand, computing the viscosity coefficient at the same quadrature points require a little bit more of computational work and is explained in the following. \\

The next step consists of determining the viscosity coefficient $\mu$ that is not obtained by solving a PDE, but computed on the fly from the definition recalled in \eqt{eq:visc_sct2}. The definition of the viscosity coefficient $\mu(\vec{r},t)$ involves two other viscosity coefficients: a first-order viscosity coefficient $\mu_{max}(\vec{r},t)$ that is an upper bound, and a high-order viscosity coefficient often also called entropy-viscosity coefficient that is denoted by $\mu_e(\vec{r},t)$. A common element to the definition of $\mu_{max}(\vec{r},t)$ and $\mu_e(\vec{r},t)$ is the mesh size $h$ that can vary through the computational domain and is defined as the shortest distance between two nodes of an element. Thus, when considering a $1$-D mesh with linear test function, the local mesh size is simply $\Delta x$. For a shape regular mesh, the mesh size is finite and usually available through a function call. For instance, when using libMesh (REF), a function can be called in order to get the mesh size or diameter of the cell under consideration. Once the mesh size $h$ is available, it remains to compute the local maximum eigenvalue, the residual $R_e$ and the jump $J$. \\
Obtaining the local maximum eigenvalue is self explanatory using the test functions, $u^q_k = \sum_j u_{k,j} \phi^q_j$ , and has to be done for each quadrature point in a given element $k$: at this point, the first-order viscosity coefficient $\mu_{max}$ is available at each quadrature point and is referred to as $\mu_{max,k}^q = \frac{h_k}{2} | u^q_k |$. The high-order viscosity coefficient is trickier to compute since it involves the entropy residual $R_e$ at the quadrature points and the jumps $J$ at the interface between cells. The entropy residual $R_e$ is a PDE but is not discretized in a finite element sense. Instead, each term of the entropy residual is locally computed using the test functions but without integration over the computational domain as follows:
%
\begin{equation}\label{eq:res_disc_sct2}
R_{e,k}^{q,n} = w_0 s^{n,q}_k + w_1 s^{n-1,q}_k + w_2 s^{n-2,q}_k +  \vec{n} \cdot  \sum_j \Phi_{k,j}\grad \phi^q_j
\end{equation}
when considering three successive solutions $s^{n,q}_k$, $s^{n-1,q}_k$ and $s^{n-2,q}_k$ in time. The weights $w_0$, $w_1$ and $w_2$ are defined in (SECTION). The values of the entropy function $s$ at the quadrature points is computed using the test function: $s^q_k = \sum_j s_{k,j} \phi^q_j = \sum_j \frac{u_{k,j}^2}{2} \phi^q_j$, which requires to access the values of $u^2$ at the nodes $j$. The same method is used for the conservative flux $\Phi$. It is noted that the entropy residual can be recast under a non-conservative form as shown in \eqt{eq:res_disc2_sct2} that can be easier to evaluate depending on what is available to the user.
%
\begin{equation}\label{eq:res_disc2_sct2}
R_e(\vec{r},t) = \partial_t \left( \frac{u(\vec{r},t)^2}{2} \right) +  u(\vec{r},t)^2 \grad \left( \vec{n} u(\vec{r},t) \right)
\end{equation}
%
It remains, now, to compute the jump $J$ that is set constant in each element. In Galerkin continuous finite elements, the variables are continuous at the faces, but their derivative are discontinuous. Thus, the jump of the gradient of a variable to choose, seems to be a good entropy production indicator since it will inform us on the presence of a sharp discontinuity. In the remaining of this section, a generic method is detailed to compute the jump of the gradient of a variable when using Galerkin finite element method. Then, the jump used in the definition of the viscosity coefficient for solving Burger's equation is given. 

To be more specific, let us consider an element $k$ and its set of $n$ boundaries $\delta k = \left\{ \delta k_1, \cdots, \delta k_n \right\}$. We also assume that the outward normal $\hat{n}_i$ to each boundary $\delta k_i$ is available to us. The objective is to compute the jump $J_k$ of the gradient of the variable $v(\vec{r},t)$ for the element $k$. Since an element $k$ shares $n$ boundaries with $n$ other elements of the computational domain, a jump $J_{k.i}$ can be computed for each boundary $\delta k_i$, and is defined as follows:
%
\begin{equation}\label{eq:jump_def_sct2}
J_{k,i} = | \left( \grad v(\vec{r},t)_{k,i} - \grad v(\vec{r},t)_{neighbor,i} \right) \cdot \hat{n}_i |
\end{equation}
% 
where the quantity $\grad v(\vec{r},t)_{neighbor,i}$ denotes the gradient of $v(\vec{r},t)$ in the neighbor cell to the element $k$ sharing the interface $\delta k_i$. The difference of gradients between the two elements sharing the interface $\delta k_i$ is multiplied by the outward normal vector $\hat{n}_i$ to obtain the jump normal to the interface. Once all the jumps $J_{k,i}$ are computed for each face $i$ of the element $k$ (a loop over the faces $i$ of element $k$ applies), the jump $J_k$ is computed by choosing the maximum over the $J_{k,i}$:
%
\begin{equation}\label{eq:jump_def2_sct2}
J_k = \max_i \left( J_{k,i} \right)
\end{equation}
%
With the definition given in \eqt{eq:jump_def2_sct2}, the jump $J_k$ is constant in each element $k$ of the computational domain $\Omega$. From this point, the entropy residual $R_e$ and the jump $J$ are known in the element $k$, at a given time $n$ and at every quadrature points $q$. It remains to compute the normalization parameters $|| s - \bar{s} ||_\infty$ that is obtained from a post processing for every new non-linear iteration of the solver and thus is a function of time. The average value of the entropy function of the computational domain is computed from an integral as follows:
%
\begin{equation}\label{eq:sbar_sct2}
\bar{s} = \frac{1}{\Omega} \int_{\Omega} s(\vec{r},t) \text{d}\Omega
\end{equation}
%   
The high-order viscosity coefficient $\mu_{e,k}^{q,n}$ can now be computed at a given quadrature potions $q$ and given time $n$:
%
\begin{equation}\label{eq:visc_ev2_sct2}
\mu_{e,k}^{q,n} = h_k^2 \frac{\max \left( R_{e,k}^{q,n}, J_k^n \right)}{|| s - \bar{s} ||_\infty^n}
\end{equation}
%
The definition of the viscosity coefficient $\mu$ from \eqt{eq:visc2_sct2} follows:
%
\begin{equation}\label{eq:visc2_sct2}
\mu_{k}^{q,n} = \min \left( \mu_{e,k}^{q,n}, \mu_{max,k}^{q,n} \right)
\end{equation}
%
From this point, all variables are known to compute the integral of the dissipative term $\int_{k} \mu \grad u(\vec{r},t) \grad \phi = \sum_q \mu_{k}^{q,n} \grad u_k^{q,n} \grad \phi^q$.
%The dissipative terms consist of second order spatial derivatives that are reduced to first order derivative by integrating per part after multiplying by the test function and integrating over the computational domain. An integral over the boundary of the computational domain also appears and raises the question of how to treat the dissipative terms on the boundary. It is often suitable to set the viscosity coefficients to zero on the boundary so that the boundary term coming from integrating per part the dissipative terms is also zero. Hyperbolic systems of equations alike the multi-D Euler equations, require particular treatment to compute the flux on the boundaries based on the study of the characteristic and the eigenvalues: the boundary terms of the dissipative terms could interfere with the hyperbolic flux and lead to the wrong value at the boundaries. The dissipative terms require to evaluate at each quadrature points of each cell the first-order spatial derivatives of one or multiple variables. With Galerkin finite element method, this task is easily done by using the derivative of the test functions. \\
%Once the dissipative terms are implemented, the viscosity coefficient has to be computed: the viscosity coefficient is evaluated on the fly from the entropy residual and the jumps. The viscosity coefficient requires to compute the first-order viscosity and the high-order viscosity coefficients at the quadrature points.