%%%%%%%%%%%%%%%%%%%%%%%%%%%%%%%%%%%%%%%%%%%%%%%%%%%
%
%  New template code for TAMU Theses and Dissertations starting Fall 2012.  
%  For more info about this template or the 
%  TAMU LaTeX User's Group, see http://www.howdy.me/.
%
%  Author: Wendy Lynn Turner 
%	 Version 1.0 
%  Last updated 8/5/2012
%
%%%%%%%%%%%%%%%%%%%%%%%%%%%%%%%%%%%%%%%%%%%%%%%%%%%

%%%%%%%%%%%%%%%%%%%%%%%%%%%%%%%%%%%%%%%%%%%%%%%%%%%%%%%%%%%%%%%%%%%%%%
%%                           APPENDIX A 
%%%%%%%%%%%%%%%%%%%%%%%%%%%%%%%%%%%%%%%%%%%%%%%%%%%%%%%%%%%%%%%%%%%%%

\phantomsection

\chapter{\uppercase{Derivation of the dissipative terms for the Euler equations with variable area using the entropy minimum principle}} \label{app:diss_terms}

Euler equations (without viscous regularization) with variable area are recalled here
%
\begin{subequations}
\label{app:euler_variable_A}
%
\begin{equation}
\partial_t \left( \rho A \right) + \div \left( \rho \vec{u} A \right) = 0 
\end{equation}
%
\begin{equation}
\partial_t \left( \rho \vec{u} A \right) + \div \left[A\left( \rho \vec{u} \otimes \vec{u} + P \mathbf{I} \right) \right] = P \grad A 
\end{equation}
% 
\begin{equation}
\partial_t \left( \rho E A \right) + \div \left[ \vec{u} A \left( \rho E + P \right) \right] = 0 \,.
\end{equation}
\end{subequations}
%
The specific entropy is a function of the density $\rho$ and the internal energy $e$, i.e., $s(e,\rho)$ , the above system of equations satisfies the minimum entropy principle \cite{Toro} \tcr{is this really shown in Toro?} \tcb{I think it is and will double check tomorrow}
%
\begin{equation}
A \rho \left( \partial_t s + \vec{u} \cdot \div s \right) \geq 0 \, .
\end{equation}
%
The entropy function $s$ satisfies the second law of thermodynamics, $T ds = de - \frac{P}{\rho^2} d \rho$, which implies $s_e := T^{-1}$ and $s_\rho := -P T^{-1} \rho^{-2}$. One can show that \cite{jlg}
%
\begin{equation}
s_e = T^{-1} \geq 0 \text{ and }
Ps_e + \rho^2 s_{\rho} 
\end{equation}
%
In order to apply the entropy viscosity method to the variable-area Euler equations, dissipative terms need to be added to each equation in \eqt{app:euler_variable_A}. The functional forms of these terms need to be such that the entropy residual derived with these terms present also satisfies the minimum entropy principle. 
To prove the minimum entropy principle, the extra terms appearing in the entropy residual are either recast as conservative terms or shown to be positive. The rest of this appendix presents this demonstration. 
Following \cite{jlg}, we first write the variable-area equation with dissipative terms. 
%
%
\begin{subequations}
\label{app:euler_variable_A_diss}
%
\begin{equation}
\partial_t \left( \rho A \right) + \div \left( \rho \vec{u} A \right) = \div f 
\end{equation}
%
\begin{equation}
\partial_t \left( \rho \vec{u} A \right) + \div \left[A\left( \rho \vec{u} \otimes \vec{u} + P \mathbf{I} \right) \right] = P \grad A + \div g
\end{equation}
% 
\begin{equation}
\partial_t \left( \rho E A \right) + \div \left[ \vec{u} A \left( \rho E + P \right) \right] = \div ( h + \vec{u} \cdot g )  \,.
\end{equation}
\end{subequations}
%
where $f$, $g$ and $h$ are dissipative fluxes to be determined. Starting from the modified system of equations given in \eqt{app:euler_variable_A_diss}, the entropy residual is derived again. The derivation requires the following steps : express the governing laws in terms of primitive variables $(\rho, \vec{u}, e)$, multiply the continuity equation by $\rho s_\rho$ and the internal energy equation by $s_e$, and invoke multivariate chain rule, e.g., $\partial s /\partial x = s_e \partial e /\partial x + s_\rho \partial \rho /\partial x$. These steps are similar to the ones form the standard Euler equations \cite{jlg}. Some of the lengthy algebra is omitted here. The above steps yield:
%
\begin{multline}
\label{eq:ent_res_app}
A \rho \left( \partial_t s + \vec{u} \cdot \grad s \right) = s_e \left[ \div h + g : \grad u + \left( \frac{u^2}{2}-e \right) \div f \right] 
+ \rho s_{\rho} \div f
\end{multline}
%
The next step consists of choosing a definition for each of the dissipative terms so that the left hand-side is positive. The right hand-side of \eqt{eq:ent_res_app} can be simplified using the following relations, $g = A \mu \grad^s \vec{u} + f \otimes \vec{u}$ and $h = \tilde{h} - 0.5 || \vec{u} ||^2 f$, which yields:
%
\begin{eqnarray}
\label{eq:ent_res_app2}
A \rho \left( \partial_t s + \vec{u} \cdot \div s \right) &=& s_e \left[ \div \tilde{h}-e \div f \right] + \rho s_{\rho} \div f  + A s_e \mu \grad \vec{u}^s : \grad \vec{u}\nonumber
\end{eqnarray}
%
The right hand-side is now integrated by parts:
%
\begin{eqnarray}
\label{eq:ent_res_app3}
A \rho \left( \partial_t s + \vec{u} \cdot \div s \right) &=& \div \left[ s_e \tilde{h}-s_e e f  + \rho s_{\rho} f \right] -\nonumber \\
\div \tilde{h} \grad s_e  &+& f \cdot \grad (e s_e) -  f \cdot \grad ( \rho s_{\rho} ) + A s_e \mu \grad^s \vec{u} : \grad \vec{u} \nonumber
\end{eqnarray}
%
where $\grad^s$ is the symmetric gradient. The term $A s_e \mu \grad^s \vec{u} : \grad \vec{u}$ is positive and thus, does not need any further modification. It %
remains to treat the other terms of the right hand-side that we now call $rhs$:
%
\begin{equation}
rhs = \div \left[ s_e \tilde{h}-s_e e f  + \rho s_{\rho} f \right] - \tilde{h} \cdot \grad s_e  + f \cdot \grad (e s_e) - f \cdot \grad ( \rho s_{\rho} ) \nonumber
\end{equation}
%
The first term of $rhs$ is a conservative term. By choosing carefully a definition for $\tilde{h}$ and $f$, the conservative term can be expressed as a function of the entropy $s$. It is also required to include the variable area in the choice of the dissipative terms so that when assuming constant area, the regular multi-D Euler equations are recovered. The following definitions for $\tilde{h}$ and $f$ are chosen:
%
\begin{equation}
\tilde{h} = A \kappa \grad ( \rho e ) \text{ and } f = A \kappa \grad \rho, \nonumber 
\end{equation}
%
which yields, using the chain rule:
%
\begin{equation}
rhs = \div (\rho A \kappa \grad s ) - A \kappa \underbrace{\left[ \grad (\rho e) \grad s_e  - \grad \rho \grad (e s_e) +  \grad \rho \grad ( \rho s_{\rho} )  \right]}_{\mathbf{Q}} \nonumber
\end{equation}
%
It remains to treat the term $\mathbf{Q}$ that can be recast under a quadratic form, following the work done in \cite{jlg}:
%
\begin{eqnarray}
\mathbf{Q} &=& X^t \Sigma X \nonumber \\
\text{with } X &=& \begin{bmatrix}
\grad \rho \\
\grad e 
\end{bmatrix}
\text{and } \Sigma = \begin{bmatrix}
       \partial_{\rho} (\rho^2 \partial_{\rho} s) & \partial_{\rho,e} s  \\[0.3em]
       \partial_{\rho,e} s & \partial_{e,e} s           \\[0.3em]
     \end{bmatrix} \nonumber 
\end{eqnarray}
%
The matrix $\Sigma$ is symmetric and identical to the matrix obtained in \cite{jlg}. The sign of the quadratic form can be simply determined by studying the positiveness of the matrix $\Sigma$. In this particular case, it is required to prove that the matrix is negative definite: the quadratic form is in the right hand-side and is preceded of a negative sign. According to \cite{jlg}, the convexity of the opposite of the entropy function $s$ with respect to the internal energy $e$ and the specific volume $1/ \rho$ is sufficient to ensure that the matrix $\Sigma$ is negative definite. \\
Thus, the right hand-side of the entropy residual \eqt{eq:ent_res_app}, are now either recast as conservative terms, or known to be positive. Following the work done by \cite{jlg}, the entropy minimum principle holds.

\pagebreak{}