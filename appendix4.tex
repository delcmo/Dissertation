%%%%%%%%%%%%%%%%%%%%%%%%%%%%%%%%%%%%%%%%%%%%%%%%%%%
%
%  New template code for TAMU Theses and Dissertations starting Fall 2012.  
%  For more info about this template or the 
%  TAMU LaTeX User's Group, see http://www.howdy.me/.
%
%  Author: Wendy Lynn Turner 
%	 Version 1.0 
%  Last updated 8/5/2012
%
%%%%%%%%%%%%%%%%%%%%%%%%%%%%%%%%%%%%%%%%%%%%%%%%%%%

%%%%%%%%%%%%%%%%%%%%%%%%%%%%%%%%%%%%%%%%%%%%%%%%%%%%%%%%%%%%%%%%%%%%%%
%%                           APPENDIX D
%%%%%%%%%%%%%%%%%%%%%%%%%%%%%%%%%%%%%%%%%%%%%%%%%%%%%%%%%%%%%%%%%%%%%

\chapter{\uppercase{Proof of the entropy minimum principle for the radiation-hydrodynamic equations with dissipative terms}}
\label{app:appendixA}

In this appendix, a demonstration of the entropy minimum principle for the system of equations \eqt{eq:equation7} is given. This proof, inspired by \cite{jlg}, details the steps that lead to the derivation of the dissipative terms for the multi-D Euler equations by using the entropy minimum principle.\\
We start with the hyperbolic system given in \eqt{eq:equation3} and add dissipative terms to each equation as follows:
\begin{equation}
\label{eq:app_equ1bis}
\left\{
\begin{array}{llll}
\frac{d \rho}{dt} + \rho \partial_x u = \partial_x f \\
\partial_t (\rho u) + \partial_x \left(\rho u^2 +  P + \frac{\epsilon}{3} \right) = \partial_x g  \\
\partial_t (\rho E) + \partial_x \left[ u \left( \rho E +P \right) \right] = \partial_x \left( h + ug \right) \\
\partial_t \epsilon + u \partial_x \epsilon + \frac{4}{3} \epsilon \partial_x u = \partial_x l
\end{array}
\right.
\end{equation}
where $f$, $g$, $h$ and $l$ are dissipative terms to be determined.
\eqt{eq:app_equ1bis} is then recast as a function of the primitive variables $(\rho, u, e, \epsilon)$ to yield:
\begin{equation}
\label{eq:app_equ1}
\left\{
\begin{array}{llll}
\frac{d \rho}{dt} + \rho \partial_x u = \partial_x f \\
\rho \frac{du}{dt} + \partial_x \left( P + \frac{\epsilon}{3} \right) = \partial_x g - u \partial_x f  \\
\rho \frac{de}{dt} + P \partial_x u = \partial_x h + g \partial_x u + \left( 0.5 u^2 - e \right) \partial_x f \\
\frac{d\epsilon}{dt} + \frac{4}{3} \epsilon \partial_x u = \partial_x l
\end{array}
\right.
\end{equation}
The right-hand side of the internal energy equation can be simplified by choosing the dissipative terms $g$ and $h$ as follows: $h = \tilde{h} -0.5 u^2 f$ and $g = \rho \mu \partial_x u + uf$ where $\mu \geq 0$ is a dissipative coefficient. Using these definitions, the system of equation given in \eqt{eq:app_equ1} becomes:
\begin{equation}
\label{eq:app_equ1ter}
\left\{
\begin{array}{llll}
\frac{d \rho}{dt} + \rho \partial_x u = \partial_x f \\
\rho \frac{du}{dt} + \partial_x \left( P + \frac{\epsilon}{3} \right) = \partial_x g - u \partial_x f  \\
\rho \frac{de}{dt} + P \partial_x u = \rho \mu (\partial_x u)^2 + \partial_x \tilde{h} - e \partial_x f \\
\frac{d\epsilon}{dt} + \frac{4}{3} \epsilon \partial_x u = \partial_x l
\end{array}
\right.
\end{equation}
This system of equation admits an entropy function $s$ that depends on density $\rho$, internal energy $e$ and radiation energy density $\epsilon$. In order to prove the entropy minimum principle, a conservation statement satisfied by the entropy is needed. This equation which is referred to as an entropy residual $D_e(x,t)$, can be obtained by a combination of the equations given in \eqt{eq:app_equ1ter}. This process is motivated by the following (chain rule) 
%\comment{I think you still had a mistake. Check my correction.\color{blue}{I do not see where the mistake is}}
\begin{equation}
\label{eq:app_equ2}
\partial_{\alpha} s = \partial_{\rho} s \partial_{\alpha} \rho +  \partial_{e} s \partial_{\alpha}e +  \partial_{\epsilon} s \partial_{\alpha} \epsilon \text{,}
\end{equation}
 which holds for any independent variable $\alpha=x,t$. It is also required to define the dissipative terms $\tilde{h}$, $f$ and $l$. The following definitions are chosen:
 \begin{equation}
 \left\{
 \begin{array}{ccc}
f &= \kappa \partial_x \rho \\
\tilde{h} &= \kappa \partial_x (\rho e) \\
l &= \kappa \partial_x \epsilon
 \end{array}
 \right.
 \end{equation}
 where $\kappa$ is another positive dissipative coefficient. \\
 Thus, using the continuity, the internal energy and the radiation equations of \eqt{eq:app_equ1ter} and using \eqt{eq:app_equ2} along with the definition of the dissipative terms, a conservation statement satisfied by the entropy $s$ is obtained:
 \begin{align}
 \label{eq:app_equ3}
 \frac{ds}{dt} + \underbrace{\left( P \partial_e s + \rho^2 \partial_{\rho} s + \frac{4}{3} \rho \epsilon \partial_{\epsilon} s \right) \partial_x u}_\textrm{(a)} &= \partial_x \left( \rho \kappa \partial_x s \right) + \nonumber \\ \kappa \partial_e s \partial_x s &- \rho \kappa \underbrace{X A X^t}_\textrm{(b)} + \underbrace{ s_e \rho \mu (\partial_x u)^2}_\textrm{(c)}
 \end{align}
 where $X$ is a row vector defined as $X=\left( \rho, e, \epsilon \right)$ and $A$ is the $3$x$3$ symmetric matrix:
 \begin{equation}
 A = 
 \left[
 \begin{array}{ccc}
\partial_{\rho} \left( \rho^2 \partial_{\rho} s \right) & \partial_{\rho,e} s & \partial_{\rho} \left( \rho \partial_{\epsilon} s \right) \\
 \partial_{\rho,e} s & \partial_{e,e} s & \partial_{e,\epsilon} s \\
 \partial_{\rho} \left( \rho \partial_{\epsilon} s \right) & \partial_{e,\epsilon} s & \partial_{\epsilon,\epsilon} s
 \end{array}
 \right]
 \end{equation}
 In order to show that an entropy minimum principle holds, the signs of the terms $(a)$, $(b)$ and $(c)$ in \eqt{eq:app_equ3} need to be studied.\\
Regarding $(a)$, it is assumed that $P \partial_e s + \rho^2 \partial_{\rho} s + \frac{4}{3} \rho \epsilon \partial_{\epsilon} s=0$. The motivation for this is two-fold: First, in order to have a negative sign for the term $(a)$, it would require $P \partial_e s + \rho^2 \partial_{\rho} s + \frac{4}{3} \rho \epsilon \partial_{\epsilon} s$ to have a sign of opposite to that $\partial_x u$. The thermodynamic variables cannot be a function of the material velocity or its derivative under a non-relativistic assumption. Such a statement would not be true when dealing with relativistic equations of state. Second, a similar assumption was made in \cite{jlg} for multi-D Euler equations (without the radiation energy): $P \partial_e s + \rho^2 \partial_{\rho} s = 0$.\\
The term $(b)$, $XAX^t$, is a quadratic form and its sign is determined by simply looking at the positiveness of the matrix $A$ \cite{Evans}. Here we need to prove that $A$ is negative-definite which is equivalent to showing the three following inequalities:
\begin{equation}
 \left\{
 \begin{array}{ccc}
 A_1 \geq 0 \\
 A_2 \leq 0 \\
 A_3 = A \geq 0
 \end{array}
 \right.
 \end{equation}
 where $A_k$ is the $k^{th}$ order leading principle minor. Determining the sign of the last inequality that corresponds to the determinant of the $3$ by $3$ matrix $A$ can be difficult and needs to be simplified. Zeroing out the off-diagonal entries of the last row or column would simplify the expression for the determinant of $A$. This can be achieved by assuming $\partial_{\rho}(\rho \partial_{\epsilon} s)$ and $\partial_{e, \epsilon} s$ are zero, which requires the following form for the entropy function:
\begin{equation}
\label{eq:app_equ4}
s(\rho, e, \epsilon) = \tilde{s}(\rho,e) + \frac{\rho_0}{\rho}\hat{s}(\epsilon) \text{. } 
\end{equation}
where $\tilde{s}$ and $\hat{s}$ are two functions whose properties will be provided later. The constant $\rho_0$ is used for a dimensionality purpose.
Next, using the expression of the entropy given in \eqt{eq:app_equ4}, matrix $A$ becomes:
 \begin{equation}
 A = 
 \left[
 \begin{array}{ccc}
\partial_{\rho} \left( \rho^2 \partial_{\rho} \tilde{s} \right) & \partial_{\rho,e} \tilde{s} & 0 \\
 \partial_{\rho,e} \tilde{s} & \partial_{e,e} \tilde{s} & 0 \\
 0 & 0 & \rho^{-1} \partial_{\epsilon,\epsilon} \hat{s}
 \end{array}
 \right] \nonumber
 \end{equation}
 Proving that the matrix $A$ is  negative-definite is now straightforward by inspecting the sign of the leading principal minors:
 \begin{equation}
 \label{eq:A_matrix}
 \left\{
 \begin{array}{lll}
 A_1 = \partial_{\rho} \left( \rho^2 \partial_{\rho} \tilde{s} \right) \leq 0 \\
 A_2 = \partial_{\rho} \left( \rho^2 \partial_{\rho} \tilde{s} \right) \partial_{e,e} \tilde{s} - \left( \partial_{\rho,e} \tilde{s} \right)^2 \geq 0\\
 A_3 =  \rho^{-1} \partial_{\epsilon,\epsilon} \hat{s} A_2 \leq 0
 \end{array}
 \right.
 \end{equation} 
This is easily achieved when assuming that the functions $-\tilde{s}$ and $-\hat{s}$ are convex. Thus, the sign of $(b)$ is now determined. \\
 %This result is very strong since $\tilde{s}$ corresponds to the same entropy function as the one described in \cite{jlg}.\\
Finally, it remains to determine the sign of the term $(c) = \partial_e s \rho \mu (\partial_x u)^2$. The density $\rho$ and the viscosity coefficient $\mu$ are both positive: the latest proof for positivity of the density can be found in \cite{jlg}. Then, only the sign of $\partial_e s$ remains unknown but it can be determined by studying $(a)$. It was assumed earlier in this appendix that $P \partial_e s + \rho^2 \partial_{\rho} s + \frac{4}{3} \rho \epsilon \partial_{\epsilon} s=0$. This equation is now recast and split into two equations using \eqt{eq:app_equ4}. Separation of variables yields:
 \begin{equation}
 P \partial_e \tilde{s} + \rho^2 \partial_{\rho} \tilde{s} = \alpha \text{ and } \hat{s} - \frac{4\epsilon}{3} \partial_{\epsilon} \hat{s} = \alpha \nonumber
 \end{equation}
 where $\alpha$ is a constant to determine. If one sets $\alpha=0$, then the two physics are decoupled, which allows us to reconnect to the result derived in \cite{jlg} for the multi-D Euler equations: $P \partial_e \tilde{s} + \rho^2 \partial_{\rho} \tilde{s} = 0$. Then, following \cite{jlg}, definitions for $\partial_e \tilde{s}$ and $\partial_{\rho} \tilde{s}$ are obtained:
 \begin{equation}
 \label{eq:definition}
 \left\{
 \begin{array}{ll}
 \partial_e s = \partial_e \tilde{s} = T^{-1} \nonumber\\
 \partial_{\rho} \tilde{s} = -\frac{P}{\rho^2} \partial_e \tilde{s}
 \end{array}
 \right.
 \end{equation} 
 where $T$ is the material temperature which ensures positivity of $\partial_e s$. Thus, $(c)$ is positive. \\
From the above results, the entropy minimum principle follows, so that the sign of the entropy residual is known:
\begin{equation}
\boxed{\partial_t s + u \partial_x s \geq 0}
\end{equation}
 %%%%%%%%%%%%%%%
\begin{remark}
By assuming $\alpha=0$, an expression for the $\hat{s}$ can be derived by solving the ODE, $\hat{s} - \frac{4\epsilon}{3} \partial_{\epsilon} \hat{s} = 0$, which yields:
$\hat{s}(\epsilon) = \beta \exp \left(\frac{4 \epsilon^{2}}{3} \right)$, where $\beta$ is a constant. The sign of $\beta$ is determined by using the condition, $\partial_{\epsilon,\epsilon} \hat{s} \leq 0$, derived above, so that $\beta\leq0$.
\end{remark}
\begin{remark}
The viscous regularization derived in this appendix, has two viscosity coefficients: $\mu$ and $\kappa$. For the purpose of this paper, these coefficients are set equal. Under this assumption, the above viscous regularization is equivalent to the parabolic regularization of \cite{Parabolic}.
\end{remark}

\pagebreak{}