%%%%%%%%%%%%%%%%%%%%%%%%%%%%%%%%%%%%%%%%%%%%%%%%%%%
%
%  New template code for TAMU Theses and Dissertations starting Fall 2012.  
%  For more info about this template or the 
%  TAMU LaTeX User's Group, see http://www.howdy.me/.
%
%  Author: Wendy Lynn Turner 
%	 Version 1.0 
%  Last updated 8/5/2012
%
%%%%%%%%%%%%%%%%%%%%%%%%%%%%%%%%%%%%%%%%%%%%%%%%%%%

%%%%%%%%%%%%%%%%%%%%%%%%%%%%%%%%%%%%%%%%%%%%%%%%%%%%%%%%%%%%%%%%%%%%%%
%%                           SECTION 4
%%%%%%%%%%%%%%%%%%%%%%%%%%%%%%%%%%%%%%%%%%%%%%%%%%%%%%%%%%%%%%%%%%%%%%

%%%%%%%%%%%%%%%%%%%%%%%%%%%%%%%%%%%%%%%%%%%%%%%%%%%%%%%%%%%%%%%%%%%%%%
\chapter{\uppercase {Application of the entropy viscosity method to the seven equations model}}\label{chap:seven}
%%%%%%%%%%%%%%%%%%%%%%%%%%%%%%%%%%%%%%%%%%%%%%%%%%%%%%%%%%%%%%%%%%%%%%
%
Compressible two-phase flows have numerous industrial applications and thus, have been a field of focus over the last years. A variety of models with different level of complexity has been developed such as: five equation model, six equation model and the more recently the seven equation model. These models are all obtains by integrating the single-phase flow balance equations weighed by a characteristic function for each phase. The resulting system of equation contains non-conservative terms that describe the interaction between phases but also an equation for the volume fraction variable. Once a system of equations describing the physics is derived, the other challenging step is to develop a consistent numerical solver in order to obtain a numerical solution. Assuming that the system of equations is hyperbolic under some conditions, a Riemann solver could be used but often rule out because of the complexity due to the number of equations involved. Furthermore, careless approximation for the treatment of the non-conservative terms can lead to failure in computing the numerical solution. An alternative is to use an approximate Riemann solver, that are well-known for single-phase flows, while deriving a consistent discretization scheme for the non-conservative terms. 

This methodology was applied to the seven equation model (SEM) that was first introduced by (REF). This model is known to be unconditionally hyperbolic which is highly valuable when working with approximate Riemann solver and can treat a wide range of applications. Its particularity comes from the pressure and velocity relaxation terms in the volume fraction, momentum and energy equations that can bring the two phases in equilibrium when using infinite relaxation parameters. In other words, the seven equation model can degenerate into the six and five equation models. Solving for the seven equation model require a numerical solver and a tremedous amount of work was dedicated to this task for discontinuous schemes. Because each phase is assumed to obey the Euler equations, most of the numerical solvers are adapted from the single-phase approximate Riemann solvers. For example, Saurel et al. employed a HLL-type scheme to solve for the SEM but noted that excessive dissipation was added to the contact discontinuity. A more advanced HLLC type scheme was developed in (REF) but only for the subsonic case and then extended to supersonic flows in (REFS). More recently, Ambroso et al. (REF) proposed an approximate Riemann solver accounting for source terms such as gravity and drag forces.\\

 We proposed to investigate how the EVM applies to the seven equation model when discretized with a CFEM. First, the multi-D seven equation model is recalled and detailed in \sect{sec:multi-seven-equ-model}, and a particular attention is brought to the entropy equation. Then, the dissipative terms are derived using the entropy inequality, in \sect{sec:sev-equ-visc-reg-sect4}, on the same model of what is done in \chap{chap:euler} for the multi-D Euler equations. In \sect{sec:sev-equ-visc-coeff-sect4}, a low-Mach asymptotic limit is performed in order to derived a definition for the viscosity coefficients consistent with the incompressible limit. Lastly, $1$-D numerical results are presented in \sect{sec:1d-num-res-sect4}.
%\begin{itemize}
%\item background on 7 emu model: first introduced. 
%\item method to solve this system: discontinuous scheme. Approximate Riemann solvers. Quote several papers
%\item differences between system of equation -> closure relations for relaxation coefficients.
%\end{itemize}
%===================================================================================
\section{Descriptions of the multi-D seven equation model}\label{sec:multi-seven-equ-model}
%===================================================================================
The multi-D seven equation model is obtained by assuming that each phase obeys the pure Euler equations and by integrating over a control volume after multiplying by a characteristic function. The detailed derivation can be found in \cite{SEM}. In this section, the governing equations are recalled for each phase and the source terms are described. 
%-----------------------------------------------------------------------------------------
\subsection{The system of equation for the liquid and vapor phases}\label{sec:multi-d-7eqn-model-sect4}
%-----------------------------------------------------------------------------------------
The liquid phase obeys a mass, momentum and energy balance equations and a non-conservative volume fraction equation:
%
\begin{subequations}\label{eq:liq-7-eqn-sect5}
\begin{align}
  % liquid mass conservation
  \label{multi-d-7-equ-liq}
  \frac{\partial \left( \alpha \rho \right)_{liq} A}{\partial t}
  + \div \left( \alpha \rho \mbold u \right)_{liq} A
  &= - \Gamma A_{int} A
\end{align}
  \begin{align}
  % liquid momentum
  \frac{\partial \left( \alpha \rho \mbold u \right)_{liq} A}{\partial t}
  + \div \left[ \alpha_{liq} A \left( \rho \mbold u \otimes \mbold u + P  \right)_{liq} \right]
  &= P_{int} A \grad \alpha_{liq} + P_{liq} \alpha_{liq} \grad A
    \nonumber
  \\
  &+ A \lambda_u (\mbold u_{vap} - \mbold u_{liq})
  - \Gamma A_{int} \mbold u_{int} A
\end{align}
\begin{align}
  % liquid total energy
  \frac{\partial \left( \alpha \rho E \right)_{liq} A}{\partial t}
  + \div \left[ \alpha_{liq} \mbold u_{liq} A \left( \rho E + P \right)_{liq} \right]
  &= P_{int} \mbold u_{int} A \grad \alpha_{liq} - \bar{p}_{int} A \mu_P (P_{liq} - P_{vap})
        \nonumber
  \\
  + \bar{\mbold u}_{int} A \lambda_u (\mbold u_{vap} - \mbold u_{liq})
&  + \Gamma A_{int} \left( \frac{p_{int}}{\rho_{int}} - H_{liq, int} \right) A
\end{align}
\begin{align}
  % liquid volume fraction
  \label{eqn:multi-d-7-eqn-liq-vol}
  \frac{\partial \alpha_{liq} A}{\partial t} + A\mbold u_{int} \cdot \grad \alpha_{liq}
  &= A \mu_P (P_{liq} - P_{vap}) - \frac{\Gamma A_{int} A}{\rho_{int}}
\end{align}
\end{subequations}
%
On the same model, the equations for the vapor phase are:
%
\begin{subequations}\label{eq:vap-7-eqn-sect5}
\begin{align}
  \label{multi-d-7-equ-vap}
  % vapor mass conservation
  \frac{\partial \left( \alpha \rho \right)_{vap} A}{\partial t}
  + \div \left( \alpha \rho \mbold u \right)_{vap} A
  =  \Gamma A_{int} A
\end{align}
\begin{align}
  % vapor momentum
  \frac{\partial \left( \alpha \rho u \right)_{vap} A}{\partial t}
  + \div \left[ \alpha_{vap} A \left( \rho \mbold u \otimes \mbold u + P\mathbb{I} \right)_{vap} \right]
  &= P_{int} A \grad \alpha_{vap} + P_{vap} \alpha_{vap} \grad A
  \\
  \nonumber
  &+ A \lambda_u (\mbold u_{liq} - \mbold u_{vap})
  + \Gamma A_{int} u_{int} A
\end{align}
\begin{align}
  % vapor total energy
  \frac{\partial \left( \alpha \rho E \right)_{vap} A}{\partial t}
  &+ \div \left[ \alpha_{vap} \mbold u_{vap} A \left( \rho E + P \right)_{vap} \right]
  = P_{int} \mbold u_{int} A \grad \alpha_{vap} - \bar{P}_{int} A \mu_P (P_{vap} - P_{liq})
  \\
  \nonumber
  &+ \bar{\mbold u}_{int} A \lambda_u (\mbold u_{liq} - \mbold u_{vap})
- \Gamma A_{int} \left( \frac{p_{int}}{\rho_{int}} - H_{vap, int} \right) A
\end{align}
\begin{align}
  % vapor phase volume fraction
  \label{eqn:multi-d-7-eqn-vap-vol}
  \frac{\partial \alpha_{vap} A}{\partial t} + A \mbold u_{int} \cdot \grad \alpha_{vap}
  &= A \mu_P (P_{vap} - P_{liq}) + \frac{\Gamma A_{int} A}{\rho_{int}}
\end{align}
\end{subequations}
%
where $\alpha_k$, $\rho_k$, $\mbold u_k$ and $E_k$ denote the volume fraction, the density, the velocity vector and the total energy of phase $k=\left\{ liq, vap \right\}$, respectively. The phase pressure $P_k$ is computed from an equation of state. The interfacial variables are denoted by the subscript $int$ and their definition will be given in \sect{sec:source-terms-7-eqt-sect5}. The interfacial pressure and velocity, and their corresponding average values are denoted by $P_{int}$, $\mbold u_{int}$, $\bar{P}_{int}$ and $\bar{\mbold u}_{int}$, respectively. $\Gamma$ is the net mass transfer per unit interfacial area from the liquid to the vapor phase and $A_{int}$ is the interfacial area per unit volume of mixture.  Also, $H_{liq, int}$ and $H_{vap, int}$ are the liquid and gas total enthalpies at the interface, respectively. $\mu_P$ is the pressure relaxation coefficient and $\lambda_u$ denotes the velocity relaxation coefficient. Lastly, the cross-section $A$ is only spatial dependent. In the case of two-phase flow, the equation for the vapor volume fraction, \eqt{eqn:multi-d-7-eqn-vap-vol}, is simply replaced by the algebraic relation
%
\begin{align}
 \alpha_{vap}= 1 - \alpha_{liq}
\end{align}
%
The set of eight equations given in \eqt{eq:liq-7-eqn-sect5} and in \eqt{eq:vap-7-eqn-sect5} is now reduced to seven which yields the multi-D seven equation model.

For the phase $k$, an entropy equation can be derived when accounting only for the pressure and velocity relaxation terms (all of the terms proportional to the net mass transfer term $\Gamma$ are removed). The entropy function for a phase $k$ is denoted by $s_k$ and function of the density $\rho_k$ and the internal energy $e_k$. The derivation is detailed in \app{app:sev-equ-model-entropy} and only the final result is recalled here:
%
\begin{align}\label{eq:ent-eqn-7-eqn-model}
(s_{e})_k^{-1} \alpha_k \rho_k A \frac{Ds_k}{Dt} &= \mu_P \frac{Z_k}{Z_k+Z_j} (P_j - P_k)^2 + \lambda_u \frac{Z_j}{Z_k+Z_j} (\mbold u_j -\mbold  u_k)^2 \nonumber
\\
& \frac{Z_k}{\left( Z_k+Z_j \right)^2} \left[ Z_j (\mbold u_j-\mbold u_k)+\frac{\grad \alpha_k}{|| \grad \alpha_k ||}(P_k-P_j)\right]^2,
\end{align}
where $Z_{k}$ denotes the phasic acoustic impedance and is defined as the product of the density and the speed of sound: $Z_k = \rho_k c_k$. The partial derivative of the entropy function $s_k$ with respect to the internal energy $e$, $(s_e)_k$, is defined proportional to the inverse of the phasic temperature alike in \chap{chap:euler} for the singe phase Euler equations. The right hand-side of \eqt{eq:ent-eqn-7-eqn-model} is unconditionally positive since all terms are squared. Furthermore, \eqt{eq:ent-eqn-7-eqn-model} is valid for each phase $k=\left\{liq, vap \right\}$ and ensures positivity of the total entropy equation that is obtained by summing over the phases:
%
\begin{equation}\label{eq:tot-ent-res-sct4}
\sum_k (s_{e})_k^{-1} \alpha_k \rho_k A \frac{Ds_k}{Dt} = \sum_k (s_{e})_k^{-1} \alpha_k \rho_k A \left( \partial_t s_k + \mbold u_k \cdot \grad s_k \right) \geq 0.
\end{equation}
%
It is noted that when one phase disappears, \eqt{eq:tot-ent-res-sct4} degenerates into the single phase entropy equation given in \eqt{eq:ent_res}.
%\begin{itemize}
%\item give system of equation with relaxation terms and mass transfer term from DEM paper
%\item explain the relaxation terms
%\item explain the source terms: exchanges between phases.
%\item entropie equation.
%\end{itemize}
%-----------------------------------------------------------------------------------------
\subsection{The source terms}\label{sec:source-terms-7-eqt-sect5}
%-----------------------------------------------------------------------------------------
In this section, insights about the relaxation terms, the net mass transfer term and the interfacial heat transfer terms are given.
%---------------------------------------------------------------------------------------------------------------
\subsubsection{Interface Pressure and Velocity, Mechanical Relaxation Coefficients}
%---------------------------------------------------------------------------------------------------------------
The relaxation terms are used to bring the two phases in equilibrium by making pressure and velocity equal. The mechanical relaxation coefficients $\mu_P$ and $\lambda_u$ can be seen as relaxation times: the larger the relaxation coefficients, the faster the two phases will be brought to equilibrium. Derivation of the relaxation terms is achieved by using rational thermodynamic to ensure consistency with the second thermodynamic law \cite{Truesdell}. The methodology is very similar to what is done for the derivation of the dissipative terms using the entropy inequality.

In the continuous limit of small mesh spacing and time steps along with employment of the Godunov weak wave limit, it can be shown that the pressure and velocity relaxation terms obeys the following relations \cite{Berry_2008b, Chinnayya_2004}:
%
\begin{align}
  \label{E-R:83}
  p_{int} &= \bar{p}_{int} + \frac{Z_{liq}Z_{vap}}{Z_{liq}+Z_{vap}} \frac{\grad \alpha_{liq}}{|| \grad \alpha_{liq} ||} \cdot (\mbold u_{vap}-\mbold u_{liq})
  \\
  \bar{p}_{int} &= \frac{Z_{vap}p_{liq}+Z_{liq}p_{vap}}{Z_{liq}+Z_{vap}}
\end{align}
%
The interfacial velocities $\mbold u_{int}$ and its average value $\bar{\mbold u}_{int}$ are computed from:
%
\begin{align}
  \label{E-R:84}
  \mbold u_{int} &= \bar{\mbold u}_{int} +  \frac{\grad \alpha_{liq}}{|| \grad \alpha_{liq} ||} \frac{p_{vap}-p_{liq}}{Z_{liq}+Z_{vap}}
  \\
  \bar{\mbold u}_{int} &= \frac{Z_{liq} \mbold u_{liq}+Z_{vap}\mbold u_{vap}}{Z_{liq}+Z_{vap}}.
\end{align}
%
The pressure, $\mu_P$, and velocity, $\lambda_u$, relaxation coefficients are proportional to each other and function of the interfacial area $A_{int}$:
%
\begin{align}
  \label{E-R:85}
  \lambda_u &= \frac{1}{2} \mu_P Z_{liq} Z_{vap}
  \\
  \label{E-R:86}
  \mu_P &= \frac{A_{int}}{Z_{liq}+Z_{vap}}
\end{align}
%where $\lambda_u$ is the velocity relaxation coefficient function, $\mu_P$
%is the pressure relaxation coefficient function, $Z_k = \rho_k c_k$,
%$(k=liq, vap)$, is the phasic acoustic impedance and 
The specific interfacial area (i.e. the interfacial surface area per unit
volume of two-phase mixture), $A_{int}$, must be specified from some type of
flow regime map or function. In \cite{SEM}, it is chosen function of the liquid volume fraction as follows:
%
\begin{equation}
A_{int} = A_{int}^{max} \left[ 6.75 \left(1-\alpha_{liq} \right)^2 \alpha_{liq} \right],
\end{equation}
% 
where $A_{int}^{max} = 5100$ $m^2 / m^3$. With such definition, the interfacial area is zero in the limits $\alpha_{liq} = 0,1$. 
%The DEM model for two-phase flow of
%water and its vapor in a one dimensional duct of spatially varying
%cross-section was derived and demonstrated with these closures by
%Berry et al.~\cite{SEM}.
From this specification of $\lambda_u$ and $\mu_P$ it is clear
that special coupling is rendered.  To relax the seven-equation model to
the ill-posed classical six-equation model, the pressures should be
relaxed toward a single pressure for both phases.  This is
accomplished by specifying the pressure relaxation coefficient to be
very large, i.e. letting it approach infinity.  But if the pressure
relaxation coefficient goes to infinity, so does the velocity
relaxation rate also approach infinity.  This then relaxes the
seven equation model not to the classical six equation model, but to the
mechanical equilibrium five equation model of Kapila.  This reduced
five equation model is also hyperbolic and well-posed. The five equation
model provides a very useful starting point for constructing
multi-dimensional interface resolving methods which dynamically
captures evolving, and even spontaneously generating,
interfaces~\cite{Saurel_2009}. Thus the seven equation model
can be relaxed locally to couple seamlessly with such a
multi-dimensional, interface resolving code.

Numerically, the mechanical relaxation coefficients $\mu_P$
(pressure) and $\lambda_u$ (velocity) can be relaxed independently to
yield solutions to useful, reduced models (as explained previously).  It
is noted, however, that relaxation of pressure only by making $\mu_P$
large without relaxing velocity will indeed give ill-posed and
unstable numerical solutions, just as the classical six equation
two-phase model does, with sufficiently fine spatial resolution, as
confirmed in~\cite{SEM,Herrard_2005}.

Even though the implementation of the seven equation two-phase
model does not use
the generalized approach of DEM, the interfacial pressure and velocity
closures as well as the pressure and velocity relaxation coefficients
of Equations~\eqref{E-R:83} to~\eqref{E-R:86} are utilized.
%-----------------------------------------------------------------------------------------
\subsubsection{Interphase Mass Transfer}
%-----------------------------------------------------------------------------------------
For a vapor to be formed from the liquid phase (vaporization) energy
must be added to the liquid to produce vapor at nucleation sites;
whether the liquid is heated directly or decompressed below its
saturation pressure.  A liquid to vapor phase change may occur based
on two main mechanisms.  The first is related to vaporization induced
by external heating or heat transfer in a nearly constant pressure
environment which is called heterogeneous boiling, or simply
boiling.  This heat input can occur through a solid/liquid
interface with the solid typically hotter than the liquid, or through
a liquid/gas interface with the gas being hotter than the liquid.

\begin{figure}
  \centering
  %\fbox{
   \includegraphics[clip=true,viewport=200 50 550 500,width=.8\textwidth]{figures/saturation}
   % }
   \caption{Interface control volume (top); $T$-$p$ state space around
     saturation line, $T_{liq} < T_{vap}$, (bottom).\label{Berry-Fig:2}}
\end{figure}

To examine the mass flow rate between phases, local mechanisms of the
vaporization (condensation) process are considered between the liquid
phase and its associated vapor in the presence of temperature
gradients.  The mechanisms of interest here are dominated by heat
diffusion at the interface.  The pertinent local equations to consider
are the mass and energy equations.  As a vaporization front propagates
slowly (on the order of 1 mm/s to 1 m/s) compared to acoustic waves
present in the medium (which propagate with speeds of the order 1
km/s), acoustic propagation results in quasi-isobaric pressure
evolution through vaporization fronts.  The momentum equation is
therefore not needed -- because the quasi-isobaric assumption
(neglecting the pressure and kinetic energy variations in the total
energy equation) is made.  A simple expression for the interphase
mass flow rate is obtained
\begin{align}
  \nonumber
  \Gamma = \Gamma_{vap}
  &= \frac{h_{T,  liq} \left( T_{liq} - T_{int} \right) + h_{T,  vap} \left( T_{vap} - T_{int} \right)}{h_{vap,  int} - h_{liq,  int}}
  \\
  &= \frac{h_{T,  liq} \left( T_{liq} - T_{int} \right) + h_{T,  vap} \left( T_{vap} - T_{int} \right)}{L_v \left( T_{int} \right)}
\end{align}
where $L_v \left( T_{int} \right) = h_{vap,  int} - h_{liq,  int}$
represents the latent heat of vaporization.  The interface
temperature is determined by the saturation constraint
$T_{int}=T_{sat}(p)$ with the appropriate pressure $p=\bar{p}_{int}$
determined above, the interphase mass flow rate is thus determined.
The lower graphic of Figure~\ref{Berry-Fig:2}, schematically shows the
$p$-$T$ state space in the vicinity of the saturation line (shown
for the case with $T_{liq} < T_{vap}$).

To better illustrate the model for vaporization or condensation,
Figure~\ref{Berry-Fig:3} shows pure liquid and pure vapor regions
separated by an interface.
\begin{figure}
  \centering
  %\fbox{
   \includegraphics[clip=true,viewport=175 50 625 525,width=.8\textwidth]{figures/vaporization_condensation}
 %}
   \caption{Vaporization and condensation at a liquid-vapor interface
     (after Moody~\cite{Moody_1990}).\label{Berry-Fig:3}}
\end{figure}
Representative temperature profiles are shown for heat transfer from
vapor to liquid or liquid to vapor.  As discussed by
Moody~\cite{Moody_1990}, either vaporization or condensation can occur
for both temperature profiles. The interphase mass transfer is
determined by the net interfacial heat transfer: if net heat transfer
is toward the interface, vapor will form; conversely, if net heat
transfer is away from the interface, liquid will condense.
Figure~\ref{Berry-Fig:3} shows heat transfer rates $q_{vap}$ and
$q_{liq}$ from the vapor and liquid sides of the interface.  For
bidirectional phase change (vaporization and condensation), mass
transfer based on heat balance at the interface is adopted.  When
vaporization occurs, vapor is assumed to form at a saturated interface
temperature $T_{int}=T_{sat}(\bar{p}_{int})$.  If condensation occurs,
liquid is assumed to form also at a saturated interface temperature
$T_{int}=T_{sat}(\bar{p}_{int})$.  The interfacial total enthalpies
correspond to the saturated values in order that the interphase mass
transfer rate and conservation of total energy be compatible:
\begin{equation}
  \label{E-R:95}
  H_{k,  int} = h_{k,  int} + \frac{1}{2} u_{int}^2
\end{equation}
for phase $k=(liq, vap)$, where $h_{k,int}$ is the phase $k$ specific enthalpy
evaluated at the interface condition.  Phasic specific enthalpy
depends upon the equation of state used and will be discussed with the
equations of state.  The interfacial density corresponds to the liquid
saturated density $\rho_{int} = \rho_{liq, sat}(p_{int})$.
%-----------------------------------------------------------------------------------------
\subsubsection{Interface Direct Heat Transfer}
%-----------------------------------------------------------------------------------------
Without wall boiling, a simple model for the direct, convective heat transfer
from the wall to fluid phase $k$ will be the same as that of a single-phase
except the duct wall area over which this heat transfer can occur is weighted
by the wetted fraction of the phase.  That is,
\begin{equation}
  Q_{ \text{wall}, k } = H_{w,k} a_w \left(T_k  - T_{ \text{wall} } \right) \alpha_k A
\end{equation}
for phase $k=(liq, vap)$, where $H_{w,k}$ is the wall convective wall heat transfer
coefficient associated with phase $k$. Similarly, the direct heat
transfer from/to the interface to/from the phase $k$, which will also
be used to determine the mass transfer between the phases, is
\begin{equation}
  Q_{int,  k} = h_{T,  k}  \left( T_{int} - T_k \right)  A_{int}  A
\end{equation}
with $h_{T,  k}$ denoting the convective heat transfer coefficient
between the interface and phase $k$. The phasic bulk
temperature $T_k$ is determined from the respective phase's equation of
state.
%-----------------------------------------------------------------------------------------
\subsubsection{Stiffened Gas Equation of State for Two-phase Flows} \label{sec:SGEOS}
%-----------------------------------------------------------------------------------------
With the seven equation two-phase model each phase is compressible and
behaves with its own convex equation of state (EOS).  For initial
development purposes it was decided to use a simple form capable of
capturing the essential physics.  For this purpose the stiffened
gas equation of state (SGEOS)~\cite{SGEOS} was selected (as
it was also for single phase)
\begin{equation}
  \label{E-R:96}
  p(\rho,e) = (\gamma -1) \rho (e - q) - \gamma p_{\infty}
\end{equation}
where $p$, $\rho$, $e$, and $q$ are the pressure, density,
internal energy, and the binding energy of the fluid considered.  The
parameters $\gamma$, $q$, and $p_{\infty}$ are the constants
(coefficients) of each fluid.  The first term on the right hand side
is a repulsive effect that is present for any state (gas, liquid, or
solid), and is due to molecular vibrations.  The second term on the
right represents the attractive molecular effect that guarantees the
cohesion of matter in the liquid or solid phases.  The parameters used
in this SGEOS are determined by using a reference curve, usually in
the $\left(p, \frac{1}{\rho}\right)$ plane.

To extend this equation of state for two phases,
LeMetayer~\cite{SGEOS} uses the saturation curves as this
reference curve to determine the stiffened gas parameters for liquid
and vapor phases.  The SGEOS is the simplest prototype that contains
the main physical properties of pure fluids, repulsive and attractive
molecular effects, thereby facilitating the handling of the essential
physics and thermodynamics with a simple analytical formulation.  Thus
each fluid has its own thermodynamics.  For each phase the
thermodynamic state is determined by the SGEOS:
\begin{align}
  \label{E-R:97}
  e(p,\rho) &= \frac{p+\gamma p_{\infty}}{(\gamma -1) \rho} + q
  \\
  \label{E-R:98}
  \rho (p,T) &= \frac{p+p_{\infty}}{(\gamma -1) c_v T}
  \\
  \label{E-R:99}
  h(T) &= \gamma  c_v T + q
  \\
  \label{E-R:100}
  g(p,T) &= \left(\gamma c_v - q'\right) T - c_v T \ln \frac{T^\gamma}{\left(p+p_{\infty}\right)^{\gamma-1}} + q
\end{align}
where $T$, $h$, and $g$ are the temperature, enthalpy,
and Gibbs free enthalpy, respectively, of the phase considered.  In
addition to the three material constants mentioned above, two
additional material constants have been introduced, the constant
volume specific heat $c_v$ and the parameter $q'$.  The method to
determine these parameters in liquid-vapor systems, and in particular
the coupling of liquid and vapor parameters, is given
in~\cite{SGEOS}.  The values for water and its vapor from that
reference are given in Table 2.  These parameter values appear to
yield reasonable approximations over a temperature range from 298 to
473K.  For higher temperature range the parameters can easily be
refit.

Unlike van der Waals type modeling where mass transfer is a
thermodynamic path, with the seven equation two-phase model the mass
transfer modeling, which produces a relaxation toward thermodynamic
equilibrium, is achieved by a kinetic process.  Thus the seven equation
model preserves hyperbolicity during mass transfer.
From equation~\eqref{E-R:99} it is readily seen that the phase
$k$ specific enthalpy evaluated at the interface condition from
equation~\eqref{E-R:95} is
\begin{equation}
  h_{k,  int} = c_{p, k}  T_{int} + q_k
\end{equation}
because $c_{p, k} = \gamma_k  c_{v, k}$.

The bulk interphase mass transfer from the liquid phase to the vapor
phase $\Gamma$ is due to their difference in Gibb's free energy.  At
saturated conditions the Gibb's energies of the two-phases are equal.
It is necessary to determine the saturation temperature $T_{sat}(p)$
for given pressure $p=\bar{p}_{int}$ and the heat of vaporization
$L_v\left(T_{sat}(\bar{p}_{int}) \right)$ at this saturation temperature
with the SGEOS for each phase.  For this calculation the procedure
of~\cite{SGEOS} is adopted.  This procedure for the
determination of SGEOS parameters can be made very accurate provided
the two reference states are picked sufficiently close to represent
the experimental saturation curves as locally quasi-linear.
Restrictions occur near the critical point, but away from this point
wide ranges of temperatures and pressures can be considered.  At
thermodynamic equilibrium at the interface, the two phasic Gibbs free
enthalpies must be equal, $g_{vap}=g_{liq}$, so the use of equation~\eqref{E-R:100}
yields
\begin{equation}
  \label{E-R:102}
  \ln \left( p + p_{\infty,  vap} \right) = A + \frac{B}{T} + C  \ln(T) + D  \ln \left( p + p_{\infty,  liq} \right)
\end{equation}
where
\begin{align}
  A &= \frac{c_{p, liq} - c_{p, vap} + q'_{vap} - q'_{liq}}{c_{p,  vap} - c_{v,  vap}} \\
  B &= \frac{q_{liq}-q_{vap}}{c_{p,  vap} - c_{v,  vap}} \\
  C &= \frac{c_{p, vap} - c_{p, liq}}{c_{p,  vap} - c_{v,  vap}} \\
  D &= \frac{c_{p, liq} - c_{v, liq}}{c_{p,  vap} - c_{v,  vap}} \,\,.
\end{align}
Relation~\eqref{E-R:102} is nonlinear, but can used to compute the
theoretical curve $T_{sat}(p)$.  A simple Newton iterative numerical
procedure is used.  With $T_{sat}(p)$ determined, the heat of
vaporization is calculated as
\begin{align}
  \nonumber
  L_v \left( T_{int} \right) &= h_{vap,  int} - h_{liq,  int}
  \\
  \nonumber
  &= h_{k,  int}
  \\
  &= \left( \gamma_{vap}  c_{v, vap}  T + q_{vap} \right) - \left( \gamma_{liq}  c_{v, liq}  T + q_{liq} \right) \,.
\end{align}
%
%
%\subsection{Seven-Equation Two-Phase Flow Constitutive Models}
%Without additional closure equations the balance relations derived
%above are generic, i.e.\ they apply to all materials (fluids).  They
%must made to apply to the unique material (fluid) being considered --
%material specific.  Also, though averaging the microlevel
%balance equations led to the ``simplified'' or perhaps more tractable
%model above, this simplification (averaging) led to a loss of information,
%and some additional relations must also be specified to supply (or
%restore) at least some information that was lost in this
%process\footnote{The process of averaging the balance equations produced a
%system with more unknowns than equations; thus postulates or empirical
%correlations are required to resolve this deficiency.}.  Collectively,
%any additional relations, or sub-models, that must be specified to
%render mathematical closure (allowing a solution to be obtainable) to
%the generic balance equations are known as ``constitutive
%relations''.
%
%Because the seven equation two-phase model's most unique features are
%reflected in the presence of a volume fraction evolution equation,
%interfacial pressure and velocity, and mechanical relaxation terms
%involving pressure and velocity relaxation, it is natural to begin
%with their constitutive relations.  Constitutive ideas associated with
%the volume fraction evolution equation were discussed previously for
%pedagogical reasons.  Thermodynamical relaxation will be discussed
%subsequently, followed by other closures.

%\subsubsection{Wall and Interphase Friction}
%A simple wall friction model results from making the same assumptions
%as for single-phase duct flow with the exception that the duct wall
%area over which the shear stress acts is reduced by the fraction of
%the wall area which the phase occupies.  Thus
%\begin{equation}
%  F_{\text{wall friction}, k} = \frac{f_k}{2 d_h} \rho_k u_k \left|u_k \right| \alpha_k A
%\end{equation}
%for phases $k=(liq, vap)$, where $f_{k}$ is the wall friction factor
%associated with phase $k$.  The hydraulic diameter
%$d_h$ depends on the shape of the cross section, and the position $x$
%in the pipe.
%
%The friction force acting between the two phases due to their relative
%motion is also given in analogy to that of single-phase duct flow:
%\begin{equation}
%  F_{\text{friction}, k'} = f_{k, \, k'} \frac{1}{2} \rho_k (u_k - u_{int}) \left| u_k -u_{int} \right|  A_{int}  A
%\end{equation}
%for $k=(liq, vap)$, $k'=(vap, liq)$, with $f_{k, k'}$ denoting the friction factor acting upon phase $k$
%due to the (relative) motion of the other phase $k'$. 
%
%The frictional pressure drop in each phase will be different in 
%general due the different velocities of the two phases.  However, 
%because of the tendency toward pressure equilibrium between the phases 
%an effective pressure drop will be realized.
%
%===================================================================================
\section{A viscous regularization for the multi-D seven equation model}\label{sec:sev-equ-visc-reg-sect4}
%===================================================================================
In this section, the dissipative terms for the multi-D seven equation model \emph{with pressure and velocity relaxation source terms} are derived. The methodology proposed in  \chap{chap:theory_chp1} is followed. For clarity purpose, the seven equation model with pressure and velocity relaxation terms is recalled when considering a phase $k$ in interaction with a second phase $j$:
%
\begin{subequations}\label{eq:sev_equ}
\begin{align}
\partial_t \left( \alpha_k  A\right) + \mbold u_{int} A \grad \alpha_k = A \mu \left( P_k - P_j \right)
\end{align}
\begin{align}
\partial_t \left( \alpha_k \rho_k A \right) + \div \left( \alpha_k \rho_k \mbold u_k A \right) = 0
\end{align}
\begin{align}
\partial_t \left( \alpha_k \rho_k u_k A \right) + \div \left[ \alpha_k A \left( \rho_k \mbold u_k \otimes \mbold u_k + P_k \mathbb{I} \right) \right] &=\nonumber\\
\alpha_k P_k \grad A + P_{int} A \grad \alpha_k &+ A \lambda \left( \mbold u_j - \mbold u_k \right)
\end{align}
\begin{align}
\partial_t \left( \alpha_k \rho_k E_k A \right) + \div \left[ \alpha_k A \mbold u_k \left( \rho_k E_k + P_k \right) \right] &=\nonumber\\
P_{int} \mbold u_{int} A \grad \alpha_k - \mu \bar{P}_{int} \left( P_k-P_j \right) &+ \bar{\mbold u}_{int}A \lambda \left( \mbold u_j - \mbold u_k \right)
\end{align}
\end{subequations}
%
%where $\rho_k$, $u_k$, $E_k$ and $P_k$ are the density, the velocity, the specific total energy and the pressure of $k^{th}$ phase, respectively. The pressure and velocity relaxation parameters are denoted by $\mu$ and $\lambda$, respectively. The variables with index $_I$ correspond to the interfacial variables and a definition for those can be found in \cite{SEM}. The cross-section $A$ is only function of space: $\partial_t A = 0$.
In order to apply the EVM, dissipative terms are added to each equation of the system given in \eqt{eq:sev_equ}, which yields:
%
\begin{subequations}\label{eq:sev_equ-with-diss-terms}
\begin{align}\label{eq:sev_equ-with-diss-terms-vf}
\partial_t \left( \alpha_k  A\right) + \mbold u_{int} A \grad \alpha_k = A \mu_P \left( P_k - P_j \right) + \div \mbold l_k
\end{align}
\begin{align}\label{eq:sev_equ-with-diss-terms-cont}
\partial_t \left( \alpha_k \rho_k A \right) + \div \left( \alpha_k \rho_k \mbold u_k A \right) = \div \mbold f_k
\end{align}
\begin{align}\label{eq:sev_equ-with-diss-terms-mom}
\partial_t \left( \alpha_k \rho_k \mbold u_k A \right) + \div \left[ \alpha_k A \left( \rho_k \mbold u_k \otimes \mbold u_k + P_k \mathbb{I} \right) \right] &=\nonumber\\
\alpha_k P_k \grad A + P_{int} A \grad \alpha_k &+ A \lambda_u \left( \mbold u_j - \mbold u_k \right) + \div \mbold g_k
\end{align}
\begin{align}\label{eq:sev_equ-with-diss-terms-ener}
\partial_t \left( \alpha_k \rho_k E_k A \right) + \div \left[ \alpha_k A \mbold u_k \left( \rho_k E_k + P_k \right) \right] &=\nonumber\\
P_{int} \mbold u_{int} A \grad \alpha_k - \mu_P \bar{P}_{int} \left( P_k-P_j \right) &+ \bar{\mbold u}_{int}A \lambda_u \left( \mbold u_j - \mbold u_k \right) + \div \left( \mbold h_k + \mbold u \cdot \mbold g_k \right)
\end{align}
\end{subequations}
%
where $\mbold f_k$, $\mbold g_k$, $\mbold h_k$ and $\mbold l_k$ are the dissipative terms. The next step consists of deriving the entropy equation for the phase $k$, on the same model as what is done in \app{app:sev-equ-model-entropy}. Extra terms will appear in the right hand side of the entropy equation due to the dissipative terms. By choosing properly the definition of the dissipative terms, the sign of these extra terms can be controlled in order to ensure positivity of the entropy residual:
%
\begin{enumerate}
\item recast the system of equation given in \eqt{eq:sev_equ-with-diss-terms} in terms of the primitive variables $(\alpha_k, \rho_k, \mbold u_k, e_k)$.
\item derive the entropy equation by using the chain rule:
\begin{equation}
\label{eq:chain_rule-sct4}
\frac{ds_k}{dt} = \left( s_{\rho} \right)_k \frac{d \rho_k}{dt} + \left( s_{e} \right)_k \frac{d e_k}{dt} 
\end{equation}
where $\frac{d \cdot}{dt}$ is the material derivative. The terms $(s_e)_k$ and $(s_{\rho})_k$ denote the partial derivative of the entropy $s_k$ with respect to $e_k$ and $\rho_k$, respectively.
\item isolate the terms of interest and choose an appropriate expression for each of the dissipative terms in order to ensure positivity of the right-hand side.
\end{enumerate}
%
We first recast \eqt{eq:sev_equ-with-diss-terms} in terms of the primitive variables: the volume fraction equation remains unchanged. The equation for the primitive variable $\rho_k$ is derived by combining \eqt{eq:sev_equ-with-diss-terms-vf} and \eqt{eq:sev_equ-with-diss-terms-cont}:
%
\begin{equation}\label{eq:rho-7-eqn-model-sect4}
\alpha_k A \left[ \partial_t \rho_k + \left( \mbold u_k - \textcolor{blue}{\mbold u_{int}} \right) \grad \rho_k \right] = \textcolor{blue}{A \rho_k \mu_P \left( P_k - P_j \right)} + \div \mbold f_k - \rho_k \div \mbold l_k
\end{equation}
%
The velocity equation is obtained by subtracting the density equation from the momentum equation:
%
\begin{align}\label{eq:vel-7-eqn-model-sect4}
\alpha_k \rho_k  A \left[ \partial_t \mbold u_k + \mbold u_k \cdot \div \mbold u_k \right]  + \div \left( \alpha_k \rho_k A P_k \mathbb{I} \right) &=\nonumber\\
\textcolor{blue}{\alpha_k P_k \grad A + P_{int} A \grad \alpha_k + A \lambda \left( \mbold u_j - \mbold u_k \right)} &+ \div \mbold g_k - \mbold u_k \otimes \mbold f_k
\end{align}
%
After multiplying \eqt{eq:vel-7-eqn-model-sect4} by the velocity vector $\mbold u_k$, the resulting kinetic energy equation is subtracted from the total energy equation to obtain the internal energy equation for phase $k$:
%
\begin{align}\label{eq:int-ener-7-eqn-model-sect4}
\alpha_k \rho_k  A \left[ \partial_t \mbold e_k + \mbold u_k \cdot \div \mbold e_k \right]  + \alpha_k \rho_k A P_k \grad \mbold u_k &=\nonumber\\
\textcolor{blue}{P_{int} A \left(\mbold u_{int}-\mbold u_k \right) \cdot \grad \alpha_k} &-  \textcolor{blue}{\alpha_k P_k \mbold u_k \grad A} \nonumber \\ 
\textcolor{blue}{-\bar{P}_{int} A \mu_P \left(P_k-P_j \right)} &+ \textcolor{blue}{A \lambda_u \left(\mbold u_j-\mbold u_k  \right) \cdot \left(\bar{\mbold u}_{int}- \mbold u_k \right)}\nonumber \\
&+ \div \mbold h_k + \mbold g_k : \grad \mbold u_k + || \mbold u ||^2_k \mbold f_k
\end{align}
%
The blue terms in \eqt{eq:rho-7-eqn-model-sect4} and \eqt{eq:int-ener-7-eqn-model-sect4} yield the positive terms in the right hand-side of \eqt{eq:ent-eqn-7-eqn-model} and thus, are ignored in the remaining of the derivation. The entropy equation is now obtained by combining the density equation (\eqt{eq:rho-7-eqn-model-sect4}) and the internal energy equation (\eqt{eq:int-ener-7-eqn-model-sect4}) through the chain rule given in \eqt{eq:chain_rule-sct4} to yield:
%
\begin{equation}\label{eq:ent-res-7-eqn-diss-terms}
\alpha_k \rho_k A \frac{Ds_k}{Dt} = \left(s_e\right)_k \left[ \div \mbold h_k + \mbold g_k : \grad \mbold u_k +  \left( || \mbold u ||^2_k - e_k\right) \div \mbold f_k  \right] + (\rho s_\rho)_k \left[ \div \mbold f_k - \rho_k \div \mbold l_k \right],
\end{equation}
%
where it was assumed that the phase $k$ obeys the second thermodynamic law: $P_k (s_e)_k + \rho_k (s_\rho)_k = 0$. 
From this point, two options are available to us in order to derive the dissipative terms: either we consider the total entropy residual of the system by summing over \eqt{eq:ent-res-7-eqn-diss-terms}, or we can consider each phase independently. This dilemma can be answered by remembering that the seven equation model degenerates into the single phase flow equations in the limits $\alpha_k = 0,1$. Thus, the dissipative terms also have to be consistent with the single phase flow limits. As a result, it is chosen to derive the dissipative terms by considering each phase independently which will automatically ensure positivity of the total entropy residual as well.

The right-hand side of \eqt{eq:ent-res-7-eqn-diss-terms} can be further simplified by using the following expression
for the dissipative terms $\mbold f_k$,  $\mbold g_k$ and $\mbold h_k$:
\begin{align}\label{eq:def-diss-terms-sect4}
  \mbold f_k &= \tilde{\mbold f}_k + \rho_k \mbold  l_k 
  \\
  \mbold g_k &= \alpha_k \rho_k A \mu_k \mathbb{F}(\mbold u_k) + \mbold u_k \otimes \mbold f_k
  \\
  \mbold h_k &= \tilde{\mbold h}_k - \frac{|| \mbold u||^2 }{2} \mbold f_k + (\rho e)_k \mbold l_k,
\end{align}
Note the area function $A$ in the definition of $\mbold g$. It yields:
%
\begin{align}\label{eq:ent-res-7-eqn-diss-terms2}
&\alpha_k \rho_k A \frac{Ds_k}{Dt} = \nonumber \\
&\underbrace{\left(s_e\right)_k \alpha_k \rho_k A \mu_k \mathbb{F}(\mbold u_k) : \grad \mbold u_k}_{\mathcal{R}_1} +
\underbrace{\left[ \div \tilde{\mbold h}_k  - e_k \div \mbold f_k  \right] + (\rho s_\rho)_k \div \mbold f_k}_{\mathcal{R}_2} + \nonumber \\
&\underbrace{(s_e)_k \div \left( \rho_k e_k \mbold l_k \right) -  (s_e)_k e_k \div \left( \rho_k \mbold l_k \right) + \rho_k (s_\rho)_k \div \left( \rho_k \mbold l_k \right) 
  - \rho_k^2 (s_\rho)_k \div \mbold l_k}_{\mathcal{R}_3},
\end{align}
%
where $\mu_k$ is a positive viscosity coefficient for phase $k$. The right-hand side of \eqt{eq:ent-res-7-eqn-diss-terms2} is split into three terms for simplicity. \\

In \cite{SEM}, the entropy equation is derived for each phase, by assuming that there exists a phase wise entropy function $s_k$, function of the density $\rho_k$ and the specific internal energy $e_k$.
\begin{eqnarray}
\label{eq:sev_equ2}
\rho_k A \left( \partial_t s_k + u_k \partial_x s_k \right) = \frac{Z_j}{Z_j+Z_k} \lambda \left( u_k - u_j \right)^2 + \frac{Z_k}{Z_j+Z_k} \mu \left( P_k-P_j \right)^2 + \nonumber \\ 
\frac{Z_k}{Z_j+Z_k} \left| \partial_x \alpha \right| \left[ Z_j \left( u_k-u_j \right) +sgn\left( \partial_x \alpha \right) \left( P_k-P_j \right) \right]^2
\end{eqnarray}
where $Z_j = \rho_j c_j$ and $c_j$ represents the acoustic impedance and the speed of sound of phase $j$, respectively. The symbol $sgn(x)$ denotes a function that returns the sign of the quantity $x$.\\
From \eqt{eq:sev_equ2}, it is clear that the entropy minimum principle is verified since the right-hand side is only made of positive terms. In the rest of this document all of the source terms $r_x$ are dropped in order to simplify the derivation. It will not affect the final result, since all of the source terms combine in a sum of positive terms when deriving the entropy function. The phase index $k$ is also dropped.\\
By adding dissipative terms to each equation, the entropy equation gets modified: extra terms will appear in the right-hand side, function of the dissipative terms. The sign of these new terms need to be studied in order to conserve positivity of the right-hand side. This can be achieved by following these steps:
The first step consists of recasting the system of equation in terms of the primitive variables:
\begin{equation}
\label{eq:sev_equ3-bis}
\left\{
\begin{array}{llll}
A \partial_t  \alpha + u_I A \partial_x \alpha =  \partial_x \left( \alpha A l \right)\\
\alpha A \left( \partial_t \rho + u \partial_x  \rho \right) + \rho \alpha \partial_x \left( u A \right) + \Gamma = \partial_x\left( \alpha A f \right) - \rho \partial_x \left( \alpha A l \right) \\
\alpha \rho A \left( \partial_t u + u \partial_x u \right)  + \partial_x \left(\alpha P A \right) =  \alpha P \partial_x A + \partial_x\left( \alpha A g \right) - u \partial_x \left( \alpha A f \right) \\
\alpha \rho A  \left( \partial_t e +u  \partial_x e \right) + \alpha P A \partial_x u + \alpha u P \partial_x A = \partial_x \left( \alpha A h \right) + \alpha A g \partial_x u + \left( \frac{u^2}{2}-e \right) \partial_x \left( \alpha A f \right)
\end{array}
\right.
\end{equation}
where $\Gamma=\rho A \left( u - u_I  \right)\partial_x \alpha$. The function $\beta$ can be ignored since it is used to get the right-hand side of \eqt{eq:sev_equ2} . As a result, the continuity, velocity and internal energy equations get simpler:
\begin{equation}
\label{eq:sev_equ4}
\left\{
\begin{array}{llll}
A \partial_t  \alpha + u_I A \partial_x \alpha =  \partial_x \left( \alpha A l \right)\\
\alpha A \left( \partial_t \rho + u \partial_x  \rho \right) = \partial_x \left( \alpha A f \right) - \rho \partial_x \left( \alpha A l \right) \\
\alpha \rho A \left( \partial_t u + u \partial_x u \right)  + \partial_x \left(\alpha P A \right) =  \partial_x \left( \alpha A g \right) - u \partial_x \left( \alpha A f \right) \\
\alpha \rho A  \left( \partial_t e +u  \partial_x e \right) + \alpha P A \partial_x u = \partial_x \left( \alpha A h \right) + \alpha A g \partial_x u + \left( \frac{u^2}{2}-e \right) \partial_x \left( \alpha A f \right)
\end{array}
\right.
\end{equation}
The continuity and internal energy equations can be combined using the chain rule given in \eqt{eq:chain_rule} to get the entropy equation:
\begin{eqnarray}
\label{eq:ent_equ2}
\alpha \rho A \frac{ds}{dt} + \alpha \left( \rho s_{\rho} + Ps_e \right) \left( A \partial_x u + u \partial_x A \right) =  \nonumber \\ 
s_e \left[ \partial_x \left( \alpha A h \right) + \alpha A g \partial_x u + \left( \frac{u^2}{2}-e \right) \partial_x \left( \alpha A f \right) \right] 
+ \rho s_{\rho} \left( \partial_x \left( \alpha A f \right) - \rho \partial_x \left( \alpha A l \right) \right) 
\end{eqnarray}
The above equation, \eqt{eq:ent_equ2}, can be split into four different terms. For clarity, let us consider each of this term separately in a first approach.\\ The first term of the left-hand side is the lagrange (or material) derivative of the entropy function $s$. It does not need to be modified since its sign has to be determined by looking at the other terms. \\The second term in the left-hand side is usually set to zero by assuming that $\rho s_{\rho} +P s_e =0$. Any other alternative would require the term $\rho^2 s_{\rho} +P s_e$ to be function of the velocity $u$ or its derivatives, and the cross-section $A$, in order to be able to determine its sign. In addition, any entropy function obeying to the relation $\rho s_{\rho} +P s_e =0$, also obeys to the second thermodynamic law:
\begin{equation}
\label{eq:second_th}
\left\{
\begin{array}{l}
Tds = de - \frac{P}{\rho^2} d\rho \Rightarrow s_e = \frac{1}{T} \geq 0 \text{, } s_{\rho} = -s_e \frac{P}{\rho^2}
\end{array}
\right.
\end{equation}
where $T$ is the fluid temperature. \\
The right-hand side of \eqt{eq:ent_equ2} is more difficult to handle. It requires further assumptions in the definition of the dissipative terms $h$ and $g$.  
The first term of the right-hand side can be simplified by using the following expression for the dissipative terms $h$ and $g$:
\begin{eqnarray}
\boxed{
\left\{
\begin{array}{ll}
g = \mu \partial_x u + u f \\
h = \tilde{h} - \frac{u^2}{2} f,
\end{array}
\right.}
\end{eqnarray}
which results in:
\begin{eqnarray}
\label{eq:ent_equ3}
s_e \left[  \partial_x \left( \alpha A h \right) + \alpha A g \partial_x u + \left( \frac{u^2}{2}-e \right) \partial_x \left( \alpha A f \right)  \right] = \\
s_e \left[ \partial_x \left( \alpha A \tilde{h} \right) - e \partial_x \left( \alpha A f \right) \right] + s_e \mu \left( \partial_x u \right)^2
\end{eqnarray}
where $\mu$ is a positive viscosity coefficient, and $\tilde{h}$ is a new dissipative term. In \eqt{eq:ent_equ3}, it is noted that the term $s_e \mu (\partial_x u)^2$ is always positive and does not need any further modification. Thus, it remains to determine the sign of the other term $s_e\left( \partial_x \left( \alpha A \tilde{h} \right) - e \partial_x \left( \alpha A f \right) \right)$, along with $\rho s_{\rho} \left( \partial_x \left( \alpha A f \right) - \rho \partial_x \left( \alpha A l \right) \right)$, that are now independent of the velocity $u$. Then, we define the variable $rhs$ as the following:
\begin{equation}
\label{eq:rhs1}
rhs = s_e \left( \partial_x \left( \alpha A \tilde{h} \right)-e \partial_x \left( \alpha A f \right) \right) + \rho s_{\rho} \left( \partial_x \left( \alpha A f \right) - \rho \partial_x \left( \alpha A l \right) \right)
\end{equation}
Before going further in the derivation, let us remind that all of the above steps are valid for any phase of the system under consideration, and that the objective is to prove the entropy minimum principle. The remain of the derivation can be treated in at least two different ways. Either we enforce positivity of the entropy residual for each phase (phase wise treatment), or, we consider the system as a whole and enforce positivity of the total entropy residual which requires to sum over the phases. The latest version is probably  better from a physical point of view. The entropy of each phase can actually either decrease or increase through a transient because of the heat and mass exchanges since each phase is not a close and isolated system. Therefore, even though positivity of the entropy residual for each phase can be achieved, it would not make physically sense. In addition, the relaxation and source terms were derived by considering the total entropy of the system (ref). Then, it would be wise to follow the same reasoning for consistency. Both options are studied in this report in order to compare solutions and verify our initial assumptions. It is noted that derivation of the dissipative terms is not unique.
%First, dissipative terms are added to each equation of the system under interest. Then, the entropy equation is re-derived with the dissipative terms that are determined by using the entropy inequality as a condition. 
%===================================================================================
\section{Low-Mach asymptotic limit and viscosity coefficients}\label{sec:sev-equ-visc-coeff-sect4}
%===================================================================================
%===================================================================================
\section{Numerical results:}\label{sec:1d-num-res-sect4}
%===================================================================================
Because it is not economical to solve the entire two-phase flow field
with highly resolved three-dimensional computational fluid dynamics for an
entire light water reactor coolant system,
it is necessary to construct a one-dimensional model for flow in
pipes, nozzles, and other components.  The one-dimensional model is
constructed to allow the representation of continuously variable
cross-sectional area.

Consider flow through a duct with local cross-sectional area
$A=A(x,t)$.  Actually, most of the time we consider local
cross-sectional area to depend upon position coordinate $x$ only,
for which a time rate of change of cross-sectional area is not
necessary because for this case $\frac{\partial A}{\partial
t} = 0$.  However, $A(x,t)$ is left inside the time derivative terms
for generality and possible future use.  The seven-equation two-phase
system model can be stated as balances of mass, momentum, and total energy,
along with volume fraction evolution as
\begin{align}
  % liquid mass conservation
  \label{E-R:74}
  \frac{\partial \left( \alpha \rho \right)_{liq} A}{\partial t}
  + \frac{\partial \left( \alpha \rho u \right)_{liq} A}{\partial x}
  &= - \Gamma A_{int} A
  \\
  % liquid momentum
  \nonumber
  \frac{\partial \left( \alpha \rho u \right)_{liq} A}{\partial t}
  + \frac{\partial \alpha_{liq} A \left( \rho u^2 + p \right)_{liq} }{\partial x}
  &= p_{int} A \frac{\partial \alpha_{liq}}{\partial x} + p_{liq} \alpha_{liq} \frac{\partial A}{\partial x}
  \\
  \nonumber
  &+ A \lambda_u (u_{vap} - u_{liq})
  \\
  \nonumber
  &- \Gamma A_{int} u_{int} A
  \\
  \nonumber
  &- F_{\text{wall friction}, liq} - F_{\text{friction}, vap}
  \\
  &+ \left( \alpha \rho \right)_{liq} A \vec{g} \cdot \vec{n}_{axis}
\end{align}
\begin{align}
  % liquid total energy
  \nonumber
  \frac{\partial \left( \alpha \rho E \right)_{liq} A}{\partial t}
  + \frac{\partial \alpha_{liq} u_{liq} A \left( \rho E + p \right)_{liq}}{\partial x}
  &= p_{int} u_{int} A \frac{\partial \alpha_{liq}}{\partial x} - \bar{p}_{int} A \mu_P (p_{liq} - p_{vap})
  \\
  \nonumber
  &+ \bar{u}_{int} A \lambda_u (u_{vap} - u_{liq})
  \\
  \nonumber
  &+ \Gamma A_{int} \left( \frac{p_{int}}{\rho_{int}} - H_{liq, int} \right) A
  \\
  &+ Q_{int, liq} + Q_{\text{wall}, liq}
  \\
  % liquid volume fraction
  \label{eqn:7eqn_va_alpha_liq}
  \frac{\partial \alpha_{liq} A}{\partial t} + u_{int} A \frac{\partial \alpha_{liq}}{\partial x}
  &= A \mu_P (p_{liq} - p_{vap}) - \frac{\Gamma A_{int} A}{\rho_{int}}
\end{align}
for the liquid phase, and
\begin{align}
  % vapor mass conservation
  \frac{\partial \left( \alpha \rho \right)_{vap} A}{\partial t}
  + \frac{\partial \left( \alpha \rho u \right)_{vap} A}{\partial x}
  &=  \Gamma A_{int} A
  \\
  % vapor momentum
  \nonumber
  \frac{\partial \left( \alpha \rho u \right)_{vap} A}{\partial t}
  + \frac{\partial \alpha_{vap} A \left( \rho u^2 + p \right)_{vap} }{\partial x}
  &= p_{int} A \frac{\partial \alpha_{vap}}{\partial x} + p_{vap} \alpha_{vap} \frac{\partial A}{\partial x}
  \\
  \nonumber
  &+ A \lambda_u (u_{liq} - u_{vap})
  \\
  \nonumber
  &+ \Gamma A_{int} u_{int} A
  \\
  \nonumber
  &- F_{\text{wall friction}, vap} - F_{\text{friction}, liq}
  \\
  &+ \left( \alpha \rho \right)_{vap} A \vec{g} \cdot \vec{n}_{axis}
\end{align}
\begin{align}
  \nonumber
  % vapor total energy
  \frac{\partial \left( \alpha \rho E \right)_{vap} A}{\partial t}
  + \frac{\partial \alpha_{vap} u_{vap} A \left( \rho E + p \right)_{vap}}{\partial x}
  &= p_{int} u_{int} A \frac{\partial \alpha_{vap}}{\partial x} - \bar{p}_{int} A \mu_P (p_{vap} - p_{liq})
  \\
  \nonumber
  &+ \bar{u}_{int} A \lambda_u (u_{liq} - u_{vap})
  \\
  \nonumber
  &- \Gamma A_{int} \left( \frac{p_{int}}{\rho_{int}} - H_{vap, int} \right) A
  \\
  &+ Q_{int, vap} + Q_{\text{wall}, vap}
  \\
  % vapor phase volume fraction
  \label{E-R:81}
  \frac{\partial \alpha_{vap} A}{\partial t} + u_{int} A \frac{\partial \alpha_{vap}}{\partial x}
  &= A \mu_P (p_{vap} - p_{liq}) + \frac{\Gamma A_{int} A}{\rho_{int}}
\end{align}
%for the vapor phase.  It is noted that for two-phase flow,
%either of the differential relations~\eqref{eqn:7eqn_va_alpha_liq}
%or~\eqref{E-R:81} may be replaced with the algebraic relation
%
%throughout, reducing the total number of equations to be solved to seven.
%
%In equations~\eqref{E-R:74}--\eqref{E-R:81}, $\Gamma$ is the net mass transfer per unit
%interfacial area from the liquid to the vapor phase and $A_{int}$ is
%the interfacial area per unit volume of mixture.  Also, $H_{liq, int}$
%and $H_{vap, int}$ are the liquid and gas total enthalpies at the
%interface, respectively.  The nomenclature has also been modified so
%that now $u_{int}$ and $\bar{u}_{int}$ are, respectively, the
%interfacial velocity and average interfacial velocity; and $p_{int}$
%and $\bar{p}_{int}$ are, respectively, the interfacial pressure and
%average interfacial pressure.  In the momentum balance equations
%$\vec{n}_{axis}$ is the unit vector directly along the axis of the
%duct, which is also the $\pm$ flow direction.  Of course
%$F_{\text{wall friction}, k}$ is the frictional force due to the wall acting on phase
%$k$ and $F_{\text{friction}, k'}$ is the frictional force acting on
%phase $k$ due to the presence of the other phase $k'$.
%Similarly, $Q_{int, k}$ is the direct heat transfer from the interface
%to phase $k$ and $Q_{\text{wall}, k}$ is the direct heat transfer from
%the wall to phase $k$.
%
%Equation system~\eqref{E-R:74}--\eqref{E-R:81} is the basic system
%solved with RELAP-7.  The system was implemented within the MOOSE
%computational framework following a series of logically-complete
%steps~\cite{Berry_2013} designed to confidently allow physically- and
%mathematically-meaningful benchmark testing at each step of increased
%complexity.  This seven equation two-phase model allows both phases to be
%compressible.
%\section{This is a Very Long Section Title This is a Very Long Section Title This is a Very Long Section Title }