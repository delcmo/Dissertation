%%%%%%%%%%%%%%%%%%%%%%%%%%%%%%%%%%%%%%%%%%%%%%%%%%%
%
%  New template code for TAMU Theses and Dissertations starting Fall 2012.  
%  For more info about this template or the 
%  TAMU LaTeX User's Group, see http://www.howdy.me/.
%
%  Author: Wendy Lynn Turner 
%	 Version 1.0 
%  Last updated 8/5/2012
%
%%%%%%%%%%%%%%%%%%%%%%%%%%%%%%%%%%%%%%%%%%%%%%%%%%%

%%%%%%%%%%%%%%%%%%%%%%%%%%%%%%%%%%%%%%%%%%%%%%%%%%%%%%%%%%%%%%%%%%%%%%
%%                           CONCLUSIONS
%%%%%%%%%%%%%%%%%%%%%%%%%%%%%%%%%%%%%%%%%%%%%%%%%%%%%%%%%%%%%%%%%%%%%%

%%%%%%%%%%%%%%%%%%%%%%%%%%%%%%%%%%%%%%%%%%%%%%%%%%%%%%%%%%%%%%%%%%%%%%
\chapter{\uppercase {Conclusions.}}\label{chap:conclusion}
%%%%%%%%%%%%%%%%%%%%%%%%%%%%%%%%%%%%%%%%%%%%%%%%%%%%%%%%%%%%%%%%%%%%%%
%A new version of the entropy viscosity method that is valid for a wide range of Mach numbers has been derived 
%and presented for the inviscid Euler equations.
%In the future, we plan to further extend the entropy viscosity method to the seven-equation two-phase flow fluid model \cite{SEM}. 
%This two-phase flow system of equations is a good candidate for two reasons: it is unconditionally hyperbolic and degenerates to the standard Euler equations when one phase disappears.\\

The entropy viscosity method has been successfully applied to three hyperbolic system of equations: the multi-D Euler equations, the $1$-D seven-equation two-phase model and the $1$-D grey radiation-hydrodynamic equations. The numerical method was implemented using a continuous Galerkin finite element method and a second-order implicit temporal solver. The method relies on the derivation of dissipative terms consistent with the entropy inequality in order to ensure uniqueness of the numerical solution and on the definition of smart viscosity coefficients that are able to detect shock waves and discontinuities, allowing second-order accuracy when the numerical solution is smooth. More precisely, the viscosity coefficients are defined proportional to the entropy residual that is known to be peaked in the shock region, and also to the inter-element jumps that will allow detection of other discontinuities. The definition of the viscosity coefficients also requires a normalization parameter that is derived using the non-dimensionalized form of the hyperbolic system of equations under consideration, in order to have well-scaled dissipative terms.

This approach allowed us to derive and present a new version of the entropy viscosity method valid for a wide range of Mach number when applying the entropy viscosity method to the multi-D Euler equations. The definition of the viscosity coefficients
is now consistent with the low-Mach asymptotic limit, does not require an analytical expression 
for the entropy function, and is therefore applicable to a larger variety of flow regimes, from very 
low-Mach flows to supersonic flows. 
The method has also been extended to Euler equation with variable area to solve nozzle flow problems.
In 1-D, convergence of the numerical solution to 
the exact solution was demonstrated by computing the convergence rates of the $L_1$ and $L_2$ norms 
for flows in a converging-diverging nozzle and in straight pipes. For smooth solutions, second-order 
convergence was verified; solutions with shocks converged with the expected theoretical rates of 1 (L$_1$-norm)
and 0.5 (L$_2$-norm).

The effectiveness of the method was also demonstrated in 2-D using a series of benchmark problems
for both subsonic and supersonic flows in various geometries, with Mach numbers ranging from $10^{-7}$ to 
2.5. For very low-Mach flows, we numerically verified that the pressure fluctuations were proportional to 
the square of the Mach number, as expected in the incompressible limit.

%The definition of the viscosity coefficients is now consistent with the low-Mach asymptotic limit, does not require an analytical expression of the entropy function, and thus, could be used with any equation of state having a convex entropy. Tests were performed with the Ideal and Stiffened Gas equation of states. In 1-D, convergence of the numerical solution (either smooth or with shocks) to the exact solution was demonstrated by computing the convergence rates of the L$1$ and L$2$ norms of the error for flows in convergence-divergent nozzle and a straight pipe. 2-D simulations were also performed for both subsonic and supersonic flows, and various geometries: the entropy viscosity method behaves well for a wide range of Mach number. The numerical results obtained for a flow over a circular bump (subsonic and transonic flows) illustrates the capabilities of the method to adapt to the flow type.
The effect of source terms onto the entropy viscosity method was also investigated and justifications were provided on how to account for the source terms in the definition of the viscosity coefficients. $1$-D tests were performed for a simple model of a PWR using RELAP-7, and showed promising results.\\

The entropy-viscosity method was also applied to the $1$-D seven-equation two-phase model through the same theoretical approach as for the multi-D Euler equation. After deriving the viscous regularization using the entropy minimum principle for each phase, a definition for the viscosity coefficients was derived consistent with the low-Mach asymptotic limit and also with the single-phase limit cases $\alpha \to 0$ and $\alpha \to 1$ for the multi-D seven-equation model. Particular attention was given to the volume fraction equation whom dissipative term and the associated viscosity coefficient were determined by analogy with Burger's equation. 
Numerical tests showed that the numerical method behaves as expected for various $1$-D shock tubes and also various geometries. The stabilization method does not create any artificial mixture waves and can effectively resolve shocks and other discontinuities in the two limit cases: with and without relaxation terms. The numerical solutions compared well against either the exact solution when available, or solutions from other numerical methods.\\

Furthermore, we have also shown that the entropy-based viscosity method is a valid candidate for solving the 1-D radiation-hydrodynamic equations. A theoretical derivation is given for the derivation of the dissipative terms that are consistent with the entropy minimum principle. The viscosity coefficient $\kappa$ is defined proportional to the entropy residual that measures the local entropy production allowing detection of shocks. Through the manufactured solution method, it is demonstrated, firstly, that second-order accuracy is achieved when the solution is smooth, and secondly, that the artificial dissipative terms do not affect the physical solution in the equilibrium-diffusion limit. 
The entropy-based numerical scheme also behaves well in the tests performed for Mach numbers ranging from $1.05$ to $50$. The main features such as the embedded hydrodynamic shock and the Zeldovich spike are resolved accurately without spurious oscillations. The viscosity coefficient is peaked in the shock region only and behaves as expected. All of these results were obtained by using an unique definition of the viscosity coefficient that is computed on the fly. The addition of dissipative terms to the set of equations requires more computational work but is rather simple to implement.\\

As future work, extension to multi-dimensional geometries tests should be considered for both the seven-equation model and the radiation-hydrodynamic equations. 
All of the derivations presented in this dissertation hold. The definition of the viscosity coefficients do not need to be modified and the viscous regularizations were derived in the multi-D case for both system of equations. The multi-D seven equation model will require a preconditioner accounting for the relaxation terms when using a non-linear solver. As for the radiation-hydrodynamic equations, it would also be interesting to model the radiation equation with an $S_n$ transport approximation and apply the entropy based artificial viscosity to the resultant radiation-hydrodynamics equations. Given the advective nature of the $S_n$ equations, dissipation would need to be added to these equations.