%%%%%%%%%%%%%%%%%%%%%%%%%%%%%%%%%%%%%%%%%%%%%%%%%%%
%
%  New template code for TAMU Theses and Dissertations starting Fall 2012.  
%  For more info about this template or the 
%  TAMU LaTeX User's Group, see http://www.howdy.me/.
%
%  Author: Wendy Lynn Turner 
%	 Version 1.0 
%  Last updated 8/5/2012
%
%%%%%%%%%%%%%%%%%%%%%%%%%%%%%%%%%%%%%%%%%%%%%%%%%%%

%%%%%%%%%%%%%%%%%%%%%%%%%%%%%%%%%%%%%%%%%%%%%%%%%%%%%%%%%%%%%%%%%%%%%%
%%                           SECTION I
%%%%%%%%%%%%%%%%%%%%%%%%%%%%%%%%%%%%%%%%%%%%%%%%%%%%%%%%%%%%%%%%%%%%%
%
\chapter{\uppercase {Hyperbolic conservation laws}}
%
In this chapter, some key properties of the hyperbolic conservation laws are recalled. The objective is to introduce the reader to the notion of shock, weak solution and entropy minimum principle by studying a simple hyperbolic scalar equation through a linear progression. First, the mathematical properties of the hyperbolic scalar equation are studied and include the derivation of the eigenvalue and the characteristic equation. Then, it is explained how shock are formed which will bring us to show that the solution is non-unique. The reader will be introduced to the notion of weak solution. Lastly, it will be explained how the entropy condition is used to ensure uniqueness of the weak solution and more importantly, convergence of the numerical solution to the physical one. In the last section of this chapter, the notions introduced for the hyperbolic scalar equation are generalized to the hyperbolic systems of equation.
%%%%%%%%%%%%%%%%%%%%
\section{Hyperbolic scalar equations:}\label{sec:hyp_scalar_sct1b}
%%%%%%%%%%%%%%%%%%%%
The study of a hyperbolic scalar equation is detailed in order to provide the reader with a better understanding of its mathematical properties that are useful to the comprehension of shock formation among others.
%=============================
\subsection{Eigenvalue and characteristic curves:}\label{sec:mat_ppr_sct1b}
%=============================
For academic purpose, we consider a simple hyperbolic scalar equation with initial conditions to form what is called an \emph{Initial Boundary Value Problem} (IBVP) as shown in \eqt{eq:ivp_sct1b}. We define a computational domain $\Omega$ of dimension $d$, bounded by the boundary $\Gamma$ of dimension $d-1$. Each variable is assumed to be a function of space, $\mbold{r} \in R^d$, and time $t \in R_+$.
%
\begin{equation}\label{eq:ivp_sct1b}
\left\{
\begin{array}{l}
\partial_t u(\mbold{r},t) + \div \mbold f(u) = 0 \text{, } \left( \mbold{r}, t \right) \in R^d \times R_+  \\
u(\mbold{r},0) = u( \mbold{r}_0) 
\end{array}
\right.
\end{equation}
%
where $u$ and $\mbold f(u)$ are the solution and the inviscid flux, respectively. The inviscid flux $\mbold f(u)$ is assumed to be a differential function of the solution $u$. Two definitions of the inviscid flux will be considered in this chapter in order to illustrate the differences between linear and non-linear hyperbolic scalar systems: a linear flux $\mbold f_1$ and a non-linear flux $\mbold f_2$, as follows:
%
\begin{subequations}\label{eq:ivp2_sct1b}
%
\begin{equation}\label{eq:trans_sct1b}
\partial_t u(\mbold{r},t) + \div \mbold f_1(u) = \partial_t u(\mbold{r},t) + \div \left( au \right) = 0
\end{equation}
%
\begin{equation}\label{eq:burger_sct1b}
\partial_t u(\mbold{r},t) + \div \mbold f_2(u) = \partial_t u(\mbold{r},t) + \div \left( \frac{u^2}{2} \right) = 0
\end{equation}
%
\end{subequations}
%
\eqt{eq:trans_sct1b} and \eqt{eq:burger_sct1b} are respectively known as the linear advection and Burger's equations. They have been widely studied in the literature and are well understood (REFS). 

The eigenvalue, denoted by $\lambda$, of the hyperbolic equation is obtained by looking at the jacobian matrix of the inviscid flux, $\mbold f(u)$, with respect to the solution $u$, and corresponds to the wave propagation speed. When considering the fluxes $\mbold f_1$ and $\mbold f_2$, it is found that their eigenvalues are $\lambda_1 = a$ and $\lambda_2 = u$, respectively. For the linear advection equation, the wave speed is a constant through the computational domain. On the other hand, the wave speed is a function of space and time for the Burger's equation, since equal to the solution itself.

Once the eigenvalues are determined, the next step consists of deriving the characteristic equation and the characteristic curves. For simplicity purpose, we limit ourself to the $x-t$ plane. Under this assumption, characteristic curves are defined as curves $x = x(t)$ in the plane $x-t$ along with the PDE becomes an ODE \cite{Toro}. To determine the characteristic curves, \eqt{eq:ivp_sct1b} is recast as a function of the eigenvalue, $\lambda$, by using the chain rule as shown in \fig{eq:ivp3_sct1b}.
%
\begin{eqnarray}\label{eq:ivp3_sct1b}
&&\partial_t u(\mbold{r},t) + \frac{df}{du}\partial_x u = 0 \nonumber\\
&&\partial_t u(\mbold{r},t) + \lambda \partial_x u = 0 \nonumber \\
&&\partial_t u(\mbold{r},t) + \frac{dx}{dt} \partial_x u = 0 \text{ along } \frac{dx}{dt} = f'(u) = \lambda 
\end{eqnarray}
%
\eqt{eq:ivp3_sct1b} represents the rate of change of the solution $u$ along the curve $\frac{dx}{dt} = f'(u) = \lambda$. It tells us that the solution $u$ is constant along the curve $\frac{dx}{dt} = \lambda$ since its rate of change is zero. The eigenvalue is the slope of the characteristic curve and is referred to as the characteristic speed. 
For a given characteristic curve, the characteristic speed is a constant, since the solution $u$ is constant as well, and given by the initial condition, $f'(u(x,0))=f'(u_0)$ which allows us to integrate in order to obtain an analytical expression for $x(t)$:
%
\begin{eqnarray}\label{eq:ivp4_sct1b}
\frac{dx}{dt} &=& f'(u_0) \nonumber \\
\Leftrightarrow x(t) &=& x_0 + f'(u_0)t
\end{eqnarray}
%
where we set $x(t=0) = x_0$ that can be seen as the initial position of a particle traveling along the characteristic curve. It is current to represent the characteristic curves in a $t-x$ plane and example will be given for the linear advection and Burger's equation. \eqt{eq:ivp4_sct1b} informs us on the position $x$ of a particle carrying the initial value $u_0$ for each time $t$. Now, assuming that the initial value of the solution is $u_0(x_0)$ along the characteristic curve passing through the point $x_0$ given by \eqt{eq:ivp4_sct1b}, the solution $u(x,t)$ at position $x$ and time $t$ can be expressed as follows:
%
\begin{equation}\label{eq:ivp5_sct1b}
u(x,t) = u_0(x_0) = u_0(x - f'(u_0)t)
\end{equation}
%
\eqt{eq:ivp5_sct1b} can be seen as an analytical solution of the hyperbolic scalar equation (\eqt{eq:ivp_sct1b}). It is also understood that the derivative of the flux that corresponds to the eigenvalue of the system, has profound consequences on the behavior of the solution as explained in \sect{sec:shock_form_sct1b}. 
%=============================
\subsection{Formation of shocks and vanishing viscosity equation/solution}\label{sec:shock_form_sct1b}
%=============================
Hyperbolic scalar equations are known to develop shocks even from a smooth initial profile. This section aims at detailing how shocks form based on the mathematical properties introduced in \sect{sec:mat_ppr_sct1b} and the two examples of \eqt{eq:trans_sct1b} and \eqt{eq:burger_sct1b}, i.e., the $1$-D linear advection and Burger's equations.\\

When considering the $1$-D linear advection equation recalled in \eqt{eq:trans_sct1b} with the flux $f_1(u) = au$, the eigenvalue is found equal to $\lambda_1=a$ and constant. Thus, the slope of the characteristic curve remains constant and each particle travels at the same velocity through the computational domain. In other word, the initial profile $u_0(x)$ of the solution is simply translated at speed $a$ to the right if $a \geq 0$ and to the left if $a \leq 0$. Obviously, if $a=0$, the flux is also null and the solution remains static. A representation of the characteristic curve for the linear advection equation, \eqt{eq:trans_sct1b}, is given in \fig{fig:char_curve_sct1b} in a $t-x$ plane: all of the characteristic curves are parallel since their slope is given by the eigenvalue that is constant for the linear advection equation.
%
\begin{figure}[H]
\centering
\includegraphics[width=\textwidth]{figures/charact_curves_linear_transport.png}
\caption{Characteristic curves for the linear advection equation.}
\label{fig:char_curve_sct1b}
\end{figure}
%
In the case of the $1$-D Burger's equation, the eigenvalue is no longer constant and equal to the solution itself $\lambda_2 = u(x,t)$. The slope of the characteristic curve is now a function of space and more precisely of the initial solution $u_0$ which requires to look at two distinct cases: a constant and a non-constant initial solution. In the former case, the slope of the characteristic is constant which is the same situation as the linear advection equation previously investigated. In the later case, the characteristic curve will not have the same slope and thus, may intersect. When two characteristic curves intersect, it means that at a given time and position, two values of the solution are allowed (each characteristic curve carries different initial values of the solution): the solution displays an infinite gradient also called shock wave as shown in \fig{fig:char_curve_bg_sct1b}. 
%
\begin{figure}[H]
\centering
\includegraphics[width=\textwidth]{figures/shock_formation_burger.png}
\caption{Characteristic curves for the $1$-D Burger's equation.}
\label{fig:char_curve_bg_sct1b}
\end{figure}
%
The time the shock occurs at, $T_{shock}$, can be analytically determined. Let consider a $1$-D non-linear flux $f(u)$ and two characteristic curves originating from the position $x_0$ and $x_0+dx$ carrying the initial values $u_0(x_0)$ and $u_0(x_0+dx)$, respectively. The characteristic curves are:
%
\begin{eqnarray}\label{eq:cc1_sct1b}
&&x_1(t) = x_0 + f'(u_0(x_0)) t \nonumber \\ 
&&x_2(t) = (x_0 + dx) + f'(u_0(x_0+dx)) t 
\end{eqnarray}
%
Let now assume that the two characteristic curves intersect at time $T_{shock}$, which implies $x_1(T_{shock}) = x_2(T_{shock})$. Using \eqt{eq:cc1_sct1b}, it yields:
%
\begin{equation}\label{eq:cc1b_sct1b}
x_0 + f'(u_0(x_0)) T_{shock} = (x_0 + dx) + f'(u_0(x_0+dx)) T_{shock}
\end{equation}
%
From \eqt{eq:cc1b_sct1b} and after a few lines of calculation, an expression for $T_{shock}$ can be derived as follows: 
\begin{equation}\label{eq:cc2a_sct1b}
T_{shock} = \frac{-dx}{f'(u_0(x_0+dx))-f'(u_0(x_0))}
\end{equation} 
Taking the limit $dx \to 0$ and using the definition of the derivative, \eqt{eq:cc2a_sct1b} becomes:
\begin{equation}\label{eq:cc2_sct1b}
T_{shock} = \frac{-1}{f''(u_0) u'_0} = \frac{-1}{f'(u_0)},
\end{equation} 
The trivial case $f''(u) = 0$ is ruled out since it was assumed a non-linear flux. However, let assume that the flux is linear. In that case, the second-order derivative is null and by taking the limit of \eqt{eq:cc2_sct1b}, it yields $T_{shock} \to \infty$ which means that a shock wave never forms. This result is consistent with the conclusion made earlier in this section when studying the linear advection equation (\eqt{eq:trans_sct1b}). Going back to a non-linear flux, breaking first occurs where the derivative of the flux is negative (to ensure positivity of $T_{shock}$) and its absolute value is maximum (so that $\frac{1}{f'(u_0)}$ is minimal). Once the shock is formed, at a time $t \geq T_{shock}$, more than two characteristic curves may intersect leading to a tripled-value situation as shown in \fig{fig:triple_pt_bg_sct1b}. 
%
\begin{figure}[H]
\centering
\includegraphics[width=\textwidth]{figures/charact_curves_burger.png}
\caption{Example of a tripled-value situation.}
\label{fig:triple_pt_bg_sct1b}
\end{figure}
%
In this case, uniqueness of the solution is not ensured since for a given position and time the solution admits three values. This particular phenomenon makes sense when solving the shallow-water equation that is used to model a breaking wave on a sloping beach. However, when considering a gas flow, uniqueness of the thermodynamic properties is required to ensure a point-wise singled-value density. From the last example, we understand that preventing the tripled-value situation from forming may be the key to obtain the correct physical behavior when solving hyperbolic scalar and system of equations alike Euler equations. A solution to this problem could come from the study of an advection-diffusion equation type that is used, for instance, to model the propagation of particles in a material by both advection and diffusion. This type of equation is known to have unconditionally \emph{smooth solution} for all time and spatial location. A $1$-D generic form is given in \eqt{eq:adv_diff_sct1b}.
%
\begin{equation}\label{eq:adv_diff_sct1b}
\partial_t u(xt) + \partial_x f(x,t) = \epsilon \partial_{xx} u(x,t),
\end{equation}
% 
where $\epsilon$ is a diffusion coefficient that can be solution-dependent in theory but is assumed constant for the purpose of this section. Since the main difference between the hyperbolic problem given in \eqt{eq:ivp_sct1b} and \eqt{eq:adv_diff_sct1b} lies in the diffusion term $\epsilon \partial_xx u(x,t)$, it is proposed to investigate its effect on the numerical solution. If the solution $u(x,t)$ is smooth, the diffusion term $\epsilon \partial_{xx} u(x,t)$ in \eqt{eq:adv_diff_sct1b} is negligible and the numerical solution is driven by the advection term $\partial_x f(u(x,t))$ so that \eqt{eq:adv_diff_sct1b} and \eqt{eq:ivp_sct1b} have similar behaviors. As the solution becomes stiffer, the diffusion terms becomes large enough to influence the behavior of the numerical solution and will prevent the wave from braking as it happens in hyperbolic problems. In other terms, the diffusion term, by monitoring the change of curvature in the numerical solution, locally affects the numerical solution when needed. The diffusion coefficient $\epsilon$ can be seen as a tuning coefficient that will also affect the smoothness of the numerical solution as shown in \fig{fig:epsilon}. As $\epsilon$ goes to zero, the numerical solution becomes sharper and tends to the solution obtained when solving the hyperbolic problem given in \eqt{eq:adv_diff_sct1b}. On the opposite, with an infinite viscosity coefficient, the shock is smoothed out.
%
\begin{figure}[H]
\centering
\includegraphics[width=\textwidth]{figures/effect_of_epsilon.png}
\caption{Influence of the viscosity coefficient $\epsilon$ on the numerical solution stiffness.}
\label{fig:epsilon}
\end{figure}
%
Thus, by adding a diffusion term also called viscosity term, the numerical solution remains smooth and single valued, and should allow us to retrieve the correct physical behavior of a hyperbolic problem in the limit $\epsilon \to 0$. This approach is referred to as a \emph{vanishing viscosity method} and the numerical solution obtained with this method is denoted by $u^{\epsilon}(x,t)$. We can now define the notion of \emph{generalized solution}. 
%
\begin{definition}
\emph{
A generalized definition $u$ of a hyperbolic scalar equation conservation law 
\begin{equation}
\partial_t u + \div \mbold f(u) = 0, \nonumber
\end{equation}
is called admissible vanishing viscosity solution if there is a sequence of smooth unique solution $u^{\epsilon}$ of the parabolic equation 
\begin{equation}
\partial_t u^{\epsilon} +  \grad f(u^{\epsilon}) = \epsilon \nabla u^{\epsilon}, \nonumber
\end{equation}
that converges to $u$ as $\epsilon \to 0$.}
\end{definition}
%
It is now clear that by adding a viscosity term to an hyperbolic equation as previously explained, a generalized solution can be obtained with the vanishing viscosity approach. We will see in \sect{weak_sct1b}, that a generalized solution can be also defined by the use of a mathematical technique resulting in a weak formulation of the hyperbolic scalar equation. Before doing so, it is proposed to investigate one more property of a shock wave: its speed. Knowing the breaking time $T_{shock}$ and position is not sufficient information to track a shock wave once it is formed. An useful information will be to derive an expression that provides us with the speed of the shock. One of the reason for deriving such expression is, to obtain an analytical solution that can be used for comparison against numerical solutions in order to assess their accuracy. To do so, we consider, one again, a general $1$-D hyperbolic scalar equation for simplicity, as shown in \eqt{eq:rh_sct1b}:
%
\begin{equation}\label{eq:rh_sct1b}
\partial_t u(x,t) + \partial_x f(u(x,t)) = 0
\end{equation}
%
We assume that the position of the shock is given by a function of time denoted by $s(t)$ and that the associated speed is $S = \frac{ds}{dt}$. At this particular position, the derivatives of the solution $u$ and the flux $f(u)$ are no longer continuous. We also define a control volume $\left[ x_1; x_2 \right]$ that contains the shock wave so that $x_1 \leq s(t) \leq x_2$. \eqt{eq:rh_sct1b} is integrated over the control volume as shown in \eqt{eq:rh2_sct1b}:
%
\begin{equation}\label{eq:rh2_sct1b}
\frac{d}{dt} \int_{x_1}^{s(t)} u(x,t) dx + \frac{d}{dt} \int_{s(t)}^{x_2} u(x,t) dx + f(x_2,t) - f(x_1,t) = 0
\end{equation}
% 
The first two integrals can be recast by using the following chain rule:
%
\begin{equation}\label{eq:rh3_sct1b}
\frac{d}{dt} \int_{y_1(t)}^{y_2(t)} g(y,t) dy =  \int_{y_1(t)}^{y_2(t)} g(y,t) dy + g(y_2,t) \frac{d y_2(t)}{dt} - g(y_1,t) \frac{d y_1(t)}{dt},
\end{equation}
% 
which yields by noticing that $x_1$ and $x_2$ are fixed and thus, not function of time:
%
\begin{equation}\label{eq:rh4_sct1b}
\int_{x_1}^{s(t)} \partial_t u(x,t) dx - \int_{s(t)}^{x_2} \partial_t u(x,t) dx + \left( u(s^-(t)) - u(s^+(t)) \right) S + f(x_2,t) - f(x_1,t) = 0
\end{equation}
%
where $u(s^-(t),t)$ and $u(s^+(t),t)$ are the values of the solution $u$ before and after the shock, respectively. Assuming that the $x_1$ and $x_2$ approach the shock position $s(t)$ from the left and right, respectively, and that $\partial_t u$ is bounded, the two integrals vanish to yield the following expression for the speed of shock $S$:
%
\begin{equation}\label{eq:rh5_sct1b}
S = \frac{f(x_2,t) - f(x_1,t)}{u(s^-(t)) - u(s^+(t))} = \frac{\Delta f}{\Delta u}
\end{equation}
%
The above expression computing the speed of shock (\eqt{eq:rh5_sct1b}) is known as the Rankine-Hugoniot jump condition. The hyperbolic scalar equation given \eqt{eq:rh_sct1b} is only valid in smooth parts of the solution, and thus, require the use of the Rankine-Hugoniot jump condition in order to solve for the shock region.   
%===============================
\subsection{Weak solution and entropy condition:}\label{weak_sct1b}
%===============================
As mentioned in \sect{sec:shock_form_sct1b}, another way to define a generalized solution is to use a mathematical technique that consists of multiplying the hyperbolic scalar equation by a smooth test function that are continuously differentiable within a contact support. Then, integration per part is performed in order to transfer the derivative off the solution $u$ and onto the test function. The resulting equation involves fewer derivative on $u$ and, hence, requires less smoothness. It is assumed that the test functions are only non-zero and continuously differentiable within a bounded set: $\phi(\mbold{r},t) \in C_0^1 \left( \mathbb{R}^d \times \mathbb{R} \right)$. 

If the simple $1$-D conservation law $\partial_t u(x,t) + \partial_x f(u) = 0$ is multiplied by the a test function $\phi(x,t)$ and integrated over space and time, it yields:
%
\begin{equation}\label{eq:weak_sol_sct1b}
\int_0^{+\infty}\int_{-\infty}^{+\infty} \left[ \partial_t u(x,t) + \partial_x f(u) \right] \phi(x,t) dx dt = 0
\end{equation}
% 
Then, \eqt{eq:weak_sol_sct1b} is integrating by parts to obtain:
%
\begin{eqnarray}\label{eq:weak_sol2_sct1b}
\int_0^{+\infty}\int_{-\infty}^{+\infty} \left[ u(x,t) \partial_t \phi(x,t)   + f(u) \partial_x \phi(x,t)  \right] dx dt = \nonumber \\
-\int_{-\infty}^{+\infty} \phi(x,t) u(x,t) dx \left. \right|_0^{+\infty} - \int_0^{+\infty} \phi(x,t) u(x,t) dt \left. \right|_{-\infty}^{+\infty}
\end{eqnarray}
%
where the two integrals on the right hand-side of \eqt{eq:weak_sol2_sct1b} corresponds to the boundary terms. Recalling that the test function is identically zero outside a bounded set which means $\phi(x,t) \to 0$ as $x \to \pm  \infty$ and $t \to \infty$, all of the boundary terms vanish but for $t=0$. The resulting equation contains only one boundary term that is function of the initial condition as shown in \eqt{eq:weak_sol3_sct1b}.
%
\begin{eqnarray}\label{eq:weak_sol3_sct1b}
\int_0^{+\infty}\int_{-\infty}^{+\infty} \left[ u(x,t) \partial_t \phi(x,t)   + f(u) \partial_x \phi(x,t)  \right] dx dt = \nonumber \\
-\int_{-\infty}^{+\infty} \phi(x,0) u_0(x) dx 
\end{eqnarray}
%
\eqt{eq:weak_sol3_sct1b} is used in the definition of a weak solution as follows:
%
\begin{definition}
\emph{The function $u(x,t)$ is called a weak solution of the conservative law
%
\begin{equation}
\partial_t u(x,t) + \partial_x f(u) = 0, \nonumber
\end{equation}
%
if \eqt{eq:weak_sol3_sct1b} holds for all test functions $\phi \in C_0^1 \left( \mathbb{R} \times \mathbb{R} \right)$.}
\end{definition}
%
The formulation obtained in \eqt{eq:weak_sol3_sct1b} presents some similarities with the $1$-D conservation law when integrated over a rectangle $\left[ x_1,x_2 \right] \times \left[ t_1,t_2 \right]$:
%
\begin{eqnarray}\label{eq:weak_sol4_sct1b}
\partial_t u(x,t) + \partial_x f(u) = 0 \to&& \nonumber \\
\int_{x_1}^{x_2} \left[ x(x,t_2) - u(x,t_1) \right] dx &+& \int_{t_1}^{t_2} \left[ f(u(x_2,t)) - f(u(x_1,t)) \right]dt = 0
\end{eqnarray}
%
As a matter of fact, we can show that \eqt{eq:weak_sol3_sct1b} and \eqt{eq:weak_sol4_sct1b} are equivalent. To do so, we consider the test function $\phi(x,t)$ with the following properties:
%
\begin{equation}\label{eq:weak_sol5_sct1b}
\phi(x,t) = \left\{
\begin{array}{l}
1  \text{ for }  (x,t) \in \left[ x_1,x_2 \right] \times \left[ t_1,t_2 \right]  \\
0 \text{ for } (x,t) \notin \left[ x_1-\delta,x_2+\delta \right] \times \left[ t_1-\delta,t_2+\delta \right]
\end{array}\right.
\end{equation}
%
where $\delta$ is the width of the intermediate strip. Since the test function $\phi$ verifying \eqt{eq:weak_sol5_sct1b} vanish as $x \to \pm  \infty$ and $t \to \infty$, \eqt{eq:weak_sol3_sct1b} is still valid. The temporal and spatial derivatives of the test function $\phi(x,t)$ are zero everywhere but in the intermediate strip, and approach delta function as $\delta \to 0$. Thus, in the limit $\delta \to 0$, \eqt{eq:weak_sol3_sct1b} is shown to be equivalent to the integral form of the conservation law given in \eqt{eq:weak_sol4_sct1b}. The direct consequence of this result is that the weak solution in the sense of \eqt{eq:weak_sol3_sct1b} includes the solution of the hyperbolic conservation law we are looking for, and the vanishing viscosity generalized solution. However, uniqueness of the weak solution is not ensured by \eqt{eq:weak_sol3_sct1b} which can be demonstrated by considering a $1$-D Riemann problem for a hyperbolic conservation law. The reader can refer to \cite{Toro} and LEVEQUE for details. The question arising from the previous statement is: how to identify the correct weak solution that corresponds to the physically vanishing viscosity solution? The answer lies in the entropy condition that is defined by analogy to the gas dynamic case. Mathematically, the entropy condition can be defined of multiple ways using either the speed of shock $S$ (see page 37 in LEVEQUE) or an entropy function that will be further detailed. Because our focus is to hyperbolic system of equations, it is chosen to define the entropy condition through the derivation of an entropy residual from a hyperbolic conservation law. 

Before detailing the steps to obtain an entropy residual, we recall what the entropy condition states for gas dynamic. In gas dynamic, the physical entropy is constant along a smooth path flow, but experiences a jump to a higher value across a shock wave. Thus, the physical entropy is expected to increase if the flow has a shock wave, which gives us a condition in order to pick out the correct weak solution. It remains to translate this condition into a mathematical relation. Before doing so, we want to emphasize the difference between the mathematical and physical entropies in order to clear any confusion (hopefully it is not too late). The physical entropy is by definition positive and increases as a function of time to reach a maximum value at steady-state. The mathematical entropy is defined convex and decreases as a function of time. In order to avoid any confusion, the mathematical and physical entropies will be referred to as $\eta$ and $s$, respectively. These two entropies are related to each other by the following relation:
%
\begin{equation}
s(u) = -\eta (u) + \eta_0
\end{equation}
%  
where $\eta_0$ is taken larger than the maximum of the absolute value of the mathematical entropy $\eta$ in order to ensure positivity of the physical entropy $s$. Of course, with such relation, convexity of $\eta$ implies that $s$ is concave. It is custom to work with the mathematical entropy when dealing with hyperbolic scalar equations, whereas the physical entropy is preferred for hyperbolic system of equation alike the multi-D Euler equations. 

It is now proposed to derive an entropy condition for a hyperbolic scalar conservation law of the generic form:
%
\begin{equation}\label{eq:weak_sol6bis_sct1b}
\partial_t u( \mbold{r}, t) + \div \mbold f(u) = 0
\end{equation}
%
We assume the existence of an entropy function $\eta(u)$ for which a conservation law holds and will be determined. We first consider the case of a smooth flow and multiply \eqt{eq:weak_sol6bis_sct1b} by the derivative of $\eta$ with respect to $u$ denoted by $\eta_u$:
%
\begin{subequations}\label{eq:weak_sol6_sct1b}
\begin{equation}\label{eq:weak_sol6a_sct1b}
\eta_u \partial_t u( \mbold{r}, t) + \eta_u  \div \mbold f(u) = 0,
\end{equation}
%
and using the chain rule,
%
\begin{equation}\label{eq:weak_sol6b_sct1b}
\eta_u \partial_t u( \mbold{r}, t) + \eta_u  \mbold f'(u) \cdot \grad u = 0,
\end{equation}
%
which can be simplified to
%
\begin{equation}\label{eq:weak_sol6c_sct1b}
\partial_t \eta( u) +  \mbold f'(u) \cdot \grad \eta(u) = 0.
\end{equation}
\end{subequations}
%
The relation given in \eqt{eq:weak_sol6c_sct1b} is a conservation law for $\eta$ which means that the mathematic entropy is conserved in a smooth flow. The entropy conservation law can also be recast as a function of the entropy flux $\mbold \psi$ by setting $\mbold \psi '(u) = \eta_u \mbold f'(u)$, which yields:
%
\begin{equation}
\partial_t \eta( u) +  \div \psi(u) = 0.
\end{equation}
%
The set $\left( \eta, \mbold \psi \right)$ is called an entropy pair.

We now consider a solution with a discontinuity/shock. The manipulations performed above are no longer valid. Instead, it is proposed to look at the vanishing viscosity equation introduced in \eqt{eq:adv_diff_sct1b} and investigate the behavior of the vanishing viscosity weak solution as the viscosity coefficient tends to zero. Applying the vanishing viscosity method to \eqt{eq:weak_sol6_sct1b}, it yields:
%
\begin{equation}\label{eq:weak_sol7_sct1b}
\partial_t u( \mbold{r}, t) + \div \mbold f(u) = \div \left( \mu(\mbold r,t ) \grad u( \mbold r,t) \right),
\end{equation}
%
where $\mu(\mbold r,t )$ is a positive viscosity coefficient assumed spatial-dependent for general purpose. Because of the viscous term in \eqt{eq:weak_sol7_sct1b}, the solution remains smooth which allows us to perform the same manipulations as in \eqt{eq:weak_sol6_sct1b}. Thus, it yields:
%
\begin{equation}\label{eq:weak_sol8_sct1b}
\partial_t \eta (u) + \div \mbold \psi (u) = \eta_u \div \left( \mu(\mbold r,t ) \grad u( \mbold r,t) \right),
\end{equation}
%
By integrating per parts the right hand-side of \eqt{eq:weak_sol8_sct1b} becomes: 
%
\begin{equation}\label{eq:weak_sol9_sct1b}
\partial_t \eta (u) + \div \mbold \psi (u) =  \div \left( \eta_u \mu(\mbold r,t ) \grad u( \mbold r,t) \right) - \mu(\mbold r,t ) \grad u( \mbold r,t) \cdot \grad \eta_u
\end{equation}
%
Since we are interested in the weal solution, \eqt{eq:weak_sol9_sct1b} is integrated over the spatial-temporal domain $\Theta = \left[ \mbold r_1,\mbold r_2 \right] \times \left[ t_1,t_2 \right]$ on the same model as in \eqt{eq:weak_sol4_sct1b}. It is also assumed that the shock remains inside $\Theta$ for all time $t \in \left[ t_1, t_2 \right]$ and away from the boundaries.
%
\begin{eqnarray}\label{eq:weak_sol10_sct1b}
\int_{\mbold r_1}^{\mbold r_2} \int_{t_1}^{t_2} \left[ \partial_t \eta (u) + \div \mbold \psi (u) \right] \text{d}t \text{d} \mbold r  &=&
\int_{\mbold r_1}^{\mbold r_2} \int_{t_1}^{t_2}  \div \left( \eta_u \mu(\mbold r,t ) \grad u( \mbold r,t) \right) \text{d}t \text{d} \mbold r   \nonumber \\
&-& \int_{\mbold r_1}^{\mbold r_2} \int_{t_1}^{t_2}  \mu(\mbold r,t ) \grad u( \mbold r,t) \cdot \grad \eta_u \text{d}t \text{d} \mbold r 
\end{eqnarray}
%
The first term in the right hand-side of \eqt{eq:weak_sol10_sct1b} can be recast as follows:
%
\begin{multline}\label{eq:weak_sol11_sct1b}
\int_{\mbold r_1}^{\mbold r_2} \int_{t_1}^{t_2}  \div \left( \eta_u \mu(\mbold r,t ) \grad u( \mbold r,t) \right) \text{d}t \text{d} \mbold r = \\
\int_{t_1}^{t_2} \left[ \eta_u \mu(\mbold r_2,t ) \grad u( \mbold r_2,t) - \eta_u \mu(\mbold r_1,t ) \grad u( \mbold r_1,t) \right] \text{d}t
\end{multline}
%
Since the shock position cannot be confounded with the boundary of the domain by assumption, the gradient of the solution at the points $\mbold r_1$ and $\mbold r_2$ is smooth. Then, as the viscosity coefficient tends to zero, the whole integral vanishes since all of the terms are bounded. The second term in the right hand-side of \eqt{eq:weak_sol10_sct1b} is tricker to deal with. First, the integral is recast as follows:
%
\begin{equation}\label{eq:weak_sol12_sct1b}
\int_{\mbold r_1}^{\mbold r_2} \int_{t_1}^{t_2} \mu(\mbold r,t ) \grad u( \mbold r,t) \cdot \grad \eta_u = \int_{\mbold r_1}^{\mbold r_2} \int_{t_1}^{t_2} \mu(\mbold r,t ) \eta_{uu} \grad u( \mbold r,t) \cdot \grad u( \mbold r,t),
\end{equation}
%
where $\eta_{uu}$ denotes the second derivative of $\eta$ with respect to $u$. As $\mu$ goes to zero, the integral does not vanish because of the terms $\grad u \cdot \grad u$ that will become larger and larger at the location of the discontinuity. However, by assuming that the entropy function is convex, i.e. $\eta_{uu} \geq 0$ and noticing that $\grad u \cdot \grad u \geq 0$ and $\mu \geq 0$, the sign of \eqt{eq:weak_sol12_sct1b} is found to be positive. Using the above results, it is concluded that the vanishing viscosity weak solution satisfies the inequality:
%
\begin{equation}\label{eq:weak_sol13_sct1b}
\int_{\mbold r_1}^{\mbold r_2} \int_{t_1}^{t_2} \left[ \partial_t \eta (u) + \div \mbold \psi (u) \right] \text{d}t \text{d} \mbold r  \leq 0,
\end{equation}
% 
for smooth and discontinuous solutions. Consequently, the total integral of the mathematical entropy decreases as a function of time. Since the inequality given in \eqt{eq:weak_sol13_sct1b} holds for any $\Theta$, it can be simplified to
%
\begin{equation}\label{eq:weak_sol14_sct1b}
\partial_t \eta (u) + \div \mbold \psi (u) \leq 0
\end{equation}
%
in the weak sense and is known as the \emph{entropy inequality}. 
%
\begin{definition}
\emph{The function $u(\mbold r, t)$ is the entropy solution of the hyperbolic conservation law 
%
\begin{equation}
\partial_t u( \mbold{r}, t) + \div \mbold f(u) = 0 \nonumber
\end{equation}
%
if for all convex entropy pair $\left( \eta,\mbold \psi \right)$, the inequality 
%
\begin{equation}
\partial_t \eta (u) + \div \mbold \psi (u) \leq 0 \nonumber
\end{equation}
%
is satisfied in a weak sense.}
\end{definition}
%
The entropy inequality can be used to analyze numerical methods in order to demonstrate that the numerical solution converges to the entropy solution. It is now proposed to see how the entropy inequality is used in the theoretical derivation of the entropy viscosity method for the multi-D Burger's equation. 
%===============================
\subsection{Entropy viscosity method for the multi-D Burger equation:}
%===============================
This section aims at making the link between the theoretical results presented from \sect{sec:mat_ppr_sct1b} through \sect{weak_sct1b}, and the entropy viscosity method that is the focus of this dissertation. For academic purpose, we limit our scope to the multi-D Burger's equation and rely on the work by Guermond et al. \cite{jlg1, jlg2, jlg3, valentin}. However, the reader must keep in mind that the objective of this dissertation is to apply the entropy-viscosity method to hyperbolic system of equation.

In the previous sections, it was demonstrated the importance of ensuring uniqueness of the weak solution: it is achieved (i) by adding a viscosity term to the hyperbolic scalar equation in order to prevent tripled-value point from forming, and (ii) by using an entropy condition. When applied to the viscid multi-D Burger's equation, it yields:
%
\begin{subequations}\label{eq:bg1_ev_sct1b}
%
\begin{equation}\label{eq:bg2_ev_sct1b}
\partial_t u(\mbold{r},t) + \div \left( \frac{u(\mbold{r},t)^2}{2} \vec{n} \right) = \div \left( \mu(\mbold{r},t) \grad u(\mbold{r},t) \right)
\end{equation}
%
\begin{equation}\label{eq:bg3_ev_sct1b}
R(\mbold{r},t) = \partial_t \eta(\mbold{r},t) + \div \mbold \psi \leq 0
\end{equation}
%
\end{subequations}
%
where $\mu(\mbold{r},t)$ is a spatial-dependent viscosity coefficient and $R(\mbold{r},t)$ denotes the entropy residual. Assuming that $\mu$ is constant, \eqt{eq:adv_diff_sct1b} is retrieved. All of the other variables in \eqt{eq:bg1_ev_sct1b} were defined previously. It was shown in \sect{weak_sct1b} that the sign of the entropy residual, \eqt{eq:bg3_ev_sct1b}, is conditioned to the convexity of the entropy function $\eta$, to the positivity of the viscosity coefficient $\mu(\mbold{r},t)$, and also to the form of the viscous term $\div \left( \mu(\mbold{r},t) \grad u(\mbold{r},t) \right)$. In other word, the entropy condition could be used to derive the proper viscous term that will ensure the correct sign for the entropy residual in the shock region. In the case of the multi-D Burger's equation, the choice of the viscous term is obvious and probably unique. However, when considering hyperbolic system of equation alike the multi-D Euler equations, deriving the viscous terms consistent with the entropy condition is no longer trivial and can lead to long and fastidious derivation. This aspect of the method is detailed in \sect{sec:hyp_sect1b}. 

Once the viscous term is derived and known to be consistent with the entropy condition, it remains to define the viscosity coefficient $\mu(\mbold{r},t)$ that is known to be positive. Such a step is crucial and should not be underestimated since it will determine the accuracy of the numerical method. The easiest definition we can think of, is to set $\mu$ to a constant value. By doing so, dissipation will be added to the shock region, preventing the wave from braking, and also to the smooth region of the solution that does not need dissipation. Such a behavior is not ideal and will not accurately resolve the shock. Another option would be to to track the shock position in order to only add a significant amount of dissipation in the shock region. We also require the viscosity coefficient to be smooth based on the work from REF that a discontinuous viscosity coefficient could yield instabilities in the numerical solution. Such a behavior can be easily understood by considering the following example. Let us assume that the viscosity coefficient jumps from zero to a large value as closing to the shock region. Because the dissipative term is conservative, the gradient of the solution, $\grad u$, will have to experience the same discontinuity as the viscosity coefficient, thus, yielding the same type of behavior in the solution itself.   
Defining a smart viscosity capable of detecting and tracking a shock is not a straightforward task and needs to rely on a good understanding of the theory related to the shock formation. For example, we can think of monitoring the gradient of the solution itself that will become large in the shock region. Following this reasoning, a possible definition in $1$-D would be $\mu(x,t) \propto \left| \partial_x u(x,t) \right|$. Another approach consists of using the entropy residual $R_e$ defined in \sect{weak_sct1b}. The entropy residual was initially studied to ensure uniqueness of the weak solution, but its variations are intimately related to the solution: $R_e$ is small as the solution is smooth and become large (in absolute value) in the shock region. Thus, by monitoring the variation of the entropy residual, the shock can be detected and also tracked. This approach was used by Guermond et al. \cite{jlg1, jlg2, jlg3} to solve for hyperbolic scalar equations such as the multi-D Burger's equation and named after the Entropy Viscosity Method (EVM). Their method requires the definition of three viscosity coefficients: a high-order viscosity coefficient, $\mu_e(\mbold{r},t)$, defined proportional to the absolute value of the entropy residual $R_e$,  a first-order viscosity coefficient denoted by $\mu_{max}(\mbold{r},t)$ set proportional to the local maximum eigenvalue, and the viscosity coefficient $\mu(\mbold{r},t)$ such as $\mu(\mbold{r},t) = \min \left( \mu_{max}(\mbold{r},t), \mu_e(\mbold{r},t) \right)$, used in the dissipative term $\div \left( \mu(\mbold{r},t) \grad u(\mbold{r},t) \right)$. The idea is to detect the entropy production characteristic of a shock wave. By defining $\mu_e(\mbold{r},t)$ proportional to $\left| R_e \right|$, the high-order viscosity will be large in the shock region and small anywhere else. The first-order viscosity serves as an upper bound for $\mu(\mbold{r},t)$ and its definition lies onto two criteria:
\begin{enumerate}
\item the definition of $\mu_{max}(\mbold{r},t)$ is determined so that the viscous regularization in \eqt{eq:bg2_ev_sct1b} is equivalent to the upwind-scheme when assuming $\mu(\mbold{r},t) = \mu_{max}(\mbold{r},t)$, which yields: $\mu_{max}(\mbold{r},t) = \frac{h}{2} f'(u(\mbold{r},t))$. Derivation of the expression for $\mu_{max}$ can be found in REF and is easily done in $1$-D when discretizing Burger's equation with a finite difference method. 
\item the first-order viscosity coefficient is related to the Curant-Friedrichs-Lewy number (CFL) and more precisely to the stability of the numerical solution when using temporal explicitly schemes.
\end{enumerate}
Proof of a maximum principle for the EVM is under investigation and thus, will affect the definition of the first-order viscosity coefficient $\mu_{max}$. Based on the definition of the high- and first-order viscosity coefficients, the variation of the $\mu(\mbold{r},t)$ are the followings: as the solution is smooth, the entropy production measured by the entropy residual, $R$, is small and thus $\mu(\mbold{r},t) = \mu_e(\mbold{r},t)$. In the shock region, the entropy residual is peaked and the high-order viscosity coefficient saturates to the first-order viscosity coefficient that is known to be over-dissipative since equivalent to upwind-scheme. With such definition, the viscosity coefficient $\mu(\mbold{r},t)$ is peaked in the shock region and small anywhere else, while experiencing smooth variations. 

It remains now to complete the definition of the high-order viscosity coefficient $\mu_e(\mbold{r},t)$ which can be achieved by performing a dimensional analysis. Noticing that the units of $\mu(\mbold{r},t)$ are $m^2 \cdot s^{-1}$, a suitable definition for $\mu_e(\mbold{r},t)$ is:
%
\begin{equation}
\mu_e(\mbold{r},t) = h^2 \frac{\left| R(\mbold{r},t) \right|}{norm(s)} \nonumber
\end{equation}
%
where $norm(s)$ is a normalization function of the same unit as the entropy $s$, and $h$ is a characteristic length of the mesh. At this stage, the choice of the normalization function is unclear. Guermont et al. proposed to use $norm(s) = || s - \bar{s} ||_\infty$ where $\bar{s}$ is the average value of the entropy function over the computational domain, and $|| \cdot ||_\infty$ denotes the infinity norm. Their definition of the EVM when applied to hyperbolic scalar equation is the following:
%
\begin{subequations}
%
\begin{equation}
\partial_t u(\mbold{r},t) + \div \left( f(u(\mbold{r},t)) \vec{n} \right) = \div \left( \mu(\mbold{r},t) \grad u(\mbold{r},t) \right)
\end{equation}
%
\begin{equation}
R(\mbold{r},t) = \partial_t \eta(\mbold{r},t) + \div \mbold \psi
\end{equation}
%
\begin{equation}\label{eq:evm_def_sct1b}
\left\{
\begin{array}{l}
\mu(\mbold{r},t) = \min \left( \mu_e(\mbold{r},t), \mu_{max}(\mbold{r},t) \right) \\
\mu_{max}(\mbold{r},t) = \frac{h}{2} \left| f'(u(\mbold{r},t)) \right| \\
\mu_e(\mbold{r},t) = h^2 \frac{\max \left(\left| R(\mbold{r},t), J \right) \right|}{|| s - \bar{s} ||_\infty}
\end{array}
\right.
\end{equation}
%
\end{subequations}
%
where $h$ is defined as the gird size and $J$ denotes the jump of the entropy flux $\mbold \psi$ at the interface of each element. Information relative to the computation of the jump with continuous Galerkin finite element method (CGFEM) will be detailed in \sect{sec:spatial_discretization}. For the case of discontinuous schemes, the reader can refer to \cite{valentin}. Numerical results for the multi-D Burger's equation solved with the EVM  with CGFEM are presented in \sect{sec:burger_sct2b}.
%
\begin{remark}
The definition of the viscosity coefficients given in \eqt{eq:evm_def_sct1b} requires a isentropic mesh in order to be able to define the gird size $h$. An alternative definition without $h$ is under investigation for the case of hyperbolic scalar equations.
\end{remark}
%%%%%%%%%%%%%%%%%%%%%
\section{Hyperbolic system of equations:}\label{sec:hyp_sect1b}
%%%%%%%%%%%%%%%%%%%%%
In this section, application of the EVM to non-linear hyperbolic system of equations is discussed. The reader can refer to Toro \cite{Toro} and Leveque for an extension to the non-linear hyperbolic conservation laws, of the theoretical notions (eigenvalues, characteristic curves, $\dots$) introduced in \sect{sec:hyp_sect1b}. The objective of this section, is to provide the reader with a methodology on how to apply the EVM to any hyperbolic system of equations. For academic purpose, we will rely on the latest published version of the EVM \cite{valentin} for the multi-D Euler equations in order to understand the main steps of the method. Other more recent publications will be also used. \cite{jlg, jlg2} are good pre-requites to \cite{valentin} to see the theoretical evolution of the EVM. For hyperbolic system of equation, it is custom of using the physical solution $s$ that is of opposite sign as the mathematical entropy denoted by $\eta$.\\

To the best of our knowledge, the EVM was successfully applied to one hyperbolic system of equation: the multi-D Euler equations with the Ideal gas equation of state \cite{jlg, valentin}. Good agreements with the exact solutions were obtained in $1$- and $2$-D results when using discontinuous schemes (finite volume and discontinuous Galerkin finite element methods). We now recall the details of the latest version of the method \cite{valentin} and remind the reader that this version will be used as a starting point and modified during this dissertation. The viscous regularization derived from the entropy condition for the multi-D Euler equations with the Ideal Gas equation of state, is the following:
%
\begin{subequations}\label{eq:euler_visc_sct1b}
%
\begin{equation}
\partial_t \rho + \div \left( \rho \vec{u} \right) = \div \left( \mu \grad \rho \right) 
\end{equation}
%
\begin{equation}
\partial_t \left( \rho \vec{u} \right) + \grad \left( \rho \vec{u} \otimes \vec{u} \right) + \grad P = \div \left( \rho \mu \grad^s \vec{u} \right)
\end{equation}
%
\begin{equation}
\partial_t \left( \rho E \right) + \div \left[ \vec{u} \left( \rho E +P \right) \right] = \div \left( \kappa \vec{u} \grad^s \vec{u} + \kappa \grad T \right)
\end{equation}
%
\begin{equation}\label{eq:eos_sct1b}
P = \left( \gamma-1 \right) \rho e = \left( \gamma-1 \right) C_v \rho T
\end{equation}
%
\end{subequations}
%
where $\rho$, $\rho \vec{u}$ and $\rho E $ are the fluid density, momentum and total energy, respectively, and will be referred to as the conservative variables. The pressure $P$ and the temperature $T$ are computed from the Ideal gas equation of state (IGEOS) recalled in \eqt{eq:eos_sct1b} which is a function of the density $\rho$ and the specific internal energy $e$. The heat capacity $C_v$ is constant by definition. The viscosity coefficients $\mu$ and $\kappa$ are spatial and temporal dependent and are defined proportional to an entropy residual $R$ as follows:
% 
\begin{subequations}
%
\begin{equation}
\left\{
\begin{array}{l}
\mu\left( \mbold{r},t \right) = \min \left( \mu_e\left( \mbold{r},t \right), \mu_{max}\left( \mbold{r},t \right) \right) \\
\kappa\left( \mbold{r},t \right) = \frac{\gamma \Pr}{\gamma-1} \mu\left( \mbold{r},t \right),
\end{array}
\right.
\end{equation}
and
\begin{eqnarray}\label{eq:visc_def_sct1b}
&&\mu_{max}\left( \mbold{r},t \right) = \frac{h}{2} \left( ||\vec{u}\left( \mbold{r},t \right)|| + c\left( \mbold{r},t \right)  \right) \nonumber \\
&&\mu_e\left( \mbold{r},t \right) = C_E h^2 \frac{\max \left( \left|| R\left( \mbold{r},t \right) \right|, J \right)}{|| s\left( \mbold{r},t \right) - \bar{s}(t) ||_\infty} \\
&&R\left( \mbold{r},t \right) = \partial_t s\left( \mbold{r},t \right)  + \vec{u} \cdot \grad s\left( \mbold{r},t \right), \nonumber
\end{eqnarray}
%
\end{subequations}
%
where $C_E$ is a constant coefficient of order one, $h$ is the grid size, $c = \sqrt{\gamma P / \rho}$ is the speed of sound and $\Pr$ is a Prandtl number taken in the interval $\in \left[ 0; \frac{1}{4} \right]$. The variable $J$ denotes the jump of the entropy flux $\vec{u} s$ and is pice-wise constant. It is computed at the interfaces between each cell and its direct neighbors. Further details can be found in \cite{valentin}. The entropy $s$ is function of the density and the pressure:
%
\begin{equation}
s\left( \rho, P \right) = C_v \ln \left( \frac{P}{\rho^\gamma} \right),
\end{equation}
%
and can also be recast as a function of the density and the internal energy using the IGEOS. It remains to define the symmetric gradient $\grad^s \vec{u}$ with the following entries $\grad^s \vec{u}_{i,j} = \frac{1}{2} \left( \frac{\partial u_i}{\partial x_j}+\frac{\partial u_j}{\partial x_i} \right)$. The current definition of the EVM suffers from some theoretical gap. The normalization parameters $|| s\left( \mbold{r},t \right) - \bar{s}(t) ||_\infty$ used in the definition of the high-order viscosity coefficient $\mu_e$ in \eqt{eq:visc_def_sct1b} does not currently have any theoretical justification. The same remark can be made for the Prandtl number that is set by the user based on its experience. Plus, the viscous regularization given in \eqt{eq:euler_visc_sct1b} is only valid for the IGEOS which is limiting for engineering applications. However, new developments in the theory extended the validity of the method to any equation of state \cite{jlg} which makes it a good candidate for nuclear reactor applications for instance. Thus, based on the work done in \cite{jlg} and with the experience of \cite{jlg2, valentin} the following methodology is proposed. We consider the generic non-linear hyperbolic system in order to detail our methodology:
%
\begin{equation}\label{eq:hyp_syst_sct1b}
\partial_t \boldsymbol{U}(\mbold{r},t) + \div \boldsymbol{F} \left( \boldsymbol{U} \right) = 0,
\end{equation}
%
where $\boldsymbol{U} = \left( U_1, \dots, U_n \right)$ is a vector solution and $\boldsymbol{F} \left( \boldsymbol{U} \right)$ is a hyperbolic flux whom eigenvalues are denoted by $\left( \lambda_1, \dots, \lambda_n \right)$.
\begin{enumerate}
\item The first step consists of deriving an entropy equation for an entropy function denoted by $s\left(\boldsymbol{U}\right)$ of the form: 
%
\begin{equation}
R\left(\boldsymbol{U}\right) = \partial_t s\left(\boldsymbol{U}\right) + \boldsymbol{F}_{\boldsymbol{U}} \cdot \grad s \left(\boldsymbol{U}\right) = 0,
\end{equation}
%
where $\boldsymbol{F}_{\boldsymbol{U}} = \frac{\partial \boldsymbol{F}}{\partial \boldsymbol{U}}$ is the jacobian matrix of the hyperbolic flux $\boldsymbol{F} \left( \boldsymbol{U} \right)$.
This entropy equation/residual is obtained from the hyperbolic system of equation given in \eqt{eq:hyp_syst_sct1b} either by multiplying by the matrix $s_{\boldsymbol{U}} = \frac{\partial s (\boldsymbol{U})}{\partial \boldsymbol{U}}$ and using the chain rule, or by doing combination of the equations of the hyperbolic system. This step is well documented for the multi-D Euler equation \cite{Toro}. The objective is to understand the steps that lead to the derivation of the entropy equations since the same steps will be used to obtain the viscous terms.
\item We now want to derive the dissipative terms consistent with the entropy condition. The method is inspired of what is done for the multi-D Euler equations in \cite{jlg}. To do so, we first modify \eqt{eq:hyp_syst_sct1b} by adding a viscous flux $\boldsymbol{G} \left( \boldsymbol{U} \right)$:
%
\begin{equation}\label{eq:hyp_syst2_sct1b}
\partial_t \boldsymbol{U}(\mbold{r},t) + \div \boldsymbol{F} \left( \boldsymbol{U} \right) = \div \boldsymbol{G} \left( \boldsymbol{U} \right),
\end{equation}
%
Then, the entropy residual is derived again:
%
\begin{eqnarray}\label{eq:hyp_syst3_sct1b}
R\left(\boldsymbol{U}\right) = \partial_t s\left(\boldsymbol{U}\right) + \boldsymbol{F}_{\boldsymbol{U}} \cdot \grad s \left(\boldsymbol{U}\right) &=& s_{\boldsymbol{U}} \div \boldsymbol{G} \left( \boldsymbol{U} \right) \nonumber \\
R\left(\boldsymbol{U}\right) = \partial_t s\left(\boldsymbol{U}\right) + \boldsymbol{F}_{\boldsymbol{U}} \cdot \grad s \left(\boldsymbol{U}\right) &=& \nonumber \\
\div ( s_{\boldsymbol{U}} \boldsymbol{G}  \left( \boldsymbol{U} \right) ) &-& \boldsymbol{G}  \left( \boldsymbol{U} \right) \cdot \grad s_{\boldsymbol{U}}.
\end{eqnarray}
%
To prove that the entropy residual $R$ remains positive, the non-conservative terms of the right hand-side has to be positive. Thus, positivity of the entropy residual is tied to the definition of the viscous term $\boldsymbol{G}  \left( \boldsymbol{U} \right)$ and the entropy function $s$. For the multi-D Euler equation, Guermond et al. \cite{jlg} proved that the entropy function $s$ needs to be concave ($-s$ is convex) in order to ensure positivity of the entropy residual for any equation of state. This condition is tied to a particular choice of the dissipative terms that will be detailed in \sect{sec:entro_visc}, and also positivity of the viscosity coefficients. In the general case, the parabolic regularization (REF) can be used and consists of dissipating on the vector solution $\boldsymbol{U}$ as shown in \eqt{eq:hyp_syst4_sct1b}: 
%
\begin{equation}\label{eq:hyp_syst4_sct1b}
\partial_t \boldsymbol{U}(\mbold{r},t) + \div \boldsymbol{F} \left( \boldsymbol{U} \right) = \div \left( \mu \grad \left( \boldsymbol{U} \right) \right),
\end{equation}
%
where $\mu$ is a positive viscosity coefficient. \eqt{eq:hyp_syst4_sct1b} obeys to the entropy condition under the condition of having a concave entropy $s$ \cite{Parabolic}. Using the entropy condition, other viscous regularizations can be found with multiple viscosity coefficients (see \cite{jlg} for the case of the multi-D Euler equation). However, it is expected that they all degenerate to the parabolic regularization when assuming that all viscosity coefficients are equal to each other. Plus, for consistency with the parabolic regularization \cite{Parabolic}, the entropy function $s$ is required to be concave. In other terms, the parabolic regularization can be used as a hint in order to derive a viscous regularization with multiple viscosity coefficients. We will see in \sect{sec:lowMach} that having a viscous regularization with two viscosity coefficients is required for the multi-D Euler equation in order to have well-scaled viscous terms in the non-isentropic low Mach limit. Before moving forward to the next step it is proposed to recast the entropy residual $R$ as a function of the conservative variables. This step is justified by the difficulty encountered to obtain an analytical expression of the entropy function. This is particular true for the multi-D Euler equation when dealing with complex equation of states. We assume that the entropy residual was successfully recast as a function of the conservative variables and that the new entropy residual is denoted by $\tilde{R}$. The reader can refer to \sect{sec:new_ent_prod} as an example.
\item Once the entropy residual is proven to be positive, it remains to define the viscosity coefficient(s). Assuming that a viscous regularization was derived in step 2 with $n$ viscosity coefficients denoted by $\mu_i$, $i\in \left[1, \dots, n\right]$, a general definition can be given under the form:
%
\begin{equation}
\left\{
\begin{array}{l}
\mu_i(\mbold{r},t) = \max \left( \mu_{i,e}(\mbold{r},t), \mu_{max}(\mbold{r},t) \right) \nonumber \\
\mu_{max}(\mbold{r},t) = \frac{h}{2} \max_{i\in \left[1,n\right]} |\lambda_i(\mbold{r},t)| \nonumber \\
\mu_{i,e}(\mbold{r},t) = h^2 \frac{\max\left(\tilde{R}(\mbold{r},t), J\right)}{norm_i(\mbold{r},t)}, \nonumber
\end{array}
\right.
\end{equation}
%
where $\mu_{e,i}$ and $\mu_{max}$ are the high- and first-order viscosity coefficients, respectively. The high-order viscosity coefficient $\mu_{e,i}$ is defined proportional to the local entropy residual $R(\mbold{r},t)$ and also function of a normalization parameter $norm_i(\mbold{r},t)$ that will be further detailed. The first-order viscosity coefficient $\mu_{max}(\mbold{r},t)$ is set proportional to the maximum eigenvalue and is unique for all viscosity coefficients $\mu_i(\mbold{r},t)$. $h$ still denoted the grid size. In order to have a complete definition for the $\mu_e(\mbold{r},t)$, the normalization parameter $norm_i(\mbold{r},t)$ needs to be defined. It is well known that hyperbolic system of equations suffer from ill-scaled dissipative term in some particular asymptotic limit. This is particular true for the stabilization methods used for the multi-D Euler equations, that require a fix in the low Mach asymptotic limit in order to yield the correct asymptotic behavior \cite{LowMach1, LowMach2}. Thus, by non-dimensionalizing the equations, the normalization parameter $\norm_i(\mbold{r},t)$ can be determined for each viscosity coefficients $\mu_i(\mbold{r},t)$ to ensure well-scaled dissipative terms. 
\end{enumerate}
This three steps process is applied to the multi-D Euler equations with variable area \chap{chap:euler}, the seven-equation model \chap{chap:seven} and the radiation-hydrodynamic equations \chap{chap:hydro} to determine the dissipative terms and to define the viscosity coefficients. Details about the implementation of the EVM with continuous Galerkin finite element method are provided in \sect{sec:spatial_discretization}. The jump $J$ is given on a case by case basis since its definition depends on the variables involved in the expression of the new entropy residual $\tilde{R}$.
%\item hyperbolic system of equations can also produce shocks.
%\item same treatment as scalar equations.
%\item Lapidus, pressure-based ...
%\item low Mach
%\item 
%\end{itemize}
%%%%%%%%%%%%%%%%%%%%%