%%%%%%%%%%%%%%%%%%%%%%%%%%%%%%%%%%%%%%%%%%%%%%%%%%%
%
%  New template code for TAMU Theses and Dissertations starting Fall 2012.  
%  For more info about this template or the 
%  TAMU LaTeX User's Group, see http://www.howdy.me/.
%
%  Author: Wendy Lynn Turner 
%	 Version 1.0 
%  Last updated 8/5/2012
%
%%%%%%%%%%%%%%%%%%%%%%%%%%%%%%%%%%%%%%%%%%%%%%%%%%%

%%%%%%%%%%%%%%%%%%%%%%%%%%%%%%%%%%%%%%%%%%%%%%%%%%%%%%%%%%%%%%%%%%%%%%
%%                           SECTION I
%%%%%%%%%%%%%%%%%%%%%%%%%%%%%%%%%%%%%%%%%%%%%%%%%%%%%%%%%%%%%%%%%%%%%

\pagestyle{plain} % No headers, just page numbers
\pagenumbering{arabic} % Arabic numerals
\setcounter{page}{1}

\chapter{\uppercase {Hyperbolic scalar and system of equations:}}
%
\section{Hyperbolic scalar equations:}
%
\subsection{Formation of shocks:}
Use Burger's equations as an example.
\begin{itemize}
\item hyperbolic scalar equations are known to produce shocks even with smooth initial conditions.
\item explain how shocks can form with characteristic equations.
\item talk about diffusion equation that does not produce shocks -> same equation but with diffusion term that smoothes out shocks. Solution is unique and smooth.
\item idea is to add diffusion to the hyperbolic equation in order to control the shock.
\end{itemize}
%
\subsection{Weak solution:}
introduce the notion of weak solution and non-uniqueness of the weak solution.
%
\subsection{Entropy minimum principle:}
Condition for uniqueness of the solution. Show an example of the derivation for Burger's equations.
%
\section{Hyperbolic system of equations:}
%
\begin{itemize}
\item hyperbolic system of equations can also produce shocks.
\item same treatment as scalar equations.
\end{itemize}