%%%%%%%%%%%%%%%%%%%%%%%%%%%%%%%%%%%%%%%%%%%%%%%%%%%
%
%  New template code for TAMU Theses and Dissertations starting Fall 2012.  
%  For more info about this template or the 
%  TAMU LaTeX User's Group, see http://www.howdy.me/.
%
%  Author: Wendy Lynn Turner 
%	 Version 1.0 
%  Last updated 8/5/2012
%
%%%%%%%%%%%%%%%%%%%%%%%%%%%%%%%%%%%%%%%%%%%%%%%%%%%

%%%%%%%%%%%%%%%%%%%%%%%%%%%%%%%%%%%%%%%%%%%%%%%%%%%%%%%%%%%%%%%%%%%%%%
%%                           APPENDIX D
%%%%%%%%%%%%%%%%%%%%%%%%%%%%%%%%%%%%%%%%%%%%%%%%%%%%%%%%%%%%%%%%%%%%%

\chapter{\uppercase{Entropy equation for the multi-D seven equation model with viscous regularization}\label{app:sev-equ-model-entropy}}
This appendix aims at detailing the steps that lead to the derivation of the phase wise entropy equation of the seven equations model \cite{SEM}. For the purpose of this paper, two phases are considered and denoted by the indexes $j$ and $k$. In the seven equations model, each phase obeys to the following set of equations $(\eqt{eq:sev_equ-app})$:
\begin{equation}
\label{eq:sev_equ-app}
\left\{
\begin{array}{llll}
\partial_t \left( \alpha_k  A\right) + u_I A \partial_x \alpha_k = A \mu \left( P_k - P_j \right)\\
\partial_t \left( \alpha_k \rho_k A \right) + \partial_x \left( \alpha_k \rho_k u_k A \right) = 0 \\
\partial_t \left( \alpha_k \rho_k u_k A \right) + \partial_x \left[ \alpha_k A \left( \rho_k u_k^2 + P_k \right) \right] = \alpha_k P_k \partial_x A + P_I A \partial_x \alpha_k + A \lambda \left( u_j - u_k \right) \\
\partial_t \left( \alpha_k \rho_k E_k A \right) + \partial_x \left[ \alpha_k A u_k \left( \rho_k E_k + P_k \right) \right] = P_I u_I A \partial_x \alpha_k - \mu \bar{P_I} \left( P_k-P_j \right) + \bar{u_I}A \lambda \left( u_j - u_k \right)
\end{array}
\right.
\end{equation}
where $\rho_k$, $u_k$, $E_k$ and $P_k$ are the density, the velocity, the specific total energy and the pressure of $k^{th}$ phase, respectively. The pressure and velocity relaxation parameters are denoted by $\mu$ and $\lambda$, respectively. The variables with index $_I$ correspond to the interfacial variables and a definition is given in \eqt{eq:sev_equ2-app}. The cross-section $A$ is only function of space: $\partial_t A = 0$. 
\begin{equation}
\label{eq:sev_equ2-app}
\left\{
\begin{array}{lll}
P_I = \bar{P_I} - sgn\left( \partial_x \alpha_k \right) \frac{Z_k Z_j}{Z_k + Z_j} \left( u_k-u_j \right) \\
\bar{P_I} = \frac{Z_k P_j + Z_j P_k}{Z_k + Z_j} \\
v_I = \bar{v_I} - sgn\left( \partial_x \alpha_k \right) \frac{P_k - P_j}{Z_k + Z_j} \\
\bar{v_I} = \frac{Z_k u _k + Z_j u_j}{Z_k + Z_j}
\end{array}
\right.
\end{equation}
where $Z_k = \rho_k c_k$ and $Z_j = \rho_j c_j$ are the impedance of the phase $k$ and $j$, respectively. The speed of sound is denoted by the variable $c$. The function $sgn(x)$ returns the sign of the variable $x$.\\
The first step consists of rearranging the equations given in \eqt{eq:sev_equ2-app} using the primitive variables $(\alpha_k, \rho_k, u_k, e_k)$, where $e_k$ is the specific internal energy of $k^{th}$ phase. We introduce the material derivative $\frac{d \cdot}{dt} = \partial_t \cdot + u_k \partial_x \cdot$ for simplicity. \\
The void fraction is unchanged. The continuity equation is modified as follows:
\begin{equation}
\label{eq:cont1-app}
\alpha_k A \frac{d \rho_k}{dt} + \rho_k A \mu \left( P_k-P_j \right) + \rho_k A \left( u_k-u_j \right) \partial_x \alpha_k + \rho_k \alpha_k \partial_x \left( A u_k \right) = 0
\end{equation}
The momentum and continuity equations are combined to yield the velocity equation:
\begin{equation}
\label{eq:vel1-app}
\alpha_k \rho_k A \frac{du_k}{dt} + \partial_x \left( \alpha_k A P_k \right) = \alpha_k P_k \partial_x A + P_I A \partial_x \alpha_k + A \lambda \left( u_j-u_k \right)
\end{equation}
The internal energy is obtained from the total energy and the kinetic equation ($u_k * $\eqt{eq:vel1-app}):
\begin{eqnarray}
\label{eq:internal1}
\alpha_k \rho_k A \frac{d e_k}{dt} + \partial_x \left( \alpha_k u_k A P_k \right) - u_k \partial_x \left( \alpha_k A P_k \right) =\nonumber \\ \left(u_I-u_k \right) P_I A \partial_x \alpha_k - u_k \alpha_k P_k \partial_x A 
 - \bar{P_I} A \mu \left(P_k-P_j \right) + A \lambda \left(u_j-u_k  \right) \left(\bar{u_I}-u_k \right)
\end{eqnarray}
In the next step, we assume the existence of a phase wise entropy $s_k$ function of the density $\rho_k$ and the internal energy $e_k$. Using the chain rule,
\begin{equation}
\frac{ds}{dt} = s_{\rho_k} \frac{d \rho}{dt} + s_{e_k} \frac{de}{dt},
\end{equation}
along with the internal energy and the continuity equations, the following entropy equation is obtained:
\begin{eqnarray}
\label{eq:ent1}
\alpha_k \rho_k A \frac{ds}{dt} + \underbrace{\alpha_k \left( P_k s_{e_k} + \rho_k^2 s_{\rho_k} \right) \partial_x \alpha_k + \alpha_k u_k \left( P_k s_{e_k} + \rho_k^2 s_{\rho_k} \right) \partial_x A}_\textrm{(a)} = \nonumber\\
s_{e_k} \left[ (u_I-u_k)P_IA \partial_x \alpha_k - \bar{P_I} A \mu (P_k-P_j) + A \lambda (\bar{u_I}-u_k) (u_j-u_k)\right] - \\
\rho^2 s_{\rho_k} \left[ \mu A (P_k-P_j) + A(u_k-u_I) \partial_x \alpha_k\right] \nonumber
\end{eqnarray}
where $s_{e_k}$ and $s_{\rho_k}$ denote the partial derivatives of the entropy $s_k$ with respect to the internal energy $e_k$ and the density $\rho_k$, respectively.
The second term, (a), in the left hand side of \eqt{eq:ent1} can be set to zero by assuming the following relation between the partial derivatives of the entropy $s$;
\begin{equation}
\label{eq:ent2}
 P_k s_{e_k} + \rho_k^2 s_{\rho_k} = 0
\end{equation} 
The above equation is equivalent to the application of the second thermodynamic law when assuming reversibility:
\begin{equation}
T ds = de - \frac{P_k}{\rho_k^2} d \rho_k \text{ with } s_{e_k} = \frac{1}{T_k} \text{ and } s_{\rho_k} = - \frac{P_k}{\rho_k^2} s_{e_k}
\end{equation}
Thus, equation \eqt{eq:ent1} can be rearranged using the relation $s_{\rho_k} = - \frac{P}{\rho^2} s_{e_k}$:
\begin{eqnarray}
\label{eq:ent3}
(s_{e_k})^{-1} \alpha_k \rho_k \frac{ds}{dt} = \underbrace{\left[ (u_I-u_k)P_I + (u_k-u_I)P_k \right] \partial_x \alpha_k}_\textrm{(b)} + \nonumber\\ 
\underbrace{\mu (P_k-P_j)(P_k-\bar{P_I})}_\textrm{(c)} + \underbrace{\lambda(u_j-u_k)(\bar{u_I}-u_k)}_\textrm{(d)}
\end{eqnarray}
The right hand side of equation \eqt{eq:ent3} is split into three terms (b), (c) and (d) that will be treated independently from each other. The terms (c) and (d) are simpler to start with, and can be easily recast by using the definitions of $\bar{u_I}$ and $\bar{P_I}$ given in equation \eqt{eq:sev_equ2-app}:
\begin{eqnarray}
\label{eq:ent4}
\mu (P_k-P_j)(P_k-\bar{P_I}) = \mu \frac{Z_k}{Z_k+Z_j} (P_j - P_k)^2\nonumber\\
\lambda(u_j-u_k)(\bar{u_I}-u_k) = \lambda \frac{Z_j}{Z_k+Z_j} (u_j - u_k)^2 
\end{eqnarray}
By definition, $\mu$, $\lambda$ and $Z_k$ are all positive. Thus, the above terms are unconditionally positive. \\
It remains to look at the last term (b). Once again, by using the definition of $P_I$ and $u_I$, and the following relations:
\begin{eqnarray}
\label{eq:ent4bis}
u_I-u_k &=& \frac{Z_j}{Z_k+Z_j}(u_j-u_k) -  sgn(\partial_x \alpha_k) \frac{Pk-P_j}{Z_k+Z_j} \nonumber\\
P_I-P_k &=& \frac{Z_k}{Z_k+Z_j} (P_j-P_k) - sgn(\partial_x \alpha_k) \frac{Z_k Z_j}{Z_k+Z_j} (u_k-u_j), \nonumber 
\end{eqnarray}
(b) yields:
\begin{eqnarray}
\label{eq:ent5}
\left[ (u_I-u_k)P_I + (u_k-u_I)P_k \right] \partial_x \alpha_k = (u_I-u_k)(P_I-P_k) \partial_x \alpha_k=  \nonumber\\ \frac{\partial_x \alpha_k}{\left( Z_k+Z_j \right)^2} \left[ Z_j Z_k (u_j-u_k)(P_j-P_k)+sgn(\partial_x \alpha_k) Z_k Z_j^2 (u_j-u_k)^2 \right. + \\ \left. sgn(\partial_x \alpha_k)Z_k(P_k-P_j)^2 +  sgn(\partial_x \alpha_k)^2(P_k-P_j)Z_k Z_j (u_k-u_j) \right] \nonumber
\end{eqnarray}
The above equation can be simplified by remarking that $sgn(x)^2 = 1$. We also factorize by $sgn(\partial_x \alpha_k)$ which yields:
\begin{eqnarray}
\label{eq:ent6}
\left[ (u_I-u_k)P_I + (u_k-u_I)P_k \right] \partial_x \alpha_k =  \nonumber\\ sgn(\partial_x \alpha_k) \partial_x \alpha_k\frac{Z_k}{\left( Z_k+Z_j \right)^2} \left[ Z_j (u_j-u_k)+sgn(\partial_x \alpha_k)(P_k-P_j)\right]^2
\end{eqnarray}
The term $sgn(\partial_x \alpha_k) \partial_x \alpha_k$ in \eqt{eq:ent6} can be simplified by noticing that: 
\begin{equation}
\label{eq:ent7}
sgn(\partial_x \alpha_k) \partial_x \alpha_k = \frac{\partial_x \alpha_k}{| \partial_x \alpha_k |} \partial_x \alpha_k = \frac{(\partial_x \alpha_k)^2}{| \partial_x \alpha_k |} = | \partial_x \alpha_k | 
\end{equation}
Thus, using results from \eqt{eq:ent3}, \eqt{eq:ent4}, \eqt{eq:ent5}, \eqt{eq:ent6} and \eqt{eq:ent7}, the entropy equation obtained in \cite{SEM} holds.
\pagebreak{}